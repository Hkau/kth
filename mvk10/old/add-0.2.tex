\documentclass[a4paper, twoside, 11pt, titlepage]{article}

\usepackage{bds/bds}

\usepackage[utf8]{inputenc} % -- använd denna "när det funkar", dvs på skolans nya datorer + linux, ibland på windows
\usepackage[swedish,english]{babel}
\usepackage{placeins}

\project{Bokningssystem för Kårspexet}
\author{
	\small
	Arvidsson, Kalle -- kallear@kth.se\\
	Boström, Peter -- pbos@kth.se\\
	Eklund, Erik -- eekl@kth.se\\
	Gräsman, André -- grasman@kth.se\\
	Göransson, Rasmus -- rasmusgo@kth.se\\
	Hagsten, Per -- hagsten@kth.se\\
	Hallberg, Victor -- victorha@kth.se\\
	Modée, Anna Maria -- ammodee@kth.se\\
	Nyberg, Daniel -- dnyb@kth.se\\
	Stjernberg, Johan -- stjer@kth.se\\
	Tarandi, Andreas -- taran@kth.se
	}

\version{0.2}
\title{Architectural Design Document}

\begin{document}
\maketitle

\clearpage
\thispagestyle{empty}
\mbox{}
\newpage

\selectlanguage{english}
\begin{abstract}
	This document describes the architectural design of Nyx' booking system for Kårspexet. It aims to give Nyx' developers  a more defined picture of what needs to be delivered during the software development phase.
Within, the reader will find a brief overview of the system's architecture, external interfaces, and what methods will be used while implementing the system. A full, highly detailed description is given for all the system's components, from its models, views, and controllers, to its layouts, partials and external components. The project's feasibility is also included, with a likely time plan for the system's development phase. Lastly, a matrix is given that lets the reader trace requirements back to the System Requirements Document.
\end{abstract}
\selectlanguage{swedish}

\newpage

\setcounter{page}{1}

\startfooter

\clearpage
\section*{Ändringslogg}


\begin {table} [ht] \begin{tabular} { p{2.6cm} p{12.5cm} }
	\hline
	{\sffamily\textbf{Version}} & {\sffamily\textbf{Ändringar}} \\
	\hline
	{\sffamily\textbf{0.2}} & {Dokumentformatering, tabeller intakta.} \\
	\hline
	{\sffamily\textbf{0.1}} & {Första sammanställd version av dokumentet.} \\
	\hline
\end{tabular} \end{table} \FloatBarrier


\clearpage
\section*{Dokumentversioner}


Dokumentet har genererats från följande deldokument.

\textbf{ADD/abstract} version: \emph{4}.

\textbf{ADD/Ändringslogg} version: \emph{4}.

\textbf{Gruppmedlemmar} version: \emph{3}.

\textbf{ADD/Introduktion} version: \emph{2}.

\textbf{ADD/Introduktion/Syfte} version: \emph{5}.

\textbf{ADD/Introduktion/Mjukvarans omfattning} version: \emph{3}.

\textbf{ADD/Introduktion/Definitioner akronymer och förkortningar} version: \emph{22}.

\textbf{ADD/Introduktion/Källor} version: \emph{5}.

\textbf{ADD/Introduktion/Dokumentöversikt} version: \emph{9}.

\textbf{ADD/Systemöverblick} version: \emph{8}.

\textbf{ADD/Systemkontext} version: \emph{10}.

\textbf{ADD/Systemdesign} version: \emph{2}.

\textbf{ADD/Systemdesign/Designmetod} version: \emph{23}.

\textbf{ADD/Systemdesign/Uppdelningsbeskrivning} version: \emph{11}.

\textbf{ADD/Komponentbeskrivning} version: \emph{19}.

\textbf{ADD/Komponentbeskrivning/Model} version: \emph{67}.

\textbf{ADD/Komponentbeskrivning/Controller} version: \emph{24}.

\textbf{ADD/Komponentbeskrivning/View} version: \emph{48}.

\textbf{ADD/Komponentbeskrivning/Övriga\_komponenter} version: \emph{19}.

\textbf{ADD/Genomförbarhet- och resursuppskattning} version: \emph{31}.

\textbf{ADD/Spårningsmatris mellan mjukvarukrav och strukturella krav} version: \emph{3}.

\textbf{ADD/appendix} version: \emph{1}.

\clearpage
\section*{Gruppmedlemmar}


Projektgruppen \textbf{Nyx} består av följande medlemmar.

\textbf{Kalle Arvidsson} -- 890601-2490, kallear@kth.se

\textbf{Peter Boström} -- 890224-0814, pbos@kth.se

\textbf{Erik Eklund} -- 880816-0454, eekl@kth.se

\textbf{André Gräsman} -- 890430-3214, grasman@kth.se

\textbf{Rasmus Göransson} -- 850908-8517, rasmusgo@kth.se

\textbf{Per Hagsten} -- 870529-0115, hagsten@kth.se

\textbf{Victor Hallberg} -- 890121-0057, victorha@kth.se

\textbf{Anna Maria Modée} -- 871120-0363, ammodee@kth.se

\textbf{Daniel Nyberg} -- 900104-4495, dnyb@kth.se

\textbf{Johan Stjernberg} -- 890315-0533, stjer@kth.se

\textbf{Andreas Tarandi} -- 890416-0317, taran@kth.se

\clearpage \tableofcontents \clearpage

\clearpage
\section{Introduktion}



	\subsection{Syfte}


	Dokumentets syfte är att specificera detaljer kring produktens arkitektur. Den är speciellt skriven för att underlätta arbetet inom Nyx, men skall också kunna läsas av de som skall underhålla produkten åt kunden Kårspexet. Dokumentet redogör för vilka komponenter som finns, systemdesignen samt en uppskattning av Nyx resurser och genomförbarhet.

	\subsection{Mjukvarans omfattning}


	Produkten som Nyx utvecklar består av ett webbaserat biljettbokningssystem med ett enkelt användargränssnitt för besökare och administrationsverktyg för Kårspexets personal. Administrationsverktygen består av tre gränssnitt; ett för säljare, ett för ekonomiansvariga och ett för administratörer.

	\subsection{Definitioner akronymer och förkortningar}


	VAR SNÄLL OCH TA BORT STRÄCKET FRAMFÖR AKRONYMER OM DU ANVÄNDER DIG AV DEM.

	\textbf{Action} (Rails action) En metod/funktion tillhörande en kontroll i MVC arkitekturen.\_

	\textbf{Active record} \emph{Teknik för att kommunicera med databaser i objekt-orienterade språk. Objekten i databasen kopplas till objekt i programmet.}

	- \textbf{Algoritm} \emph{Inom matematik och datorvetenskap är detta en begränsad uppsättning tydliga instruktioner för att utföra en uppgift.}

	\textbf{Apache} \emph{Syftar i detta dokument på webbservern Apache HTTP Server.}

	\textbf{Apache HTTP Server} \emph{Världens mest använda webbserver. Är gratis att använda. [1.3.1]}

	- \textbf{Arbetsminne} \emph{Även kallat primärminne. En fysisk komponent i en dator. Används för att lagra program och data då programmet körs.}

	- \textbf{Bandbredd} \emph{I vardagligt tal en storhet för hur mycket information som kan överföras på en viss tid. Vanlig enhet är Mbit/sekund.}

	- \textbf{Bit} (Binary Digit) \emph{Den grundläggande enhet som datorer arbetar med. En bit kan anta ett utav två möjliga värden (ofta angivna som 0 eller 1).}

	- \textbf{Byte} \emph{En vanlig enhet för informationsmängd i datasammanhang. En byte är ett paket bestående av åtta bitar.}

	- \textbf{CentOS} \emph{Ett operativsystem baserat på Red Hat Enterprise Linux som är gratis att använda. [1.3.2]}

	\textbf{Databas} \emph{En databas är en samling information ordnad på ett sådant sätt att informationen i den effektivt går att hitta.}

	- \textbf{Firefox} (Mozilla Firefox) \emph{En webbläsare som går att köra på de populäraste operativsystemen.}

	- \textbf{Foreign key} \emph{Inom relationsdatabaser en begränsning som kräver att information på ett ställe finns definierat tidigare på ett annat ställe (kolumner i tabeller).}

	\textbf{Gem} \emph{Term för Rails-bibliotek som ger utökad funktionalitet.}

	- \textbf{GHz} \emph{Enhet för antalet miljarder svängningar per sekund. ``G'' är binärt prefix för $10^{9}$. ``Hz'' är förkortning för Hertz.}

	\textbf{Gränssnitt} \emph{Utformningen av kommunikationen mellan en mjukvarumodul och användare eller annan mjuk-/hårdvara.}

	\textbf{GUI} (Graphical User Interface) \emph{Se gränssnitt.}

	- \textbf{HTML} (Hyper Text Markup Language) \emph{Ett språk och webbstandard som används för att beskriva strukturering av text, bilder och annan media på en webbsida.}

	\textbf{HTTP} (HyperText Transfer Protocol) \emph{Ett standardiserat protokoll som definierar hur kommunikation över webben sker.}

	- \textbf{HTTPS} (HTTP Secure) \emph{En kombination av HTTP och SSL/TLS med syftet att förhindra avlyssning av HTTP-trafiken}

	- \textbf{Hårdvara} \emph{Även kallat Maskinvara. Ett samlingsnamn för en dators fysiska komponenter.}

	- \textbf{Internet Explorer} \emph{En webbläsare utvecklad av Microsoft för operativsystemet Windows.}

	- \textbf{InvalidAuthenticyToken} \emph{Ett fel som kan uppstå i rails om användaren backar på sidor innehållande formulär. Uppstår på grund av Rails skydd mot XSS.}

	- \textbf{KiB} (kibibyte) \emph{$2^{10}$ byte.}

	\textbf{Kontroll} (Railscontroller) \emph{En komponent i MVC principen som tar emot händelser/inmatningar från användaren och utför anrop till en modell. Kontrollen instruerar modellen och vyn hur de skall agera utifrån användarens inmatningar.}

	- \textbf{Linux} \emph{Unix-liknande operativsystem. Linux är fri mjukvara.}

	- \textbf{MiB} (mebibyte) \emph{$2^{20}$ byte.}

	- \textbf{Mib} (mebibit) \emph{$2^{20}$ bit.}

	- \textbf{Mjukvara} \emph{Även kallat programvara. En organiserad samling av data och maskininstruktioner.}

	\textbf{Modell} (Railsmodel) \emph{En datastruktur i MVC arkitekturen som laddar, sparar och hanterar data genom att vanligtvis arbete mot en databas. Modellen kan innehålla grundläggande logik för att opererea på dadan.}

	\textbf{Model-View-Controller} \emph{Ett koncept som bygger på att separera data (modeller), logik (kontroller) och användarinterface (vyer).}

	\textbf{MVC} (Model-View-Controller) \emph{Se Model-View-Controller.}

	\textbf{MySQL} \emph{En typ av relationsdatabas baserad på SQL-standarden. Ett relationsdatabas hanteringssystem där flera användare kan arbeta med flera databaser.}

	\textbf{Partial} (Railspartial) \emph{En komponent som tillåter en vy i MVC principen att dela upp en renderingsprocess i flera filer. En vy kan bestå av flera partial som var och en har en avgränsad uppgift.}

	- \textbf{Passenger} \emph{I Rails-sammanhang en modul som gör det möjligt att köra Ruby on Rails på webbservern Apache.}

	- \textbf{Processor} \emph{Den komponent i en dator som utför beräkningar efter instruktioner.}

	- \textbf{RDoc} (Ruby Doc) \emph{Verktyg för att generera dokumentation för Ruby-källkod i HTML-format.}

	\textbf{Router} (Railsrouter) \emph{En komponent som identifierar URLer och anropar rätt action på en kontroller. Routern kan generera URLer för länkar och knappar i vyer så hårdkodade URLer kan undvikas.}

	\textbf{Ruby} \emph{Ett objektorienterat programmeringsspråk.}

	\textbf{Rails} (Ruby On Rails) \emph{Ett abstrakt mjukvarubibliotek med öppen källkod för utveckling av webbapplikationer.}

	- \textbf{Sjöslaget} \emph{Årligen återkommande studentfest på Finlandsfärja.}

	\textbf{SQL} (Structured Query Language) \emph{Ett språk designat för att interagera med databaser.}

	- \textbf{SQL-injection} \emph{En metod för att förändra eller komma åt data i en databas genom att ange strängar i användarinterfacet som förändrar betydelsen av en SQL-fråga.}

	- \textbf{SHA-1} \emph{Ett sätt att spara strängar så de inte står i klartext. Det går inte att återskapa strängen efter omkodning.}

	- \textbf{SSL/TLS} (Secure Socket Layer/Transport Layer Security) \emph{Ett kryptografiskt protokoll för att sätta upp säkra kommunikationskanaler över internet.}

	- \textbf{Testkod} \emph{Kod som används för att testa funktioner i programmet så att det returnerar förväntat svar för att försäkra sig om att mjukvaran fungerar på ett tillfredsställande sätt.}

	- \textbf{Tutorial} \emph{En metod för att överföra kunskap som ofta används vid inlärning.}

	- \textbf{URD} (User Requirements Document) \emph{Dokument inom PSS050 standarden där användarens krav specificeras.}

	\textbf{URL} (Uniform Resource Locator) \emph{Den formella benämningen av en webbadress. En text som beskriver var en viss resurs på internet finns, samt hur den går att komma åt.}

	\textbf{Vy} (Railsview) \emph{En komponent i MVC principen som renderar innehållet från en modell till ett interaktivbart användargränssnitt. Flera vyer är ofta kopplade till samma modell fast med olika syften.}

	- \textbf{Webbapplikation} \emph{Samlingsnamn för mjukvara som användare kommer åt via en webbläsare.}

	\textbf{Webbläsare} \emph{Ett program som hämtar, tolkar och återger webbsidor kodade exempelvis som HTML.}

	\textbf{Webbserver} \emph{Program som körs på en server och distribuerar webbsidor och/eller andra filer som en webbläsare begär via HTTP-protokollet.}

	- \textbf{Webbsida} \emph{En fil, innehållandes exempelvis HTML, avsedd att visas av en webbläsare.}

	- \textbf{XSS} (Cross site scripting) \emph{En teknik som utnyttjar svagheter i en webbsida genom att låta en auktoriserad användare accessa en länk som modifierar sidan på ett sätt som den auktoriserade användaren inte önskar [1.3.3].}

	\subsection{Källor}


	Referenser till de källor som använts i dokumentet är listade här under. En och samma källa kan refereras vid flera ställen i texten. En referens är på formatet [Sektion.Rubrik.Löpnummer]. Exempelvis är [4.2.1] den första (1) referensen för rubriken ``Uppdelningsbeskrivning'' (2) under sektion ``Systemdesign'' (4).

	\textbf{Apache HTTP Server}

	\url{http://httpd.apache.org/}

	Hänvisning till källan görs från referenserna: [1.3.1].

	\textbf{CentOS}

	\url{http://www.centos.org/}

	Hänvisning till källan görs från referenserna: [1.3.2].

	\textbf{Cross-site-scripting}

	\url{http://en.wikipedia.org/wiki/Cross-site_scripting}

	Hänvisning till källan görs från referenserna: [1.3.3].

	\textbf{Ruby on Rails Guides: Getting Started with Rails}

	\url{http://guides.rubyonrails.org/getting_started.html}

	Hänvisning till källan görs från referenserna: [4.2.1].

	\subsection{Dokumentöversikt}


	\emph{1.5. Overview of the document. Similar to SRD Section 1.5, but describes the ADD.}

	\emph{Again it need not be assumed that readership on the customer side exists. In practise, this}

	\emph{document may again be company confidential to the development team.}

	Detta dokument inleds med en kort introduktion. Efter denna följer sektion 2 som ger en överblick av hela systemet. Detta bör vara utgångspunkten för läsare som inte redan är bekanta med Nyx system.

	I sektion 3 beskrivs systemets gränssnitt mot omgivningen. Sektion 4 beskriver systemets interna design, som sedan gås igenom i detalj i sektion 5 där alla delkomponenter beskrivs.

	I sektion 6 görs en bedömning av projektets genomförbarhet utifrån uppskattningar av behov och tillgänglighet av resurser, främst arbetstid för t ex programmering och testning. Detta inkluderar en riskbedömning.

	Sektion 7 kopplar krav från SRD:n till ADD:n. För varje krav listas de komponenter vars direkta syfte är att uppfylla kravet.

\clearpage
\section{Systemöverblick}


\emph{2 System Overview. Summarises: (i) the system context (how it fits into an existing}

\emph{framework of other packages and systems), and (ii) the system design. More detailed}

\emph{descriptions of (i) and (ii) are given in Sections 3 and 4 below.}

Nyx biljettbokningssystem kommer att ersätta det biljettbokningssystemet som för närvarande används av Kårspexet. Det innebär att vårt system kommer användas på Kårspexets webbplats och då vi endast utvecklar biljettbokningssystemet kommer vårt system behöva passa in på den befintliga webbplatsen. Dock ska systemet inte interagera med det befintliga systemet, annat än med länkar och därmed behöver vi bara anpassa vårt system visuellt.

Externa system som kommer användas av vårt system är MySQL, för att få tillgång till en databas, samt Apache HTTP Server, för att hantera HTTP-kommunikationen med användarens webbläsare.

Nyx valde att utveckla applikationen i dessa system eftersom vi hade mycket positiv erfarenhet av dessa inom gruppen. Dessutom så arbetar Ruby on Rails väldigt väl mot MySQL så vi får därmed mycket funktionalitet på köpet när vi använder dessa tillsammans. Anledningen till att Apache valdes var för att det är välanvänt, väldokumenterat samt att det är öppen källkod vilket underlättar det för oss.

\clearpage
\section{Systemkontext}


\emph{3 System Context. Gives a detailed description of the system context, with relevant}

\emph{diagrams. Defines the external interfaces of the product under development to these other}

\emph{systems.}

\emph{3.n External interface definition. Provides an interface definition to each separate}

\emph{external component type or physical component.}

\begin{figure}[ht] \centering \includegraphics[width=0.8\textwidth]{files/Diagram.pdf} \end{figure} \FloatBarrier

Systemet är designat för att köras på en webbserver där alla systemets komponenter befinner sig internt  på servern, som används via ett webbläsarfönster. Apache anropas som i sin tur kör igång Ruby som i förväg har laddat in Rails biblioteket, applikationen och övriga komponenter. Där behandlas anropet och Rails kommer att kommunicera med MySQL och hämtar relaterad data. Rails förbereder sedan resultatet och skickar tillbaka detta till Apache som vidarebefordrar det till klienten som i sin tur presenterar innehållet i webbläsaren.

	\subsection{MySQL}


	Applikationen kommer använda sig av en MySQL databas för att hantera data för det olika användargrupperna. Databasen anropas och relevant data läses in och modifieras från Ruby applikationens olika vyer.

	\subsection{Apache HTTP Server}


	Servern kommer att köra Apache HTTP Server för att sköta kommunikationen mellan servern och applikationen. När en användare ansluter tar Apache emot anropet och skickar det vidare till applikationen. Apache kommer även ta emot resultat från applikationen och vidarebefordrar det till användaren.

\clearpage
\section{Systemdesign}


\emph{4. System Design. Provides an overview of the design techniques used, especially any in-}

\emph{house or non-standard methods, project specific methods, or non-standard interpretation}

\emph{of standard languages/methods such as UML.}

	\subsection{Designmetod}



		\subsubsection{Model-View-Controller}


		Nyx har valt att följa designprincipen Model-View-Controller för utveckling av systemet. MVC metoden är en metod för att separera modeller, logik och det visuella interfacet i olika komponenter. Modellerna (\emph{Models}) hanterar och lagrar data som är relevant för systemet. De ser även till att ingen ogiltlig data sparas i databasen. Kontroller (\emph{Controllers}) sköter all logik och är ett mellanlager mellan modellerna och interfacet användaren ser. Vyerna (\emph{Views}) i sin tur renderar det interface som användaren ser och sköter postning tillbaka till kontrollerna.

		\subsubsection{Komponentbeskrivningar}


		Komponentbeskrivningarna under punkt fem (5) är upplagda enligt följande format.

		Alla komponenter identifieras av 5.Tn där T är någon av bokstäverna MVCLPX, som står för vilken typ komponenten är, och n är ett sekventiellt tal från ett (1) och uppåt.

		I vissa fall används noteringen \emph{@namn} för variabelnamn. Detta indikerar att variabeln är en instansvariabel.

			\paragraph{Komponentrubriker}\

			Alla komponenter har en tabell med följande innehåll.

			\begin {table} [ht] \begin{tabular} { p{2.6cm} p{12.5cm} }
				\hline
				{Typ} & {Kontroll/Modell/Layout/Vy/Partiell vy/Övrig komponent} \\
				\hline
				{Syfte} & {Specificerar vilket syfte komponenten fyller. Referenser till SR-krav.} \\
				\hline
				{Funktion} & {Anger vilka olika funktioner komponenten bidrar med.} \\
				\hline
				{Delkomponenter} & {Eventuellt andra komponenter som är en del av/ingår i denna.} \\
				\hline
				{Beroenden} & {Krav för användandet av denna komponent.} \\
				\hline
				{Gränssnitt} & {Publika metoder som går att anropa. Inkluderar metodnamn och kortare beskrivning av vad metoden gör. Om delar av gränssnittet kräver att användaren är inloggad/admin/accountant/sales ska detta specificeras här.} \\
				\hline
				{Resurser} & {Vilka resurser (modeller/klasser/gems) som komponenten utnyttjar direkt (ej indirekt).} \\
				\hline
				{Källor} & {Referenser till information om eventuellt använda gem och andra externa komponenter.} \\
				\hline
				{Process} & {Lista möjliga arbetsflöden (övergångar).} \\
				\hline
				{Data} & {Tillgängliga instansvariabler för/som används av komponenten (de som är relevanta från ett externt perspektiv).} \\
				\hline
			\end{tabular} \end{table} \FloatBarrier


			\paragraph{Komponenttyper}\


			\subparagraph{\emph{M - Model (\emph{Modell})}}\

				\emph{Beskriver en modell i MVC-modellen.}

				\textbf{Specifieringar och tillägg till komponentrubrikerna från 4.1.2}

				\begin {table} [ht] \begin{tabular} { p{2.6cm} p{12.5cm} }
					\hline
					{Syfte} & {Beskriver hur ett objekt av typen fungerar samt vilka begränsningar som ställs.} \\
					\hline
					{Delkomponenter} & {Vilka relationer till andra modeller som finns (motsvarande Rails has\_many- och belongs\_to-nyckelord).} \\
					\hline
					{Beroenden} & {Krav för att objekt av denna modell ska vara giltiga (valideringskrav), inklusive relationer.} \\
					\hline
					{Gränssnitt} & {Av Nyx definierade publika metoder som går att anropa och som antingen returnerar information (utöver standard-getters/setters) eller påverkar objektet, ej standardmetoder från ActiveRecord:Base. Inkluderar metodnamn, eventuella parametrar respektive returdata samt kortare beskrivning av vad metoden gör.} \\
					\hline
					{Resurser} & {ActiveRecord::Base samt de eventuella resurser (modeller/gem) som modellen utnyttjar direkt (ej indirekt).} \\
					\hline
					{Källor} & {Referenser till information om eventuellt använda gem, externa bibliotek, etc.} \\
					\hline
					{Process} & {Eventuella valideringssteg eller andra processer för manipulation av objektet (inkluderar ej skapande/sparande/borttagning av objekt).} \\
					\hline
					{Data} & {Lista med namn, typ och beskrivning av modellens samtliga attribut. Attributnamn följer lowercase\_with\_underscore-namngivning.} \\
					\hline
				\end{tabular} \end{table} \FloatBarrier


			\subparagraph{\emph{C - Controller (\emph{Kontroll})}}\

				\emph{Beskriver en kontroller i MVC-modellen.}

				\textbf{Specificeringar och tillägg till komponentrubrikerna från 4.1.2}

				\begin {table} [ht] \begin{tabular} { p{2.6cm} p{12.5cm} }
					\hline
					{Delkomponenter} & {I de flesta fall ej applicerbart.} \\
					\hline
					{Gränssnitt} & {Listar och beskriver alla actions. Inkluderar vilka anropsparametrar ({\tt params[]}) som utnyttjas. Actions som kräver att användaren är inloggad eller innehar specifika roller (admin/accountant/sales) nämner detta här.} \\
					\hline
					{Resurser} & {ApplicationController samt de eventuella resurser (modeller/gem) som kontrollern utnyttjar direkt (ej indirekt).} \\
					\hline
					{Källor} & {Referenser till information om eventuellt använda gem, externa bibliotek, etc.} \\
					\hline
					{Process} & {Listar möjliga arbetsflöden (övergångar) mellan actions (new > create, edit > update, index > show/destroy, etc.).} \\
					\hline
					{Data} & {Instansvariabler som görs tillgängliga för vyerna och som är relevanta från ett externt perspektiv.} \\
					\hline
				\end{tabular} \end{table} \FloatBarrier


			\subparagraph{\emph{V - View (\emph{Vy})}}\

				\emph{Beskriver en vy i MVC-modellen.}

			\subparagraph{\emph{L - Layout}}\

				\emph{Beskriver en layout.}

				En layout i Rails är en mall för hur en samling av sidor ska se ut. Den innehåller kod som annars skulle behöva finnas i vyerna.

			\subparagraph{\emph{P - Partial (\emph{Partiell vy})}}\

				\emph{Beskriver en partiell vy.}

				En partiell vy innehåller kod som är gemensam för flera vyer. Vyer kan inkludera partiella vyer och därmed slipper man kodupprepning.

			\subparagraph{\emph{X - Övriga/externa komponenter}}\

				\emph{Beskriver övriga och externa komponenter.}

				Komponenter som beskrivs här kan vara logiska komponenter (exempelvis inloggningssystemet) eller externa komponenter (exempelvis Ruby-gem).

	\subsection{Uppdelningsbeskrivning}


	\emph{4.2 Decomposition description. Gives the top level view of the systems design,}

	\emph{preferably with diagrams. Shows the major components which will be described in detail}

	\emph{in Section 5. Identifies control and data flow between components.}

	Systemet består av tre komponenttyper enligt MVC-modellen. [4.2.1]

	>> Modeller

	>> Vyer

	>> Kontroller

	\begin{figure}[ht] \centering \includegraphics[width=0.8\textwidth]{files/mvc.png} \end{figure} \FloatBarrier

	\^ misplaced image? how do I make this good? -- Peter

		\subsubsection{Modeller}


		Modeller representerar den data som applikationen använder sig av och består av och innehåller regler för hur denna data får manipuleras. I Nyx fall används modellerna främst för att bestämma hur tabeller i databasen ska interageras med. Det är här större delen av logiken i applikationen ligger.

		Bokningssystemet har följande typer av modeller:

		>> Administrativa modeller

		>> Bokningar

		>> Föreställningar

		\subsubsection{Vyer}


		Vyer representerar användargränssnittet till applikationen. Detta är olika webbsidor som visar representationer av data som finns i applikationen. De hanterar även förfrågningar som görs till systemet.

		Bokningssystemet har följande typer av vyer:

		>> Administration

		>> Ekonomi

		>> Kund

		>> Login

		>> Säljare

		\subsubsection{Kontroller}


		Kontroller kopplar samman modeller och vyer. I vårt fall är kontrollerna ansvariga för att hantera de inkommande förfrågningarna som kommer från användares webbläsare. De gör förfrågningar vidare till modellerna för data och skickar vidare till vyerna för att kunna presenteras för användaren.

		Bokningssystemet har följande typer av kontroller:

		>> Administration

		>> Ekonomiansvarig

		>> Försäljning

		>> Kund

\clearpage
\section{Komponentbeskrivning}


\emph{5. Component Description.}

\emph{Gives detailed component information according to a fixed template. Components may}

\emph{be top level components, identified in Section 4.2, or subcomponents of these. Preferably}

\emph{use a component identification scheme which is easy to read/follow/remember.}

\emph{5.n. [Component identifier] Fill in name here.}

\emph{5.n.1. Type. Could be a module, an input/output/temporary file, a program, a class, a}

\emph{script, a web page, etc.}

\emph{5.n.2. Purpose. Describe the purpose of the component, and relate this to a numbered}

\emph{software requirement in the SRD.}

\emph{5.n.3. Function. Describe the functionality of the component, including its interface}

\emph{properties (call and return types) and logical behaviour.}

\emph{5.n.4. Subordinates. List the immediate subcomponents of the component, using defined}

\emph{component identifiers.}

\emph{5.n.5. Dependencies. Describe the logical preconditions for using this component, e.g.}

\emph{files and/or objects that must exist.}

\emph{5.n.6. Interfaces. Define the control and data flow to and from the object. Gives a}

\emph{detailed picture of its context in the overall system architecture.}

\emph{5.n.7. Resources. List any resources required by the component, such as external}

\emph{components external subsystems, hardware, etc.}

\emph{5.n.8. References. Reference any external documents needed to understand the}

\emph{component.}

\emph{5.n.9. Processing. Describe the control and data flow betwen immediate subcomponents}

\emph{of this component. If the component has no immediate subcomponents (i.e. it is fully}

\emph{decomposed) then outline the method of processing used by the component to perform its}

\emph{task (e.g. with pseudo-code, state diagrams, etc).}

\emph{5.n.10. Data. Describe in detail (where possible) the local data values and data structures}

\emph{belonging to (local in scope) this component. Otherwise give an outline description.}

Komponenterna är nummrerade med en bokstav och en siffra: 5.Xn

Där X är en bokstav som representerar vilken typ av komponent det är och n är ett löpnummer som är unikt för den typen av komponenter. Bokstäverna som används är:

>> M - Model (Modell)

>> C - Controller (Kontroll)

>> V - View (Vy)

>> L - Layout

>> P - Partial (Partiell vy)

>> X - Övriga/externa komponenter

	\subsection{Modeller}


	\emph{TODO skriv om följande stycke så att varje mening inte börjar med 'modellerna'.}

	\emph{Se även designmetod, det mesta står även där, kanske ska detta stycke tas bort.}

	\emph{TODO fixa funktion.}

	Modellerna sparar och hämtar data från databasen samt håller data temporärt. De flesta har en koppling till databasen genom att ärva från ActiveRecord::Base, se ActiveRecord (5.X4). Grundläggande logik finns för att se till att databasen inte innehåller fel.

	Funktion har utelämnats i de flesta fall eftersom den oftast beskrivs av syftet. För varje modell beskrivs relationerna till andra modeller under delkomponenter. Modellernas data inkluderar, förutom vad som specificeras för respektive modell, även det som specificeras i ActiveRecord.

	SR1.29 (användande av databas) implementeras av alla modeller tillsammans.

	Skapande av databasschema och ingår i modellernas implementation.


		\subsubsection{Bokning}



			\paragraph{5.M1 Reservation}\

			\begin {table} [ht] \begin{tabular} {  p{3.5cm} p{9.6cm} }
				\hline
				{Typ} & {Modell} \\
				\hline
				{Syfte} & {Hålla information om en bokning (SR1.2, SR1.3).} \\
				\hline
				{Funktion} & { } \\
				\hline
				{Delkomponenter} & {Tillhör en Show.

Har flera Placements.

Har flera ReservationCounters.} \\
				\hline
				{Gränssnitt} & {Ärver metoder från ActiveRecord::Base, se ActiveRecord.} \\
				\hline
				{Beroenden} & {Måste referera till existerande Show.

Måste ha minst en ReservationCounter.

Om inte {\tt creator} existerar måste {\tt customer\_reservation} vara giltig.} \\
				\hline
				{Resurser} & {grouped\_validations för att hantera valideringar} \\
				\hline
				{Källor} & {grouped\_validations: \url{http://rubydoc.info/gems/grouped_validations/0.2.2/file/README.rdoc}} \\
				\hline
				{Process} & {Valideringsgrupp {\tt customer\_reservation}

{\tt email} måste vara giltig.

{\tt adress} måste existera.

{\tt post\_code} måste vara giltig.

{\tt post\_town} måste vara giltig.

{\tt paymentoption} måste finnas.

{\tt deliverymethod} måste finnas.} \\
				\hline
				{Data} & {{\tt name : string}

{\tt phone : string}

{\tt email : string}

{\tt adress : string}

{\tt post\_code : decimal}

{\tt post\_town : string}

{\tt comment : text}

{\tt paymentoption : {plusgiro, kontant, kort} (integer)}

{\tt deliverymethod : {brev, uthämtning} (integer)}

{\tt paid : boolean}

{\tt paid\_sum : decimal}

{\tt placed : boolean}

{\tt retrieved : boolean}

{\tt cost : decimal}

{\tt hash\_key : string} - identifierar bokningen vid avbokning.

{\tt creator : integer}} \\
				\hline
			\end{tabular} \end{table} \FloatBarrier


			\paragraph{5.M2 ReservationCounter}\

			\begin {table} [ht] \begin{tabular} {  p{3.5cm} p{9.6cm} }
				\hline
				{Typ} & {Modell} \\
				\hline
				{Syfte} & {Håller information om antal platser i en viss sektion, med ett visst pris för en bokning (SR1.2).} \\
				\hline
				{Funktion} & { } \\
				\hline
				{Delkomponenter} & {Tillhör en Reservation.

Tillhör en Pricing.

Tillhör en Section.} \\
				\hline
				{Gränssnitt} & {Ärver metoder från ActiveRecord::Base, se ActiveRecord.} \\
				\hline
				{Beroenden} & {Måste ha existerande Reservation, Pricing och Section, samt Antal > 0.} \\
				\hline
				{Resurser} & { } \\
				\hline
				{Källor} & { } \\
				\hline
				{Process} & { } \\
				\hline
				{Data} & {{\tt antal: decimal}} \\
				\hline
			\end{tabular} \end{table} \FloatBarrier


			\paragraph{5.M3 Placement}\

			\begin {table} [ht] \begin{tabular} {  p{3.5cm} p{9.6cm} }
				\hline
				{Typ} & {Modell} \\
				\hline
				{Syfte} & {En boknings placeringar (sittplatser) (SR1.5, SR1.11).} \\
				\hline
				{Funktion} & { } \\
				\hline
				{Delkomponenter} & {Tillhör en Reservation.

Tillhör en Seat.

Tillhör en Show.

Tillhör en User.} \\
				\hline
				{Gränssnitt} & {Ärver metoder från ActiveRecord::Base, se ActiveRecord.} \\
				\hline
				{Beroenden} & {Måste ha existerande Reservation, Show och Seat} \\
				\hline
				{Resurser} & { } \\
				\hline
				{Källor} & { } \\
				\hline
				{Process} & { } \\
				\hline
				{Data} & { } \\
				\hline
			\end{tabular} \end{table} \FloatBarrier


		\subsubsection{Föreställningar}



			\paragraph{5.M4 Theater}\

			\begin {table} [ht] \begin{tabular} {  p{3.5cm} p{9.6cm} }
				\hline
				{Typ} & {Modell} \\
				\hline
				{Syfte} & {Modell för teater (SR1.9, SR1.22).} \\
				\hline
				{Funktion} & { } \\
				\hline
				{Delkomponenter} & {Har flera Sections.

Har flera Seats, genom Sections.

Har flera Batches.

Har flera Shows, genom Batches.} \\
				\hline
				{Gränssnitt} & {Ärver metoder från ActiveRecord::Base, se ActiveRecord.} \\
				\hline
				{Beroenden} & {Måste ha internalname, publicname och picture.} \\
				\hline
				{Resurser} & { } \\
				\hline
				{Källor} & { } \\
				\hline
				{Process} & { } \\
				\hline
				{Data} & {{\tt internal\_name : string}

{\tt public\_name : string}

{\tt description : string}

{\tt comment : string}

{\tt picture : string} - bild eller referens till bild (?) för att visa sektioner och sittplatser.} \\
				\hline
			\end{tabular} \end{table} \FloatBarrier


			\paragraph{5.M5 Section}\

			\begin {table} [ht] \begin{tabular} {  p{3.5cm} p{9.6cm} }
				\hline
				{Typ} & {Modell} \\
				\hline
				{Syfte} & {Modell för sektion (SR1.9, SR1.22, SR1.23, SR1.24).} \\
				\hline
				{Funktion} & { } \\
				\hline
				{Delkomponenter} & {Tillhör en Theater.

Har flera Seats.} \\
				\hline
				{Gränssnitt} & {Ärver metoder från ActiveRecord::Base, se ActiveRecord.} \\
				\hline
				{Beroenden} & {Måste tillhöra en existerande Theater} \\
				\hline
				{Resurser} & { } \\
				\hline
				{Källor} & { } \\
				\hline
				{Process} & { } \\
				\hline
				{Data} & { } \\
				\hline
			\end{tabular} \end{table} \FloatBarrier


			\paragraph{5.M6 Seat}\

			\begin {table} [ht] \begin{tabular} {  p{3.5cm} p{9.6cm} }
				\hline
				{Typ} & {Modell} \\
				\hline
				{Syfte} & {Modell för sittplats i en teaters sektion (SR1.9, SR1.25).} \\
				\hline
				{Funktion} & { } \\
				\hline
				{Delkomponenter} & {Tillhör en Section.

Har flera Placements.

Har flera Placement\_locks.} \\
				\hline
				{Gränssnitt} & {Ärver metoder från ActiveRecord::Base, se ActiveRecord.} \\
				\hline
				{Beroenden} & {Måste referera existerande Section} \\
				\hline
				{Resurser} & { } \\
				\hline
				{Källor} & { } \\
				\hline
				{Process} & { } \\
				\hline
				{Data} & {{\tt number : integer} - stolsnummer

{\tt row : integer} (x,y)-koordinat i teaterns bild eller dylikt.} \\
				\hline
			\end{tabular} \end{table} \FloatBarrier


			\paragraph{5.M7 Batch}\

			\begin {table} [ht] \begin{tabular} {  p{3.5cm} p{9.6cm} }
				\hline
				{Typ} & {Modell} \\
				\hline
				{Syfte} & {Håller information om en omgång (SR1.10).} \\
				\hline
				{Funktion} & { } \\
				\hline
				{Delkomponenter} & {Tillhör en Theater.

Har flera Shows.} \\
				\hline
				{Gränssnitt} & {Ärver metoder från ActiveRecord::Base, se ActiveRecord.} \\
				\hline
				{Beroenden} & {Måste referera existerande Theater

description får inte vara tom.

Synlighet och startdatum måste finnas.} \\
				\hline
				{Resurser} & { } \\
				\hline
				{Källor} & { } \\
				\hline
				{Process} & { } \\
				\hline
				{Data} & {{\tt description : text}

{\tt visible : boolean} - Synlighet, huruvida kunden kan se och boka biljetter till omgången.

{\tt start\_date}} \\
				\hline
			\end{tabular} \end{table} \FloatBarrier


			\paragraph{5.M8 Show}\

			\begin {table} [ht] \begin{tabular} {  p{3.5cm} p{9.6cm} }
				\hline
				{Typ} & {Modell} \\
				\hline
				{Syfte} & {Data för en enskild föreställning (SR1.10).} \\
				\hline
				{Funktion} & { } \\
				\hline
				{Delkomponenter} & {Tillhör en Batch.

Tillhör en Theater, genom Batch.} \\
				\hline
				{Gränssnitt} & {Ärver metoder från ActiveRecord::Base, se ActiveRecord.} \\
				\hline
				{Beroenden} & {Måste referera existerande Batch.} \\
				\hline
				{Resurser} & { } \\
				\hline
				{Källor} & { } \\
				\hline
				{Process} & { } \\
				\hline
				{Data} & {{\tt datetime : datetime}} \\
				\hline
			\end{tabular} \end{table} \FloatBarrier


			\paragraph{5.M9 Pricing}\

			\begin {table} [ht] \begin{tabular} {  p{3.5cm} p{9.6cm} }
				\hline
				{Typ} & {Modell} \\
				\hline
				{Syfte} & {I varje Batch har varje Section i teatern normalpris och studentpris. Detta sköts av Pricing-modellen (SR1.30).} \\
				\hline
				{Funktion} & { } \\
				\hline
				{Delkomponenter} & {Tillhör en Batch.

Tillhör en Section.} \\
				\hline
				{Gränssnitt} & {Ärver metoder från ActiveRecord::Base, se ActiveRecord.} \\
				\hline
				{Beroenden} & {Referenser till Batch och Section måste vara giltiga.

{\tt normal\_price} och {\tt student\_price} måste vara icke-negativa heltal.} \\
				\hline
				{Resurser} & { } \\
				\hline
				{Källor} & { } \\
				\hline
				{Process} & { } \\
				\hline
				{Data} & {{\tt normal\_price}

{\tt student\_price}} \\
				\hline
			\end{tabular} \end{table} \FloatBarrier


		\subsubsection{Administrativa och övriga modeller}



			\paragraph{5.M10 User}\

			\begin {table} [ht] \begin{tabular} {  p{3.5cm} p{9.6cm} }
				\hline
				{Typ} & {Modell} \\
				\hline
				{Syfte} & {För att hålla koll på de olika roller användare kan logga in som (SR1.7, SR9.1, SR9.2).} \\
				\hline
				{Funktion} & { } \\
				\hline
				{Delkomponenter} & {Tillhör en Ability.} \\
				\hline
				{Gränssnitt} & {Ärver metoder från ActiveRecord::Base, se ActiveRecord.} \\
				\hline
				{Beroenden} & {{\tt username} måste vara unikt.

{\tt password} måste finnas.} \\
				\hline
				{Resurser} & { } \\
				\hline
				{Källor} & { } \\
				\hline
				{Process} & { } \\
				\hline
				{Data} & {{\tt username : string}

{\tt password : string}

{\tt valid\_to : date}} \\
				\hline
			\end{tabular} \end{table} \FloatBarrier


			\paragraph{5.M11 MailTemplate}\

			\begin {table} [ht] \begin{tabular} {  p{3.5cm} p{9.6cm} }
				\hline
				{Typ} & {Modell} \\
				\hline
				{Syfte} & {För att hantera och mailmallar för färdigformatterade mail (SR1.27).} \\
				\hline
				{Funktion} & { } \\
				\hline
				{Delkomponenter} & { } \\
				\hline
				{Gränssnitt} & {Ärver metoder från ActiveRecord::Base, se ActiveRecord.} \\
				\hline
				{Beroenden} & {{\tt content} måste finnas.} \\
				\hline
				{Resurser} & {ActiveRecord::Base} \\
				\hline
				{Källor} & { } \\
				\hline
				{Process} & { } \\
				\hline
				{Data} & {{\tt description :string}

{\tt title :string}

{\tt content :string}} \\
				\hline
			\end{tabular} \end{table} \FloatBarrier


			\paragraph{5.M12 UserSession}\

			\begin {table} [ht] \begin{tabular} {  p{3.5cm} p{9.6cm} }
				\hline
				{Typ} & {Modell} \\
				\hline
				{Syfte} & {(Se 5.K2 AuthLogic, samt dess dokumentation) (SR1.1)} \\
				\hline
				{Funktion} & {Håller koll på de sessioner som finns inloggade} \\
				\hline
				{Delkomponenter} & {Har en User} \\
				\hline
				{Gränssnitt} & {{\tt find} Letar upp efterfrågad session om den finns.

{\tt user} } \\
				\hline
				{Beroenden} & { } \\
				\hline
				{Resurser} & {AuthLogic::Base::Session} \\
				\hline
				{Källor} & {AuthLogic: https://github.com/binarylogic/authlogic} \\
				\hline
				{Process} & { } \\
				\hline
				{Data} & { } \\
				\hline
			\end{tabular} \end{table} \FloatBarrier


			\paragraph{5.M13 Ability}\

			\begin {table} [ht] \begin{tabular} {  p{3.5cm} p{9.6cm} }
				\hline
				{Typ} & {Modell} \\
				\hline
				{Syfte} & {(Se 5.K3 CanCan, samt dokumentationen) (SR1.1, SR9.2)} \\
				\hline
				{Funktion} & {Håller koll på vilka rättigheter varje användare har.} \\
				\hline
				{Delkomponenter} & {Har flera Users \emph{(osäker på denna)}} \\
				\hline
				{Gränssnitt} & {{\tt can} Returnerar om användaren har rättighet att göra den efterfrågade handlingen.} \\
				\hline
				{Beroenden} & { } \\
				\hline
				{Resurser} & {ActiveRecord::Base, CanCan, User} \\
				\hline
				{Källor} & {CanCan: https://github.com/ryanb/cancan} \\
				\hline
				{Process} & { } \\
				\hline
				{Data} & { } \\
				\hline
			\end{tabular} \end{table} \FloatBarrier


	\subsection{Kontroller}



			\paragraph{5.C1 ApplicationController}\

			\begin {table} [ht] \begin{tabular} {  p{3.5cm} p{9.6cm} }
				\hline
				{Typ} & {Kontroller} \\
				\hline
				{Syfte} & {Rails-applikationens baskontroller som bland annat ska hantera autentisering (SR1.1).} \\
				\hline
				{Funktion} & {Alla kontrollers ärver funktionerna som denna klass definierar, vilket inkluderar autentisering.} \\
				\hline
				{Delkomponenter} & {Inga} \\
				\hline
				{Beroenden} & {Inga} \\
				\hline
				{Gränssnitt} & {ApplicationController tillhandahåller inga actions utan enbart filter och hjälpmetoder åt övriga kontrollers. De hjälpmetoder som definieras är:

{\tt current\_user} - Returnerar en User-instans för den aktiva användaren (försöker logga in via Authlogic om nödvändigt).

{\tt current\_user\_session} - Returnerar en UserSession-instans för den aktiva inloggningen (försöker logga in via Authlogic om nödvändigt).

{\tt require\_user} - Hjälpmetod som kastar en exception om den anropas och användaren inte är autentiserad.} \\
				\hline
				{Resurser} & {AuthLogic och CanCan.} \\
				\hline
				{Källor} & {Inga} \\
				\hline
				{Process} & {Filter definerade med {\tt before\_filter} resp. {\tt after\_filter} körs före resp. efter den anropade kontrollerns action.} \\
				\hline
				{Data} & {Inga} \\
				\hline
			\end{tabular} \end{table} \FloatBarrier


		\subsubsection{Kund}



			\paragraph{5.C2 BookingController}\

			\begin {table} [ht] \begin{tabular} {  p{3.5cm} p{9.6cm} }
				\hline
				{Typ} & {Kontroller} \\
				\hline
				{Syfte} & {Hanterar bokningsproceduren för kunden (SR1.2, SR1.3).} \\
				\hline
				{Funktion} & {Skapar en ny bokning och sparar den eller avbokar en existerande bokning. Bokningsprocessen delas upp i flera privata metoder som anropas beroende på vilket steg i bokningen man befinner sig i.} \\
				\hline
				{Delkomponenter} & {Inga} \\
				\hline
				{Beroenden} & {Inga} \\
				\hline
				{Gränssnitt} & {{\tt new} - påbörjar en bokning

{\tt create} - sparar data i användarsessionen allt eftersom och vid sista steget lagrar en bokning i databasen

{\tt cancel} - presenterar en bokning från en hash\_key med möjlighet till avbokning via ett formulär

{\tt destroy} - tar bort en bokning.} \\
				\hline
				{Resurser} & {ApplicationController, Reservation, ReservationCounter, Batch och Show.} \\
				\hline
				{Källor} & {Inga} \\
				\hline
				{Process} & {new > (create)+ > create

cancel > destroy} \\
				\hline
				{Data} & {{\tt @step} - steg i bokning

{\tt @steps} - array med vilka steg som finns

{\tt @reservation} - bokningsdata} \\
				\hline
			\end{tabular} \end{table} \FloatBarrier


		\subsubsection{Inloggning}



			\paragraph{5.C3 SessionController}\

			\begin {table} [ht] \begin{tabular} {  p{3.5cm} p{9.6cm} }
				\hline
				{Typ} & {Kontroller} \\
				\hline
				{Syfte} & {Hantera inloggning och sessioner (SR1.1).} \\
				\hline
				{Funktion} & {Skapar ny session vid inloggning, tar bort session vid utloggning.} \\
				\hline
				{Delkomponenter} & {Inga} \\
				\hline
				{Beroenden} & {Inga} \\
				\hline
				{Gränssnitt} & {{\tt new} - presenterar inloggningsformuläret

{\tt create} - tar emot data från formuläret och skapar en ny UserSession

{\tt destroy} - loggar ut} \\
				\hline
				{Resurser} & {ApplicationController och UserSession.} \\
				\hline
				{Källor} & {Inga} \\
				\hline
				{Process} & {new > create

destroy > new} \\
				\hline
				{Data} & {Inga} \\
				\hline
			\end{tabular} \end{table} \FloatBarrier


		\subsubsection{Admin}



			\paragraph{5.C4 AdminIndexController}\

			\begin {table} [ht] \begin{tabular} {  p{3.5cm} p{9.6cm} }
				\hline
				{Typ} & {Kontroller} \\
				\hline
				{Syfte} & {Generera data för översikt och statistik (SR1.16, SR1.17, SR1.18).} \\
				\hline
				{Funktion} & {Presenterar en översikt samt statistiksidorna för administratören och ekonomichefen.} \\
				\hline
				{Delkomponenter} & {Inga} \\
				\hline
				{Beroenden} & {Autentiserad som administratör eller ekonomichef.} \\
				\hline
				{Gränssnitt} & {{\tt index} - översikt

{\tt statistics} - statistik} \\
				\hline
				{Resurser} & {ApplicationController, Reservation, Batch och Show.} \\
				\hline
				{Källor} & {Inga} \\
				\hline
				{Process} & {Inga} \\
				\hline
				{Data} & {{\tt @unplaced} - antal betalade bokningar att placera

{\tt @shows} - tabell med information om aktuella föreställningar (antal bokade platser mm)

{\tt @stats} - tabell över föreställningar, sektionsgrupper(prisklass), med antal ordinarie, studenter, gratis, platser kvar} \\
				\hline
			\end{tabular} \end{table} \FloatBarrier


			\paragraph{5.C5 ReservationsController}\

			\begin {table} [ht] \begin{tabular} {  p{3.5cm} p{9.6cm} }
				\hline
				{Typ} & {Kontroller} \\
				\hline
				{Syfte} & {Administrering av bokningar (SR1.13).} \\
				\hline
				{Funktion} & {Redigering av existerande bokningar samt skapande och borttagning. Ekonomichef kan bara läsa information och registrera betalningar.} \\
				\hline
				{Delkomponenter} & {Inga} \\
				\hline
				{Beroenden} & {Autentiserad som administratör eller ekonomichef.} \\
				\hline
				{Gränssnitt} & {Ärver resursactions från InheritedResources med undantag för {\tt show}.

{\tt index} - lista bokningar med ev. filter

{\tt new}, {\tt edit} - formulär för att redigera alla attributer hos en bokning bortsett från stolsplacering} \\
				\hline
				{Resurser} & {ApplicationController, InheritedResources och Reservation.} \\
				\hline
				{Källor} & {Inga} \\
				\hline
				{Process} & {Standardprocesser för resurser.} \\
				\hline
				{Data} & {Se InheritedResources.

{\tt @filters} - array med möjliga filter samt indikationer för vilka som är aktiva ({\tt index})} \\
				\hline
			\end{tabular} \end{table} \FloatBarrier


			\paragraph{5.C6 PlacementsController}\

			\begin {table} [ht] \begin{tabular} {  p{3.5cm} p{9.6cm} }
				\hline
				{Typ} & {Kontroller} \\
				\hline
				{Syfte} & {Tilldelning av stolar till webbokningar (SR1.11).} \\
				\hline
				{Funktion} & {Hanterar stolsplaceringar för en given bokning med stöd för AJAX-anrop.} \\
				\hline
				{Delkomponenter} & {Inga} \\
				\hline
				{Beroenden} & {Autentiserad som administratör.} \\
				\hline
				{Gränssnitt} & {{\tt new}, {\tt edit} - presenterar aktuella placeringar och status för alla stolar med möjlighet att placera

{\tt test\_multiple} [AJAX] - validerar placeringar, skapar temporära låsningar och returnerar dess IDn

{\tt update\_multiple} - skapar och uppdaterar nya/existerande placeringar

{\tt destroy} - tar bort existerande placeringar för given bokning} \\
				\hline
				{Resurser} & {ApplicationController och Placement.} \\
				\hline
				{Källor} & {Inga} \\
				\hline
				{process} & {new > [test\_multiple]+ > update\_multiple

edit > [test\_multiple]+ > update\_multiple

edit > destroy} \\
				\hline
				{Data} & {{\tt @reservation} - den aktulla bokningsinstansen

{\tt @placements} - placeringar

{\tt @theater} - aktuell teater (bild mm)

{\tt @show} - information om föreställningen (datum mm)

{\tt @seats} - stolar med positioner och placeringsstatus (upptagen, tillfälligt låst, ledig, vald för denna bokning)} \\
				\hline
			\end{tabular} \end{table} \FloatBarrier


			\paragraph{5.C7 TheatersController}\

			\begin {table} [ht] \begin{tabular} {  p{3.5cm} p{9.6cm} }
				\hline
				{Typ} & {Kontroller} \\
				\hline
				{Syfte} & {Hantera teatrar (SR1.9).} \\
				\hline
				{Funktion} & {Hantera teatrar, inklusive möjligheten att kopiera existerande. Plus: administrera sektioner och platser.} \\
				\hline
				{Delkomponenter} & {SectionsController och SeatsController.} \\
				\hline
				{Beroenden} & {Autentiserad som administratör.} \\
				\hline
				{Gränssnitt} & {Ärver resursactions från InheritedResources.

{\tt new} - formulär för att skapa ny teater, inkluderar val att kopiera en existerande teater.} \\
				\hline
				{Resurser} & {ApplicationController, InheritedResources och Theater.} \\
				\hline
				{Källor} & {Inga} \\
				\hline
				{Process} & {Standardprocesser för resurser.} \\
				\hline
				{Data} & {Se InheritedResources.

{\tt @theaters} - existerande teatrars attributer ({\tt new})} \\
				\hline
			\end{tabular} \end{table} \FloatBarrier


			\paragraph{5.C8 SectionsController}\

			\begin {table} [ht] \begin{tabular} {  p{3.5cm} p{9.6cm} }
				\hline
				{Typ} & {Kontroller} \\
				\hline
				{Syfte} & {Hantera sektioner (SR1.9).} \\
				\hline
				{Funktion} & {Visa och redigera sektionsindelningar av en existerande teater.} \\
				\hline
				{Delkomponenter} & {Inga} \\
				\hline
				{Beroenden} & {Autentiserad som administratör, nästlad via TheaterController.} \\
				\hline
				{Gränssnitt} & {Ärver resursactions från InheritedResources med undantag för {\tt show}.

{\tt index} - visar sektioner för aktuell teater.} \\
				\hline
				{Resurser} & {ApplicationController, InheritedResources, Theater och Section.} \\
				\hline
				{Källor} & {Inga} \\
				\hline
				{Process} & {Standardprocesser för resurser.} \\
				\hline
				{Data} & {Se InheritedResources.

{\tt @theater} - aktuell teater} \\
				\hline
			\end{tabular} \end{table} \FloatBarrier


			\paragraph{5.C9 SeatsController}\

			\begin {table} [ht] \begin{tabular} {  p{3.5cm} p{9.6cm} }
				\hline
				{Typ} & {Kontroller} \\
				\hline
				{Syfte} & {Hantera stolarnas positioner och sektionstillhörighet (SR1.9).} \\
				\hline
				{Funktion} & {Visa och redigera en given teaters stolar och deras positioner på salongsskissen. AJAX används för att hantera stolar utan att ladda om sidan.} \\
				\hline
				{Delkomponenter} & {Inga} \\
				\hline
				{Beroenden} & {Autentiserad som administratör, nästlad via TheaterController.} \\
				\hline
				{Gränssnitt} & {{\tt index} - presenterar gränssnittet för att lägga till och redigera teaterns stolar

{\tt create} [AJAX] - sparar en ny stol

{\tt update} [AJAX] - uppdaterar en existerande stol

{\tt destroy} [AJAX] - tar bort en existerande stol} \\
				\hline
				{Resurser} & {ApplicationController, Theater och Seat.} \\
				\hline
				{Källor} & {Inga} \\
				\hline
				{Process} & {index > [create/update]+} \\
				\hline
				{Data} & {{\tt @seats} - existerande stolar

{\tt @theater} - aktuell teater} \\
				\hline
			\end{tabular} \end{table} \FloatBarrier


			\paragraph{5.C10 BatchesController}\

			\begin {table} [ht] \begin{tabular} {  p{3.5cm} p{9.6cm} }
				\hline
				{Typ} & {Kontroller} \\
				\hline
				{Syfte} & {Hantera omgångar (SR1.10).} \\
				\hline
				{Funktion} & {Visa och redigera omgångar.} \\
				\hline
				{Delkomponenter} & {ShowsController och PricingsController.} \\
				\hline
				{Beroenden} & {Autentiserad som administratör.} \\
				\hline
				{Gränssnitt} & {Ärver resursactions från InheritedResources med undantag för {\tt show}.} \\
				\hline
				{Resurser} & {ApplicationController, InheritedResources och Batch.} \\
				\hline
				{Källor} & {Inga} \\
				\hline
				{Process} & {Standardprocesser för resurser.} \\
				\hline
				{Data} & {Se InheritedResources.} \\
				\hline
			\end{tabular} \end{table} \FloatBarrier


			\paragraph{5.C11 ShowsController}\

			\begin {table} [ht] \begin{tabular} {  p{3.5cm} p{9.6cm} }
				\hline
				{Typ} & {Kontroller} \\
				\hline
				{Syfte} & {Hantera föreställningar (SR1.10).} \\
				\hline
				{Funktion} & {Hanterar föreställningar som hör till en given omgång.} \\
				\hline
				{Delkomponenter} & {Inga} \\
				\hline
				{Beroenden} & {Autentiserad som administratör, nästlad via BatchesController.} \\
				\hline
				{Gränssnitt} & {Ärver resursactions från InheritedResources med undantag för {\tt show}.

{\tt index} - listar alla föreställningar för given omgång} \\
				\hline
				{Resurser} & {ApplicationController, InheritedResources och Show.} \\
				\hline
				{Källor} & {Inga} \\
				\hline
				{Process} & {Standardprocesser för resurser.} \\
				\hline
				{Data} & {Se InheritedResources.

{\tt @batch} - aktuell omgång

{\tt @batches} - existerande omgångar} \\
				\hline
			\end{tabular} \end{table} \FloatBarrier


			\paragraph{5.C12 PricingsController}\

			\begin {table} [ht] \begin{tabular} {  p{3.5cm} p{9.6cm} }
				\hline
				{Typ} & {Kontroller} \\
				\hline
				{Syfte} & {Hantering av prisklasser (SR1.30).} \\
				\hline
				{Funktion} & {Administrerng av priser på omgångsnivå. Varje sektion har ett pris för varje rabattklass. Alla pris-/sektions-kombinationer för given omgång administreras på en gång, oavsett om de redan finns eller inte.} \\
				\hline
				{Delkomponenter} & {Inga} \\
				\hline
				{Beroenden} & {Autentiserad som administratör, nästlad via BatchesController.} \\
				\hline
				{Gränssnitt} & {{\tt index} - visar formulär för att välja priser för alla sektioner för den aktuella omgången

{\tt update\_multiple} - lagrar informationen med datan från {\tt index}} \\
				\hline
				{Resurser} & {ApplicationController, Batch, Section och Pricing.} \\
				\hline
				{Källor} & {Inga} \\
				\hline
				{Process} & {index > update\_multiple > index} \\
				\hline
				{Data} & {{\tt @batch} - aktuell omgång

{\tt @sections} - sektioner i omgångens teater

{\tt @pricings} - priser för varje sektion- och rabattklasskombination} \\
				\hline
			\end{tabular} \end{table} \FloatBarrier


			\paragraph{5.C13 UsersController}\

			\begin {table} [ht] \begin{tabular} {  p{3.5cm} p{9.6cm} }
				\hline
				{Typ} & {Kontroller} \\
				\hline
				{Syfte} & {Hantera användare och deras lösenord (SR1.7, SR1.8).} \\
				\hline
				{Funktion} & {Administrering av bokningssystemets användare, inklusive lösenordsändringar och giltighetstid för säljare.} \\
				\hline
				{Delkomponenter} & {Inga} \\
				\hline
				{Beroenden} & {Autentiserad som administratör.} \\
				\hline
				{Gränssnitt} & {Ärver resursactions från InheritedResources.} \\
				\hline
				{Resurser} & {ApplicationController, InheritedResources och User.} \\
				\hline
				{Källor} & {Inga} \\
				\hline
				{Process} & {Standardprocesser för resurser.} \\
				\hline
				{Data} & {Se InheritedResources.} \\
				\hline
			\end{tabular} \end{table} \FloatBarrier


			\paragraph{5.C14 MailingController}\

			\begin {table} [ht] \begin{tabular} {  p{3.5cm} p{9.6cm} }
				\hline
				{Typ} & {Kontroller} \\
				\hline
				{Syfte} & {Hantera manuella och automatiska utskick av mail (SR1.14, SR1.15, SR1.27).} \\
				\hline
				{Funktion} & {Hantera mallar för mail och utskick av mail till förvalda bokningars kontaktpersoner. Nyckelord i mallarna byts ut mot motsvarande bokningsdata vid utskick. Plus: val av mall och editering vid utskick i samma steg med hjälp av AJAX ({\tt index}).} \\
				\hline
				{Delkomponenter} & {Inga} \\
				\hline
				{Beroenden} & {Autentiserad som administratör.} \\
				\hline
				{Gränssnitt} & {Ärver resursactions från InheritedResources

{\tt prepare} - formulär för att skriva och skicka mail utan mall

{\tt preview} - förhandsgranska mail innan utskick

{\tt send} - skickar mail} \\
				\hline
				{Resurser} & {ApplicationController, InheritedResources, Reservation och Mail.} \\
				\hline
				{Källor} & {Inga} \\
				\hline
				{Process} & {Standardprocesser för resurser.

ReservationsController\#index > index > preview > send - Utskick av existerande mall till valda mottagare.

ReservationsController\#index > index > new > create > index > preview > send - Utskick av ny mall till valda mottagare.

ReservationsController\#index > index > prepare > send - Utskick av mail utan mall till valda mottagare.} \\
				\hline
				{Data} & {Se InheritedResources.

{\tt @reciepts = session[:reciepts]} - mailmottagare (semipermanent)

{\tt @mail = session[:mail]} - den aktuella mailmallen, kan vara temporär för att möjliggöra utskick utan mall} \\
				\hline
			\end{tabular} \end{table} \FloatBarrier


		\subsubsection{Ekonomichef}


		Ekonomichefen har åtkomst till AdminIndexController (för översikt och statistik) respektive ReservationsController (för att redigera existerande bokningars betalningsstatus) per SR1.19.

		\subsubsection{Säljare}



			\paragraph{5.C15 SalesController}\

			\begin {table} [ht] \begin{tabular} {  p{3.5cm} p{9.6cm} }
				\hline
				{Typ} & {Kontroller} \\
				\hline
				{Syfte} & {Hantering av säljarens gränssnitt och funktioner (SR1.5, SR1.6, SR1.25).} \\
				\hline
				{Funktion} & {Möjlighet att genomköra kontantköp (med nya bokningar) samt ta betalt för, och lämna ut existerande bokningars biljetter. {\tt new} delas upp i flera privata metoder som anropas beroende på vilket steg i bokningen man befinner sig i.} \\
				\hline
				{Delkomponenter} & {Inga} \\
				\hline
				{Beroenden} & {Autentiserad som säljare.} \\
				\hline
				{Gränssnitt} & {{\tt index} - startsida med länkar till ny resp. sök bokning

{\tt new} - påbörjar ett nytt kontantköp

{\tt create} - sparar data i användarsessionen allt eftersom och vid sista steget lagrar en bokning i databasen

{\tt find} - sök efter existerande bokning

{\tt place} - stolsplacering av existerande bokning, görs via PlacementsController

{\tt retrieve} - presentera vilka biljetter som ska hämtas

{\tt payment} - presenterar hur mycket som ska tas betalt

{\tt finalize} - uppdaterar bokningens status till betald och utlämnad samt instruerar säljaren om att lämna ut biljetten, går ej att ångra} \\
				\hline
				{Resurser} & {ApplicationController, Reservation, ReservationCounter och Placement.} \\
				\hline
				{Källor} & {Inga} \\
				\hline
				{Process} & {index > new > [place]+ > create > retrieve > payment > finalize > index - nytt kontantköp

index > find > retrieve > finalize > index - lämna ut betald biljett

index > find > retrieve > payment > finalize > index - lämna ut obetald biljett

index > find > place > retrieve > payment > finalize > index - lämna ut oplacerad biljett} \\
				\hline
				{Data} & {{\tt @step} - steg i bokning

{\tt @steps} - array med vilka steg som finns

{\tt @reservation = session[:reservation]} - bokningsdata} \\
				\hline
			\end{tabular} \end{table} \FloatBarrier


	\subsection{Vyer}



		\subsubsection{Layouter}



			\paragraph{5.L1 admin\_layout.html}\

			\begin {table} [ht] \begin{tabular} {  p{3.5cm} p{9.6cm} }
				\hline
				{Typ} & {Layout} \\
				\hline
				{Syfte} & {Alla administratörs- och ekonomichefssidor skall ha en gemensam layout (SR1.19, SR1.28)} \\
				\hline
				{Funktion} & {Renderar en layout för administratörens och ekonomichefens sidor, inklusive meny. De länkar i menyn som visas är beroende på vem som är inloggad, ty ekonomichefen ska inte ha tillgång till alla verktyg som administratören har.} \\
				\hline
				{Delkomponenter} & {Inga} \\
				\hline
				{Beroenden} & {Inga} \\
				\hline
				{Gränssnitt} & {Ej applicerbart} \\
				\hline
				{Resurser} & {Inga} \\
				\hline
				{Källor} & {Inga} \\
				\hline
				{Process} & {Inga} \\
				\hline
				{Data} & {{\tt menu} – Array med alla menylänkar} \\
				\hline
			\end{tabular} \end{table} \FloatBarrier


			\paragraph{5.L2 sales\_layout.html}\

			\begin {table} [ht] \begin{tabular} {  p{3.5cm} p{9.6cm} }
				\hline
				{Typ} & {Layout} \\
				\hline
				{Syfte} & {Säljaren ska ha ett enkelt gränssnitt (SR1.5, SR1.5, SR1.25, SR1.28)} \\
				\hline
				{Funktion} & {Alla sidor som säljaren använder skall ha samma layout.} \\
				\hline
				{Delkomponenter} & {Inga} \\
				\hline
				{Beroenden} & {Inga} \\
				\hline
				{Gränssnitt} & {Ej applicerbart} \\
				\hline
				{Resurser} & {Inga} \\
				\hline
				{Källor} & {Inga} \\
				\hline
				{Process} & {Inga} \\
				\hline
				{Data} & {Inga} \\
				\hline
			\end{tabular} \end{table} \FloatBarrier


			\paragraph{5.L3 simple\_layout.html}\

			\begin {table} [ht] \begin{tabular} {  p{3.5cm} p{9.6cm} }
				\hline
				{Typ} & {Layout} \\
				\hline
				{Syfte} & {Kårspexets personal behöver en layout till sin inloggning (SR1.1, SR9.2, SR1.28)} \\
				\hline
				{Funktion} & {Inloggninssidan ska ha en simpel layout med en ruta där användaren kan logga in.} \\
				\hline
				{Delkomponenter} & {Inga} \\
				\hline
				{Beroenden} & {Inga} \\
				\hline
				{Gränssnitt} & {Ej applicerbart} \\
				\hline
				{Resurser} & {Inga} \\
				\hline
				{Källor} & {Inga} \\
				\hline
				{Process} & {Inga} \\
				\hline
				{Data} & { } \\
				\hline
			\end{tabular} \end{table} \FloatBarrier


			\paragraph{5.L4 customer\_layout.html}\

			\begin {table} [ht] \begin{tabular} {  p{3.5cm} p{9.6cm} }
				\hline
				{Typ} & {Layout} \\
				\hline
				{Syfte} & {Alla bokningsprocessens steg ska ha samma layout (SR1.2)} \\
				\hline
				{Funktion} & {Layout ska matcha Kårspexets nuvarande webbsida, och vara användarvänlig för kunden.} \\
				\hline
				{Delkomponenter} & {Inga} \\
				\hline
				{Beroenden} & {Inga} \\
				\hline
				{Gränssnitt} & {Ej applicerbart} \\
				\hline
				{Resurser} & {Inga} \\
				\hline
				{Källor} & {Inga} \\
				\hline
				{Process} & {Inga} \\
				\hline
				{Data} & {{\tt menu} – Array med alla menylänkar} \\
				\hline
			\end{tabular} \end{table} \FloatBarrier


		\subsubsection{Inloggning}



			\paragraph{5.V1 SessionController\#new.html}\

			\begin {table} [ht] \begin{tabular} {  p{3.5cm} p{9.6cm} }
				\hline
				{Typ} & {Vy} \\
				\hline
				{Syfte} & {Användaren ska kunna logga in på systemet (SR1.1, SR9.2)} \\
				\hline
				{Funktion} & {Renderar inloggningsformuläret.} \\
				\hline
				{Delkomponenter} & {Inga} \\
				\hline
				{Beroenden} & {SessionController} \\
				\hline
				{Gränssnitt} & {Ej applicerbart} \\
				\hline
				{Resurser} & {5.L3 simple\_layout.html} \\
				\hline
				{Källor} & {Inga} \\
				\hline
				{Process} & {Ej applicerbart} \\
				\hline
				{Data} & {Inga} \\
				\hline
			\end{tabular} \end{table} \FloatBarrier


		\subsubsection{Admin}



			\paragraph{5.V2 AdminIndexController\#index.html}\

			\begin {table} [ht] \begin{tabular} {  p{3.5cm} p{9.6cm} }
				\hline
				{Typ} & {Vy} \\
				\hline
				{Syfte} & {Administratören behöver en snabb, första översiktssida (SR1.19)} \\
				\hline
				{Funktion} & {Ger en översikt direkt efter inloggning.} \\
				\hline
				{Delkomponenter} & {Inga} \\
				\hline
				{Beroenden} & {AdminIndexController} \\
				\hline
				{Gränssnitt} & {Ej applicerbart} \\
				\hline
				{Resurser} & {5.L1 admin\_layout.html} \\
				\hline
				{Källor} & {Inga} \\
				\hline
				{Process} & {Inga} \\
				\hline
				{Data} & {@unplaced - antal betalade bokningar att placera

@shows - tabell med information om aktuella föreställningar (antal bokade platser mm)} \\
				\hline
			\end{tabular} \end{table} \FloatBarrier


			\paragraph{5.V3 AdminIndexController\#stat\_standard.html}\

			\begin {table} [ht] \begin{tabular} {  p{3.5cm} p{9.6cm} }
				\hline
				{Typ} & {Vy} \\
				\hline
				{Syfte} & {Administratören och ekonomichefen ska ha möjlighet att se statistik (SR1.16)} \\
				\hline
				{Funktion} & {Visar statistik i form av rådata i en tabell.} \\
				\hline
				{Delkomponenter} & {Inga} \\
				\hline
				{Beroenden} & {AdminIndexController} \\
				\hline
				{Gränssnitt} & {Ej applicerbart} \\
				\hline
				{Resurser} & {5.L1 admin\_layout.html} \\
				\hline
				{Källor} & {Inga} \\
				\hline
				{Process} & {Inga} \\
				\hline
				{Data} & {@stats - tabell över föreställningar, sektionsgrupper(prisklass), med antal ordinarie, studenter, gratis, platser kvar} \\
				\hline
			\end{tabular} \end{table} \FloatBarrier


			\paragraph{5.V4 AdminIndexController\#stat\_plus.html}\

			\begin {table} [ht] \begin{tabular} {  p{3.5cm} p{9.6cm} }
				\hline
				{Typ} & {Vy} \\
				\hline
				{Syfte} & {Administratören och ekonomichefen ska ha möjlighet att se organiserad statistik (SR1.17)} \\
				\hline
				{Funktion} & {Visar statistik i form av ett flertal tabeller under olika rubriker.} \\
				\hline
				{Delkomponenter} & {Inga} \\
				\hline
				{Beroenden} & {AdminIndexController} \\
				\hline
				{Gränssnitt} & {Ej applicerbart} \\
				\hline
				{Resurser} & {5.L1 admin\_layout.html} \\
				\hline
				{Källor} & {Inga} \\
				\hline
				{Process} & {Ej applicerbart} \\
				\hline
				{Data} & {@stats - tabell över föreställningar, sektionsgrupper(prisklass), med antal ordinarie, studenter, gratis, platser kvar} \\
				\hline
			\end{tabular} \end{table} \FloatBarrier


			\paragraph{5.V5 AdminIndexController\#stat\_deluxe.html}\

			\begin {table} [ht] \begin{tabular} {  p{3.5cm} p{9.6cm} }
				\hline
				{Typ} & {Vy} \\
				\hline
				{Syfte} & {Administratören och ekonomichefen ska ha möjlighet att se organiserad, grafisk statistik (SR1.18)} \\
				\hline
				{Funktion} & {Visar statistik i form av en interaktiv graf.} \\
				\hline
				{Delkomponenter} & {Inga} \\
				\hline
				{Beroenden} & {AdminIndexController} \\
				\hline
				{Gränssnitt} & {Ej applicerbart} \\
				\hline
				{Resurser} & {5.L1 admin\_layout.html} \\
				\hline
				{Källor} & {Inga} \\
				\hline
				{Process} & {Ej applicerbart} \\
				\hline
				{Data} & {@stats - tabell över föreställningar, sektionsgrupper(prisklass), med antal ordinarie, studenter, gratis, platser kvar} \\
				\hline
			\end{tabular} \end{table} \FloatBarrier


			\paragraph{5.V6 ReservationsController\#index.html}\

			\begin {table} [ht] \begin{tabular} {  p{3.5cm} p{9.6cm} }
				\hline
				{Typ} & {Vy} \\
				\hline
				{Syfte} & {Administratören ska kunna hantera bokningar (SR1.13)} \\
				\hline
				{Funktion} & {Administratören får en överblick över de val han kan göra angående bokningar.} \\
				\hline
				{Delkomponenter} & {Inga} \\
				\hline
				{Beroenden} & {ReservationsController} \\
				\hline
				{Gränssnitt} & {Ej applicerbart} \\
				\hline
				{Resurser} & {5.L1 admin\_layout.html} \\
				\hline
				{Källor} & {Inga} \\
				\hline
				{Process} & {Ej applicerbart} \\
				\hline
				{Data} & {@reservations

@filters - array med möjliga filter samt indikationer för vilka som är aktiva (index)} \\
				\hline
			\end{tabular} \end{table} \FloatBarrier


			\paragraph{5.V7 ReservationsController\#new.html}\

			\begin {table} [ht] \begin{tabular} {  p{3.5cm} p{9.6cm} }
				\hline
				{Typ} & {Vy} \\
				\hline
				{Syfte} & {Administratören ska kunna hantera bokningar (SR1.13)} \\
				\hline
				{Funktion} & {Administratören ska kunna skapa en ny bokning från sitt gränssnitt.} \\
				\hline
				{Delkomponenter} & {AD5.P1} \\
				\hline
				{Beroenden} & {ReservationsController} \\
				\hline
				{Gränssnitt} & {Ej applicerbart} \\
				\hline
				{Resurser} & {5.L1 admin\_layout.html} \\
				\hline
				{Källor} & {Inga} \\
				\hline
				{Process} & {Ej applicerbart} \\
				\hline
				{Data} & {@reservation} \\
				\hline
			\end{tabular} \end{table} \FloatBarrier


			\paragraph{5.V8 ReservationsController\#show.html}\

			\begin {table} [ht] \begin{tabular} {  p{3.5cm} p{9.6cm} }
				\hline
				{Typ} & {Vy} \\
				\hline
				{Syfte} & {Administratören ska kunna hantera bokningar (SR1.13)} \\
				\hline
				{Funktion} & {Administratören ska kunna visa att en bokning ligger i databasen genom att söka efter den med sökfunktionen eller filter.} \\
				\hline
				{Delkomponenter} & {Inga} \\
				\hline
				{Beroenden} & {ReservationsController} \\
				\hline
				{Gränssnitt} & {Ej applicerbart} \\
				\hline
				{Resurser} & {5.L1 admin\_layout.html} \\
				\hline
				{Källor} & {Inga} \\
				\hline
				{Process} & {Ej applicerbart} \\
				\hline
				{Data} & {@reservation} \\
				\hline
			\end{tabular} \end{table} \FloatBarrier


			\paragraph{5.V9 ReservationsController\#edit.html}\

			\begin {table} [ht] \begin{tabular} {  p{3.5cm} p{9.6cm} }
				\hline
				{Typ} & {Vy} \\
				\hline
				{Syfte} & {Administratören ska kunna hantera bokningar (SR1.13)} \\
				\hline
				{Funktion} & {Administratören ska kunna göra ändringar i en bokning, till exempel ändra antalet biljetter, ändra betalstatus, placeringsstatus, och så vidare.} \\
				\hline
				{Delkomponenter} & {AD5.P1} \\
				\hline
				{Beroenden} & {ReservationsController} \\
				\hline
				{Gränssnitt} & {Ej applicerbart} \\
				\hline
				{Resurser} & {5.L1 admin\_layout.html} \\
				\hline
				{Källor} & {Inga} \\
				\hline
				{Process} & {Ej applicerbart} \\
				\hline
				{Data} & {@reservation} \\
				\hline
			\end{tabular} \end{table} \FloatBarrier


			\paragraph{5.P1 ReservationsController\#\_form.html}\

			\begin {table} [ht] \begin{tabular} {  p{3.5cm} p{9.6cm} }
				\hline
				{Typ} & {Partiell vy} \\
				\hline
				{Syfte} & {Administratören ska kunna hantera bokningar (SR1.13)} \\
				\hline
				{Funktion} & {Renderar ett formulär för att redigera Reservations-objekt.} \\
				\hline
				{Delkomponenter} & {Inga} \\
				\hline
				{Beroenden} & {ReservationsController} \\
				\hline
				{Gränssnitt} & {Ej applicerbart} \\
				\hline
				{Resurser} & {Reservation} \\
				\hline
				{Källor} & {Inga} \\
				\hline
				{Process} & {Ej applicerbart} \\
				\hline
				{Data} & {@reservation} \\
				\hline
			\end{tabular} \end{table} \FloatBarrier


			\paragraph{5.V10 PlacementsController\#new.html}\

			\begin {table} [ht] \begin{tabular} {  p{3.5cm} p{9.6cm} }
				\hline
				{Typ} & {Vy} \\
				\hline
				{Syfte} & {Administratören ska kunna placera ut bokningar (SR1.11, SR1.25)} \\
				\hline
				{Funktion} & {Administratören ska kunna placera ut platser i den korrekta sektionen som kunden har bokat.} \\
				\hline
				{Delkomponenter} & {Inga} \\
				\hline
				{Beroenden} & {PlacementsController} \\
				\hline
				{Gränssnitt} & {Ej applicerbart} \\
				\hline
				{Resurser} & {5.L1 admin\_layout.html} \\
				\hline
				{Källor} & {Inga} \\
				\hline
				{Process} & {Ej applicerbart} \\
				\hline
				{Data} & {@reservation - den aktulla bokningsinstansen

@placements - placeringar

@theater - aktuell teater (bild mm)

@show - information om föreställningen (datum mm)

@seats - stolar med positioner och placeringsstatus (upptagen, tillfälligt låst, ledig, vald för denna bokning)} \\
				\hline
			\end{tabular} \end{table} \FloatBarrier


			\paragraph{5.V11 PlacementsController\#edit.html}\

			\begin {table} [ht] \begin{tabular} {  p{3.5cm} p{9.6cm} }
				\hline
				{Typ} & {Vy} \\
				\hline
				{Syfte} & {Administratören ska kunna placera ut bokningar (SR1.11, SR1.25)} \\
				\hline
				{Funktion} & {Administratören ska kunna ändra placeringen av en bokning, så länge som den inte har hämtats ut av kunden.} \\
				\hline
				{Delkomponenter} & {Inga} \\
				\hline
				{Beroenden} & {PlacementsController} \\
				\hline
				{Gränssnitt} & {Ej applicerbart} \\
				\hline
				{Resurser} & {5.L1 admin\_layout.html} \\
				\hline
				{Källor} & {Inga} \\
				\hline
				{Process} & {Ej applicerbart} \\
				\hline
				{Data} & {@reservation - den aktulla bokningsinstansen

@placements - placeringar

@theater - aktuell teater (bild mm)

@show - information om föreställningen (datum mm)

@seats - stolar med positioner och placeringsstatus (upptagen, tillfälligt låst, ledig, vald för denna bokning)} \\
				\hline
			\end{tabular} \end{table} \FloatBarrier


			\paragraph{5.V12 TheatersController\#index.html}\

			\begin {table} [ht] \begin{tabular} {  p{3.5cm} p{9.6cm} }
				\hline
				{Typ} & {Vy} \\
				\hline
				{Syfte} & {Administratören ska kunna hantera teatrar (SR1.9)} \\
				\hline
				{Funktion} & {Administratören får en överblick över de teatrar som finns i databasen och vad han kan göra med dem.} \\
				\hline
				{Delkomponenter} & {Inga} \\
				\hline
				{Beroenden} & {TheatersController} \\
				\hline
				{Gränssnitt} & {Ej applicerbart} \\
				\hline
				{Resurser} & {5.L1 admin\_layout.html} \\
				\hline
				{Källor} & {Inga} \\
				\hline
				{Process} & {Ej applicerbart} \\
				\hline
				{Data} & {@theaters - existerande teatrars attributer } \\
				\hline
			\end{tabular} \end{table} \FloatBarrier


			\paragraph{5.V13 TheatersController\#new.html}\

			\begin {table} [ht] \begin{tabular} {  p{3.5cm} p{9.6cm} }
				\hline
				{Typ} & {Vy} \\
				\hline
				{Syfte} & {Administratören ska kunna hantera teatrar (SR1.9)} \\
				\hline
				{Funktion} & {Administratören ska genom ett speciellt gränssnitt kunna skapa en ny teater.} \\
				\hline
				{Delkomponenter} & {AD5.P3} \\
				\hline
				{Beroenden} & {TheatersController} \\
				\hline
				{Gränssnitt} & {Ej applicerbart} \\
				\hline
				{Resurser} & {5.L1 admin\_layout.html} \\
				\hline
				{Källor} & {Inga} \\
				\hline
				{Process} & {Ej applicerbart} \\
				\hline
				{Data} & {@theater} \\
				\hline
			\end{tabular} \end{table} \FloatBarrier


			\paragraph{5.V14 TheatersController\#show.html}\

			\begin {table} [ht] \begin{tabular} {  p{3.5cm} p{9.6cm} }
				\hline
				{Typ} & {Vy} \\
				\hline
				{Syfte} & {Administratören ska kunna hantera teatrar (SR1.9)} \\
				\hline
				{Funktion} & {Administratören ska kunna se vilka teatrar som finns.} \\
				\hline
				{Delkomponenter} & {Inga} \\
				\hline
				{Beroenden} & {TheatersController} \\
				\hline
				{Gränssnitt} & {Ej applicerbart} \\
				\hline
				{Resurser} & {5.L1 admin\_layout.html} \\
				\hline
				{Källor} & {Inga} \\
				\hline
				{Process} & {Ej applicerbart} \\
				\hline
				{Data} & {@theater} \\
				\hline
			\end{tabular} \end{table} \FloatBarrier


			\paragraph{5.V15 TheatersController\#edit.html}\

			\begin {table} [ht] \begin{tabular} {  p{3.5cm} p{9.6cm} }
				\hline
				{Typ} & {Vy} \\
				\hline
				{Syfte} & {Administratören ska kunna hantera teatrar (SR1.9)} \\
				\hline
				{Funktion} & {Administratören ska kunna göra ändringar till en teater, till exempel om en sektion är under reparation så ska kunder inte kunna boka platser i den.} \\
				\hline
				{Delkomponenter} & {AD5.P3} \\
				\hline
				{Beroenden} & {TheatersController} \\
				\hline
				{Gränssnitt} & {Ej applicerbart} \\
				\hline
				{Resurser} & {5.L1 admin\_layout.html} \\
				\hline
				{Källor} & {Inga} \\
				\hline
				{Process} & {Ej applicerbart} \\
				\hline
				{Data} & {@theater} \\
				\hline
			\end{tabular} \end{table} \FloatBarrier


			\paragraph{5.P2 TheatersController\#\_form}\

			\begin {table} [ht] \begin{tabular} {  p{3.5cm} p{9.6cm} }
				\hline
				{Typ} & {Partiell vy} \\
				\hline
				{Syfte} & {Administratören ska kunna hantera teatrar (SR1.9)} \\
				\hline
				{Funktion} & {Renderar ett formulär för att redigera Theaters-objekt.} \\
				\hline
				{Delkomponenter} & {Inga} \\
				\hline
				{Beroenden} & {TheatersController} \\
				\hline
				{Gränssnitt} & {Ej applicerbart} \\
				\hline
				{Resurser} & {Theater} \\
				\hline
				{Källor} & {Inga} \\
				\hline
				{Process} & {Ej applicerbart} \\
				\hline
				{Data} & {@theater} \\
				\hline
			\end{tabular} \end{table} \FloatBarrier


			\paragraph{5.V16 SectionsController\#index.html}\

			\begin {table} [ht] \begin{tabular} {  p{3.5cm} p{9.6cm} }
				\hline
				{Typ} & {Vy} \\
				\hline
				{Syfte} & {Administratören och säljaren ska kunna välja sektionsplaceringen (SR1.22, SR1.23)} \\
				\hline
				{Funktion} & {Administratören och säljaren ska få en överblick över de sektioner som platser kan placeras ut i och hur detta kan göras.} \\
				\hline
				{Delkomponenter} & {Inga} \\
				\hline
				{Beroenden} & {SectionsController} \\
				\hline
				{Gränssnitt} & {Ej applicerbart} \\
				\hline
				{Resurser} & {5.L1 admin\_layout.html} \\
				\hline
				{Källor} & {Inga} \\
				\hline
				{Process} & {Ej applicerbart} \\
				\hline
				{Data} & {@sections

@theater - aktuell teater} \\
				\hline
			\end{tabular} \end{table} \FloatBarrier


			\paragraph{5.V17 SectionsController\#new.html}\

			\begin {table} [ht] \begin{tabular} {  p{3.5cm} p{9.6cm} }
				\hline
				{Typ} & {Vy} \\
				\hline
				{Syfte} & {Administratören och säljaren ska kunna välja sektionsplaceringen (SR1.22)} \\
				\hline
				{Funktion} & {Administratören och säljaren ska kunna välja en sektion att placera en boknings platser i.} \\
				\hline
				{Delkomponenter} & {AD5.P4} \\
				\hline
				{Beroenden} & {SectionsController} \\
				\hline
				{Gränssnitt} & {Ej applicerbart} \\
				\hline
				{Resurser} & {5.L1 admin\_layout.html} \\
				\hline
				{Källor} & {Inga} \\
				\hline
				{Process} & {Ej applicerbart} \\
				\hline
				{Data} & {@section

@theater - aktuell teater} \\
				\hline
			\end{tabular} \end{table} \FloatBarrier


			\paragraph{5.V18 SectionsController\#show.html}\

			\begin {table} [ht] \begin{tabular} {  p{3.5cm} p{9.6cm} }
				\hline
				{Typ} & {Vy} \\
				\hline
				{Syfte} & {Administratören och säljaren ska kunna välja sektionsplaceringen (SR1.22, SR1.23 (i plus))} \\
				\hline
				{Funktion} & {Administratören och säljaren ska kunna se alla sektioner som är valbara.} \\
				\hline
				{Delkomponenter} & {Inga} \\
				\hline
				{Beroenden} & {SectionsController} \\
				\hline
				{Gränssnitt} & {Ej applicerbart} \\
				\hline
				{Resurser} & {5.L1 admin\_layout.html} \\
				\hline
				{Källor} & {Inga} \\
				\hline
				{Process} & {Ej applicerbart} \\
				\hline
				{Data} & {@theater - aktuell teater} \\
				\hline
			\end{tabular} \end{table} \FloatBarrier


			\paragraph{5.V19 SectionsController\#edit.html}\

			\begin {table} [ht] \begin{tabular} {  p{3.5cm} p{9.6cm} }
				\hline
				{Typ} & {Vy} \\
				\hline
				{Syfte} & {Administratören och säljaren ska kunna välja sektionsplaceringen (SR1.22)} \\
				\hline
				{Funktion} & {Administratören ska kunna ändra vilka sektioner som är valbara.} \\
				\hline
				{Delkomponenter} & {AD5.P4} \\
				\hline
				{Beroenden} & {SectionsController} \\
				\hline
				{Gränssnitt} & {Ej applicerbart} \\
				\hline
				{Resurser} & {5.L1 admin\_layout.html} \\
				\hline
				{Källor} & {Inga} \\
				\hline
				{Process} & {Ej applicerbart} \\
				\hline
				{Data} & {@section

@theater - aktuell teater} \\
				\hline
			\end{tabular} \end{table} \FloatBarrier


			\paragraph{5.P3 SectionsController\#\_form}\

			\begin {table} [ht] \begin{tabular} {  p{3.5cm} p{9.6cm} }
				\hline
				{Typ} & {Partiell vy} \\
				\hline
				{Syfte} & {Administratören och säljaren ska kunna välja sektionsplaceringen (SR1.22)} \\
				\hline
				{Funktion} & {Renderar ett formulär för att redigera Sections-objekt.} \\
				\hline
				{Delkomponenter} & {Inga} \\
				\hline
				{Beroenden} & {SectionsController} \\
				\hline
				{Gränssnitt} & {Ej applicerbart} \\
				\hline
				{Resurser} & {Section} \\
				\hline
				{Källor} & {Inga} \\
				\hline
				{Process} & {Ej applicerbart} \\
				\hline
				{Data} & {@section

@theater - aktuell teater} \\
				\hline
			\end{tabular} \end{table} \FloatBarrier


			\paragraph{5.V20 SeatsController\#index.html}\

			\begin {table} [ht] \begin{tabular} {  p{3.5cm} p{9.6cm} }
				\hline
				{Typ} & {Vy} \\
				\hline
				{Syfte} & {Administratören ska kunna hantera stolsplacering (SR1.11, SR1.25)} \\
				\hline
				{Funktion} & {Administratören ska kunna identifiera och placera ut stolar för en given teater.} \\
				\hline
				{Delkomponenter} & {Inga} \\
				\hline
				{Beroenden} & {SeatsController} \\
				\hline
				{Gränssnitt} & {Ej applicerbart} \\
				\hline
				{Resurser} & {5.L1 admin\_layout.html} \\
				\hline
				{Källor} & {Inga} \\
				\hline
				{Process} & {Ej applicerbart} \\
				\hline
				{Data} & {@seats

@theater - aktuell teater} \\
				\hline
			\end{tabular} \end{table} \FloatBarrier


			\paragraph{5.V21 BatchesController\#index.html}\

			\begin {table} [ht] \begin{tabular} {  p{3.5cm} p{9.6cm} }
				\hline
				{Typ} & {Vy} \\
				\hline
				{Syfte} & {Administratören skall kunna lägga till omgångar (SR1.10)} \\
				\hline
				{Funktion} & {Ge en översikt på alla omgångar.} \\
				\hline
				{Delkomponenter} & {Inga} \\
				\hline
				{Beroenden} & {BatchesController} \\
				\hline
				{Gränssnitt} & {Ej applicerbart} \\
				\hline
				{Resurser} & {5.L1 admin\_layout.html} \\
				\hline
				{Källor} & {Inga} \\
				\hline
				{Process} & {Ej applicerbart} \\
				\hline
				{Data} & {@batches} \\
				\hline
			\end{tabular} \end{table} \FloatBarrier


			\paragraph{5.V22 BatchesController\#new.html}\

			\begin {table} [ht] \begin{tabular} {  p{3.5cm} p{9.6cm} }
				\hline
				{Typ} & {Vy} \\
				\hline
				{Syfte} & {Administratören skall kunna lägga till omgångar (SR1.10)} \\
				\hline
				{Funktion} & {Skapa en ny omgång.} \\
				\hline
				{Delkomponenter} & {AD5.3.36} \\
				\hline
				{Beroenden} & {BatchesController} \\
				\hline
				{Gränssnitt} & {Ej applicerbart} \\
				\hline
				{Resurser} & {5.L1 admin\_layout.html} \\
				\hline
				{Källor} & {Inga} \\
				\hline
				{Process} & {Ej applicerbart} \\
				\hline
				{Data} & {@batch} \\
				\hline
			\end{tabular} \end{table} \FloatBarrier


			\paragraph{5.V23 BatchesController\#show.html}\

			\begin {table} [ht] \begin{tabular} {  p{3.5cm} p{9.6cm} }
				\hline
				{Typ} & {Vy} \\
				\hline
				{Syfte} & {Administratören skall kunna lägga till omgångar (SR1.10)} \\
				\hline
				{Funktion} & {Visa information om en omgång, till exemple alla föreställningar som hör till omgången.} \\
				\hline
				{Delkomponenter} & {Inga} \\
				\hline
				{Beroenden} & {BatchesController} \\
				\hline
				{Gränssnitt} & {Ej applicerbart} \\
				\hline
				{Resurser} & {5.L1 admin\_layout.html} \\
				\hline
				{Källor} & {Inga} \\
				\hline
				{Process} & {Ej applicerbart} \\
				\hline
				{Data} & {@batch} \\
				\hline
			\end{tabular} \end{table} \FloatBarrier


			\paragraph{5.V24 BatchesController\#edit.html}\

			\begin {table} [ht] \begin{tabular} {  p{3.5cm} p{9.6cm} }
				\hline
				{Typ} & {Vy} \\
				\hline
				{Syfte} & {Administratören skall kunna lägga till omgångar (SR1.10)} \\
				\hline
				{Funktion} & {Ändra omgången och vilka föreställningar som ingår i den.} \\
				\hline
				{Delkomponenter} & {AD5.3.36} \\
				\hline
				{Beroenden} & {BatchesController} \\
				\hline
				{Gränssnitt} & {Ej applicerbart} \\
				\hline
				{Resurser} & {5.L1 admin\_layout.html} \\
				\hline
				{Källor} & {Inga} \\
				\hline
				{Process} & {Ej applicerbart} \\
				\hline
				{Data} & {@batch} \\
				\hline
			\end{tabular} \end{table} \FloatBarrier


			\paragraph{5.P4 BatchesController\#\_form}\

			\begin {table} [ht] \begin{tabular} {  p{3.5cm} p{9.6cm} }
				\hline
				{Typ} & {Partiell vy} \\
				\hline
				{Syfte} & {Administratören skall kunna lägga till omgångar (SR1.10)} \\
				\hline
				{Funktion} & {Renderar ett formulär för att redigera Batches-objekt.} \\
				\hline
				{Delkomponenter} & {Inga} \\
				\hline
				{Beroenden} & {BatchesController} \\
				\hline
				{Gränssnitt} & {Ej applicerbart} \\
				\hline
				{Resurser} & {Batch} \\
				\hline
				{Källor} & {Inga} \\
				\hline
				{Process} & {Ej applicerbart} \\
				\hline
				{Data} & {@batch} \\
				\hline
			\end{tabular} \end{table} \FloatBarrier


			\paragraph{5.V25 ShowsController\#index.html}\

			\begin {table} [ht] \begin{tabular} {  p{3.5cm} p{9.6cm} }
				\hline
				{Typ} & {Vy} \\
				\hline
				{Syfte} & {Administratören skall kunna hantera föreställningar (SR1.10, SR1.20)} \\
				\hline
				{Funktion} & {Ger översikt på alla föreställningar som hör till en vald omgång.} \\
				\hline
				{Delkomponenter} & {Inga} \\
				\hline
				{Beroenden} & {ShowsController} \\
				\hline
				{Gränssnitt} & {Ej applicerbart} \\
				\hline
				{Resurser} & {5.L1 admin\_layout.html} \\
				\hline
				{Källor} & {Inga} \\
				\hline
				{Process} & {Ej applicerbart} \\
				\hline
				{Data} & {@shows

@batch - aktuell omgång

@batches - existerande omgångar} \\
				\hline
			\end{tabular} \end{table} \FloatBarrier


			\paragraph{5.V26 ShowsController\#new.html}\

			\begin {table} [ht] \begin{tabular} {  p{3.5cm} p{9.6cm} }
				\hline
				{Typ} & {Vy} \\
				\hline
				{Syfte} & {Administratören skall kunna hantera föreställningar (SR1.10)} \\
				\hline
				{Funktion} & {Lägga till nya föreställningar till en omgång.} \\
				\hline
				{Delkomponenter} & {AD5.3.41} \\
				\hline
				{Beroenden} & {ShowsController} \\
				\hline
				{Gränssnitt} & {Ej applicerbart} \\
				\hline
				{Resurser} & {5.L1 admin\_layout.html} \\
				\hline
				{Källor} & {Inga} \\
				\hline
				{Process} & {Ej applicerbart} \\
				\hline
				{Data} & {@show

@batch - aktuell omgång

@batches - existerande omgångar} \\
				\hline
			\end{tabular} \end{table} \FloatBarrier


			\paragraph{5.V27 ShowsController\#show.html}\

			\begin {table} [ht] \begin{tabular} {  p{3.5cm} p{9.6cm} }
				\hline
				{Typ} & {Vy} \\
				\hline
				{Syfte} & {(Administratören skall kunna hantera föreställningar SR1.10)} \\
				\hline
				{Funktion} & {Visa information om en vald föreställning, till exempel antal lediga platser} \\
				\hline
				{Delkomponenter} & {Inga} \\
				\hline
				{Beroenden} & {ShowsController} \\
				\hline
				{Gränssnitt} & {Ej applicerbart} \\
				\hline
				{Resurser} & {5.L1 admin\_layout.html} \\
				\hline
				{Källor} & {Inga} \\
				\hline
				{Process} & {Ej applicerbart} \\
				\hline
				{Data} & {@batch - aktuell omgång

@batches - existerande omgångar} \\
				\hline
			\end{tabular} \end{table} \FloatBarrier


			\paragraph{5.V28 ShowsController\#edit.html}\

			\begin {table} [ht] \begin{tabular} {  p{3.5cm} p{9.6cm} }
				\hline
				{Typ} & {Vy} \\
				\hline
				{Syfte} & {Administratören skall kunna hantera föreställningar (SR1.10)} \\
				\hline
				{Funktion} & {Ändra information på en vald föreställning, till exempel datum.} \\
				\hline
				{Delkomponenter} & {AD5.3.41} \\
				\hline
				{Beroenden} & {ShowsController} \\
				\hline
				{Gränssnitt} & {Ej applicerbart} \\
				\hline
				{Resurser} & {5.L1 admin\_layout.html} \\
				\hline
				{Källor} & {Inga} \\
				\hline
				{Process} & {Ej applicerbart} \\
				\hline
				{Data} & {@show

@batch - aktuell omgång

@batches - existerande omgångar} \\
				\hline
			\end{tabular} \end{table} \FloatBarrier


			\paragraph{5.P5 ShowsController\#\_form}\

			\begin {table} [ht] \begin{tabular} {  p{3.5cm} p{9.6cm} }
				\hline
				{Typ} & {Partiell vy} \\
				\hline
				{Syfte} & {Administratören skall kunna hantera föreställningar (SR1.10)} \\
				\hline
				{Funktion} & {Renderar ett formulär för att redigera Shows-objekt.} \\
				\hline
				{Delkomponenter} & {Inga} \\
				\hline
				{Beroenden} & {ShowsController} \\
				\hline
				{Gränssnitt} & {Ej applicerbart} \\
				\hline
				{Resurser} & {Show} \\
				\hline
				{Källor} & {Inga} \\
				\hline
				{Process} & {Ej applicerbart} \\
				\hline
				{Data} & {@show

@batch - aktuell omgång

@batches - existerande omgångar} \\
				\hline
			\end{tabular} \end{table} \FloatBarrier


			\paragraph{5.V29 PricingsController\#index.html}\

			\begin {table} [ht] \begin{tabular} {  p{3.5cm} p{9.6cm} }
				\hline
				{Typ} & {Vy} \\
				\hline
				{Syfte} & {Administratören och ekonomichefen ska kunna registrera betalningar (SR1.12)} \\
				\hline
				{Funktion} & {Administratören och ekonomichefen behöver kunna registrera om en betalning är gjord, så att administratören eller eventuellt säljaren kan placera ut bokningens platser.} \\
				\hline
				{Delkomponenter} & {Inga} \\
				\hline
				{Beroenden} & {PricingsController} \\
				\hline
				{Gränssnitt} & {Ej applicerbart} \\
				\hline
				{Resurser} & {5.L1 admin\_layout.html} \\
				\hline
				{Källor} & {Inga} \\
				\hline
				{Process} & {Ej applicerbart} \\
				\hline
				{Data} & {@batch - aktuell omgång

@sections - sektioner i omgångens teater

@pricings - priser för varje sektion- och rabattklasskombination} \\
				\hline
			\end{tabular} \end{table} \FloatBarrier


			\paragraph{5.V30 UsersController\#index.html}\

			\begin {table} [ht] \begin{tabular} {  p{3.5cm} p{9.6cm} }
				\hline
				{Typ} & {Vy} \\
				\hline
				{Syfte} & {Administratören ska kunna hantera systemets användarnas konton (SR1.7, SR1.8)} \\
				\hline
				{Funktion} & {Administratören får en överblick över de användarne som finns i systemet, och deras rättigheter.} \\
				\hline
				{Delkomponenter} & {Inga} \\
				\hline
				{Beroenden} & {UsersController} \\
				\hline
				{Gränssnitt} & {Ej applicerbart} \\
				\hline
				{Resurser} & {5.L1 admin\_layout.html} \\
				\hline
				{Källor} & {Inga} \\
				\hline
				{Process} & {Ej applicerbart} \\
				\hline
				{Data} & {@users} \\
				\hline
			\end{tabular} \end{table} \FloatBarrier


			\paragraph{5.V31 UsersController\#new.html}\

			\begin {table} [ht] \begin{tabular} {  p{3.5cm} p{9.6cm} }
				\hline
				{Typ} & {Vy} \\
				\hline
				{Syfte} & {Administratören ska kunna hantera systemets användarnas konton (SR1.7, SR1.8)} \\
				\hline
				{Funktion} & {Administratören ska kunna skapa en ny användare och sätta dennnes rättigheter i systemet.} \\
				\hline
				{Delkomponenter} & {AD5.P7} \\
				\hline
				{Beroenden} & {UsersController} \\
				\hline
				{Gränssnitt} & {Ej applicerbart} \\
				\hline
				{Resurser} & {5.L1 admin\_layout.html} \\
				\hline
				{Källor} & {Inga} \\
				\hline
				{Process} & {Ej applicerbart} \\
				\hline
				{Data} & {@user} \\
				\hline
			\end{tabular} \end{table} \FloatBarrier


			\paragraph{5.V32 UsersController\#show.html}\

			\begin {table} [ht] \begin{tabular} {  p{3.5cm} p{9.6cm} }
				\hline
				{Typ} & {Vy} \\
				\hline
				{Syfte} & {Administratören ska kunna hantera systemets användarnas konton (SR1.7, SR1.8)} \\
				\hline
				{Funktion} & {Administratören ska kunna granska en enskild användare och dennes rättigheter.} \\
				\hline
				{Delkomponenter} & {Inga} \\
				\hline
				{Beroenden} & {UsersController} \\
				\hline
				{Gränssnitt} & {Ej applicerbart} \\
				\hline
				{Resurser} & {5.L1 admin\_layout.html} \\
				\hline
				{Källor} & {Inga} \\
				\hline
				{Process} & {Ej applicerbart} \\
				\hline
				{Data} & {@user} \\
				\hline
			\end{tabular} \end{table} \FloatBarrier


			\paragraph{5.V33 UsersController\#edit.html}\

			\begin {table} [ht] \begin{tabular} {  p{3.5cm} p{9.6cm} }
				\hline
				{Typ} & {Vy} \\
				\hline
				{Syfte} & {Administratören ska kunna hantera systemets användarnas konton (SR1.7, SR1.8)} \\
				\hline
				{Funktion} & {Administratören ska kunna ändra en användares rättigheter eller lösenord, till exempel säljare som inte ska säljare längre.} \\
				\hline
				{Delkomponenter} & {AD5.P7} \\
				\hline
				{Beroenden} & {UsersController} \\
				\hline
				{Gränssnitt} & {Ej applicerbart} \\
				\hline
				{Resurser} & {5.L1 admin\_layout.html} \\
				\hline
				{Källor} & {Inga} \\
				\hline
				{Process} & {Ej applicerbart} \\
				\hline
				{Data} & {@user} \\
				\hline
			\end{tabular} \end{table} \FloatBarrier


			\paragraph{5.P6 UsersController\#\_form}\

			\begin {table} [ht] \begin{tabular} {  p{3.5cm} p{9.6cm} }
				\hline
				{Typ} & {Partiell vy} \\
				\hline
				{Syfte} & {Administratören ska kunna hantera systemets användarnas konton (SR1.7, SR1.8)} \\
				\hline
				{Funktion} & {Renderar ett formulär för att redigera User-objekt.} \\
				\hline
				{Delkomponenter} & {Inga} \\
				\hline
				{Beroenden} & {UsersController} \\
				\hline
				{Gränssnitt} & {Ej applicerbart} \\
				\hline
				{Resurser} & {User} \\
				\hline
				{Källor} & {Inga} \\
				\hline
				{Process} & {Ej applicerbart} \\
				\hline
				{Data} & {@user} \\
				\hline
			\end{tabular} \end{table} \FloatBarrier


			\paragraph{5.V34 MailingController\#index.html}\

			\begin {table} [ht] \begin{tabular} {  p{3.5cm} p{9.6cm} }
				\hline
				{Typ} & {Vy} \\
				\hline
				{Syfte} & {Administratören ska kunna hantera mailutskick (SR1.14)} \\
				\hline
				{Funktion} & {Ge en överiskt på bland annat sparade mailmallar.} \\
				\hline
				{Delkomponenter} & {Inga} \\
				\hline
				{Beroenden} & {MailingController} \\
				\hline
				{Gränssnitt} & {Ej applicerbart} \\
				\hline
				{Resurser} & {5.L1 admin\_layout.html} \\
				\hline
				{Källor} & {Inga} \\
				\hline
				{Process} & {Ej applicerbart} \\
				\hline
				{Data} & {@mailing

@reciepts = session[:reciepts] - mailmottagare (semipermanent)

@mail = session[:mail] - den aktuella mailmallen, kan vara temporär för att möjliggöra utskick utan mall} \\
				\hline
			\end{tabular} \end{table} \FloatBarrier


			\paragraph{5.V35 MailingController\#new.html}\

			\begin {table} [ht] \begin{tabular} {  p{3.5cm} p{9.6cm} }
				\hline
				{Typ} & {Vy} \\
				\hline
				{Syfte} & {Administratören ska kunna hantera mailutskick (SR1.14)} \\
				\hline
				{Funktion} & {Skapa en ny mailmall för att skicka till en eller flera mottagare.} \\
				\hline
				{Delkomponenter} & {AD5.3.56} \\
				\hline
				{Beroenden} & {MailingController} \\
				\hline
				{Gränssnitt} & {Ej applicerbart} \\
				\hline
				{Resurser} & {5.L1 admin\_layout.html} \\
				\hline
				{Källor} & {Inga} \\
				\hline
				{Process} & {Ej applicerbart} \\
				\hline
				{Data} & {@mail

@reciepts = session[:reciepts] - mailmottagare (semipermanent)

@mail = session[:mail] - den aktuella mailmallen, kan vara temporär för att möjliggöra utskick utan mall} \\
				\hline
			\end{tabular} \end{table} \FloatBarrier


			\paragraph{5.V36 MailingController\#show.html}\

			\begin {table} [ht] \begin{tabular} {  p{3.5cm} p{9.6cm} }
				\hline
				{Typ} & {Vy} \\
				\hline
				{Syfte} & {Administratören ska kunna hantera mailutskick (SR1.14, SR1.15)} \\
				\hline
				{Funktion} & {Visa ett mailmall som sparats. } \\
				\hline
				{Delkomponenter} & {Inga} \\
				\hline
				{Beroenden} & {MailingController} \\
				\hline
				{Gränssnitt} & {Ej applicerbart} \\
				\hline
				{Resurser} & {5.L1 admin\_layout.html} \\
				\hline
				{Källor} & {Inga} \\
				\hline
				{Process} & {Ej applicerbart} \\
				\hline
				{Data} & {@mail

@reciepts = session[:reciepts] - mailmottagare (semipermanent)

@mail = session[:mail] - den aktuella mailmallen, kan vara temporär för att möjliggöra utskick utan mall} \\
				\hline
			\end{tabular} \end{table} \FloatBarrier


			\paragraph{5.V37 MailingController\#edit.html}\

			\begin {table} [ht] \begin{tabular} {  p{3.5cm} p{9.6cm} }
				\hline
				{Typ} & {Vy} \\
				\hline
				{Syfte} & {Administratören ska kunna hantera mailutskick (SR1.14)} \\
				\hline
				{Funktion} & {Ändra mottagare och innehåll i en befintlig mailmall.} \\
				\hline
				{Delkomponenter} & {AD5.3.56} \\
				\hline
				{Beroenden} & {MailingController} \\
				\hline
				{Gränssnitt} & {Ej applicerbart} \\
				\hline
				{Resurser} & {5.L1 admin\_layout.html} \\
				\hline
				{Källor} & {Inga} \\
				\hline
				{Process} & {Ej applicerbart} \\
				\hline
				{Data} & {@mail

@reciepts = session[:reciepts] - mailmottagare (semipermanent)

@mail = session[:mail] - den aktuella mailmallen, kan vara temporär för att möjliggöra utskick utan mall} \\
				\hline
			\end{tabular} \end{table} \FloatBarrier


			\paragraph{5.V38 MailingController\#preview.html}\

			\begin {table} [ht] \begin{tabular} {  p{3.5cm} p{9.6cm} }
				\hline
				{Typ} & {Vy} \\
				\hline
				{Syfte} & {Administratören ska kunna hantera mailutskick (SR1.14)} \\
				\hline
				{Funktion} & {Förhandsgranska de mail som ska skickas innan utskick.} \\
				\hline
				{Delkomponenter} & {Inga} \\
				\hline
				{Beroenden} & {MailingController} \\
				\hline
				{Gränssnitt} & {Ej applicerbart} \\
				\hline
				{Resurser} & {5.L1 admin\_layout.html} \\
				\hline
				{Källor} & {Inga} \\
				\hline
				{Process} & {Ej applicerbart} \\
				\hline
				{Data} & {@mail

@reciepts = session[:reciepts] - mailmottagare (semipermanent)

@mail = session[:mail] - den aktuella mailmallen, kan vara temporär för att möjliggöra utskick utan mall} \\
				\hline
			\end{tabular} \end{table} \FloatBarrier


			\paragraph{5.P7 MailingController\#\_form}\

			\begin {table} [ht] \begin{tabular} {  p{3.5cm} p{9.6cm} }
				\hline
				{Typ} & {Partiell vy} \\
				\hline
				{Syfte} & {Administratören ska kunna hantera mailutskick (SR1.14)} \\
				\hline
				{Funktion} & {Renderar ett formulär för att redigera Mailng-objekt.} \\
				\hline
				{Delkomponenter} & {Inga} \\
				\hline
				{Beroenden} & {MailingController} \\
				\hline
				{Gränssnitt} & {Ej applicerbart} \\
				\hline
				{Resurser} & {Mail} \\
				\hline
				{Källor} & {Inga} \\
				\hline
				{Process} & {Ej applicerbart} \\
				\hline
				{Data} & {@mail

@reciepts = session[:reciepts] - mailmottagare (semipermanent)

@mail = session[:mail] - den aktuella mailmallen, kan vara temporär för att möjliggöra utskick utan mall} \\
				\hline
			\end{tabular} \end{table} \FloatBarrier


		\subsubsection{Säljare}



			\paragraph{5.V39 SalesController\#index.html}\

			\begin {table} [ht] \begin{tabular} {  p{3.5cm} p{9.6cm} }
				\hline
				{Typ} & {Vy} \\
				\hline
				{Syfte} & {En säljare ska kunna genomföra ett kontantköp och lämna ut biljetter (SR1.5, SR1.6, SR1.25)} \\
				\hline
				{Funktion} & {Visa en översikt på säljarens möjliga sysslor.} \\
				\hline
				{Delkomponenter} & {Inga} \\
				\hline
				{Beroenden} & {SalesController} \\
				\hline
				{Gränssnitt} & {Ej applicerbart} \\
				\hline
				{Resurser} & {5.L2 sales\_layout.html} \\
				\hline
				{Källor} & {Inga} \\
				\hline
				{Process} & {Ej applicerbart} \\
				\hline
				{Data} & {@step - steg i bokning

@steps - array med vilka steg som finns

@reservation = session[:reservation] - bokningsdata} \\
				\hline
			\end{tabular} \end{table} \FloatBarrier


			\paragraph{5.V40 SalesController\#show\_choice.html}\

			\begin {table} [ht] \begin{tabular} {  p{3.5cm} p{9.6cm} }
				\hline
				{Typ} & {Vy} \\
				\hline
				{Syfte} & {En säljare ska kunna genomföra ett kontantköp och lämna ut biljetter (SR1.5, SR1.6, SR1.25, SR1.21)} \\
				\hline
				{Funktion} & {Välja vilken föreställning som ärendet gäller för en vald omgång som är aktiv.} \\
				\hline
				{Delkomponenter} & {Inga} \\
				\hline
				{Beroenden} & {SalesController} \\
				\hline
				{Gränssnitt} & {Ej applicerbart} \\
				\hline
				{Resurser} & {5.L2 sales\_layout.html} \\
				\hline
				{Källor} & {Inga} \\
				\hline
				{Process} & {Ej applicerbart} \\
				\hline
				{Data} & {@step - steg i bokning

@steps - array med vilka steg som finns

@reservation = session[:reservation] - bokningsdata} \\
				\hline
			\end{tabular} \end{table} \FloatBarrier


			\paragraph{5.V41 SalesController\#seats.html}\

			\begin {table} [ht] \begin{tabular} {  p{3.5cm} p{9.6cm} }
				\hline
				{Typ} & {Vy} \\
				\hline
				{Syfte} & {En säljare ska kunna genomföra ett kontantköp och lämna ut biljetter (SR1.5, SR1.24 i plus, SR1.25)} \\
				\hline
				{Funktion} & {Välja vilken plats som ärendet gäller.} \\
				\hline
				{Delkomponenter} & {Inga} \\
				\hline
				{Beroenden} & {Om administratören inte har placerat ut den redan bokade platsen så gör säljaren det, men sektionsvalet är redan låst. Om det är ett nytt köp, placerar säljaren också ut platsen. SalesController} \\
				\hline
				{Gränssnitt} & {Ej applicerbart} \\
				\hline
				{Resurser} & {5.L2 sales\_layout.html} \\
				\hline
				{Källor} & {Inga} \\
				\hline
				{Process} & {Ej applicerbart} \\
				\hline
				{Data} & {@step - steg i bokning

@steps - array med vilka steg som finns

@reservation = session[:reservation] - bokningsdata} \\
				\hline
			\end{tabular} \end{table} \FloatBarrier


			\paragraph{5.V42 SalesController\#confirm.html}\

			\begin {table} [ht] \begin{tabular} {  p{3.5cm} p{9.6cm} }
				\hline
				{Typ} & {Vy} \\
				\hline
				{Syfte} & {En säljare ska kunna genomföra ett kontantköp och lämna ut biljetter (SR1.5, SR1.6, SR1.26)} \\
				\hline
				{Funktion} & {Fråga användaren om de angivna uppgifterna är korrekta och skall införas i systemet.} \\
				\hline
				{Delkomponenter} & {Inga} \\
				\hline
				{Beroenden} & {SalesController} \\
				\hline
				{Gränssnitt} & {Ej applicerbart} \\
				\hline
				{Resurser} & {5.L2 sales\_layout.html} \\
				\hline
				{Källor} & {Inga} \\
				\hline
				{Process} & {Ej applicerbart} \\
				\hline
				{Data} & {@step - steg i bokning

@steps - array med vilka steg som finns

@reservation = session[:reservation] - bokningsdata} \\
				\hline
			\end{tabular} \end{table} \FloatBarrier


			\paragraph{5.V43 SalesController\#retrieve.html}\

			\begin {table} [ht] \begin{tabular} {  p{3.5cm} p{9.6cm} }
				\hline
				{Typ} & {Vy} \\
				\hline
				{Syfte} & {En säljare ska kunna genomföra ett kontantköp och lämna ut biljetter (SR1.5, SR1.6)} \\
				\hline
				{Funktion} & {Uppmana säljaren att gå och hämta en viss biljett.} \\
				\hline
				{Delkomponenter} & {Inga} \\
				\hline
				{Beroenden} & {SalesController} \\
				\hline
				{Gränssnitt} & {Ej applicerbart} \\
				\hline
				{Resurser} & {5.L2 sales\_layout.html} \\
				\hline
				{Källor} & {Inga} \\
				\hline
				{Process} & {Ej applicerbart} \\
				\hline
				{Data} & {@step - steg i bokning

@steps - array med vilka steg som finns

@reservation = session[:reservation] - bokningsdata} \\
				\hline
			\end{tabular} \end{table} \FloatBarrier


			\paragraph{5.V44 SalesController\#payment.html}\

			\begin {table} [ht] \begin{tabular} {  p{3.5cm} p{9.6cm} }
				\hline
				{Typ} & {Vy} \\
				\hline
				{Syfte} & {En säljare ska kunna genomföra ett kontantköp och lämna ut biljetter (SR1.5, SR1.6)} \\
				\hline
				{Funktion} & {Uppmana säljaren att ta betalt för bokningen.} \\
				\hline
				{Delkomponenter} & {Inga} \\
				\hline
				{Beroenden} & {SalesController} \\
				\hline
				{Gränssnitt} & {Ej applicerbart} \\
				\hline
				{Resurser} & {5.L2 sales\_layout.html} \\
				\hline
				{Källor} & {Inga} \\
				\hline
				{Process} & {Ej applicerbart} \\
				\hline
				{Data} & {@step - steg i bokning

@steps - array med vilka steg som finns

@reservation = session[:reservation] - bokningsdata} \\
				\hline
			\end{tabular} \end{table} \FloatBarrier


			\paragraph{5.V45 SalesController\#hand\_out.html}\

			\begin {table} [ht] \begin{tabular} {  p{3.5cm} p{9.6cm} }
				\hline
				{Typ} & {Vy} \\
				\hline
				{Syfte} & {En säljare ska kunna genomföra ett kontantköp och lämna ut biljetter (SR1.5, SR1.6)} \\
				\hline
				{Funktion} & {Uppmana säljaren att lämna ut biljetten.} \\
				\hline
				{Delkomponenter} & {Inga} \\
				\hline
				{Beroenden} & {SalesController} \\
				\hline
				{Gränssnitt} & {Ej applicerbart} \\
				\hline
				{Resurser} & {5.L2 sales\_layout.html} \\
				\hline
				{Källor} & {Inga} \\
				\hline
				{Process} & {Ej applicerbart} \\
				\hline
				{Data} & {@step - steg i bokning

@steps - array med vilka steg som finns

@reservation = session[:reservation] - bokningsdata} \\
				\hline
			\end{tabular} \end{table} \FloatBarrier


			\paragraph{5.V46 SalesController\#search.html}\

			\begin {table} [ht] \begin{tabular} {  p{3.5cm} p{9.6cm} }
				\hline
				{Typ} & {Vy} \\
				\hline
				{Syfte} & {En säljare ska kunna genomföra ett kontantköp och lämna ut biljetter (SR1.5, SR1.6, SR1.25, SR1.20)} \\
				\hline
				{Funktion} & {Låter säljaren söka bland bokningar med exempelvis namn.} \\
				\hline
				{Delkomponenter} & {Inga} \\
				\hline
				{Beroenden} & {SalesController} \\
				\hline
				{Gränssnitt} & {Ej applicerbart} \\
				\hline
				{Resurser} & {5.L2 sales\_layout.html} \\
				\hline
				{Källor} & {Inga} \\
				\hline
				{Process} & {Ej applicerbart} \\
				\hline
				{Data} & {@step - steg i bokning

@steps - array med vilka steg som finns

@reservation = session[:reservation] - bokningsdata} \\
				\hline
			\end{tabular} \end{table} \FloatBarrier


		\subsubsection{Kund}



			\paragraph{5.V47 BookingController\#show\_choice.html}\

			\begin {table} [ht] \begin{tabular} {  p{3.5cm} p{9.6cm} }
				\hline
				{Typ} & {Vy} \\
				\hline
				{Syfte} & {Kunden ska kunna boka sin biljett via Kårspexets webbsida (SR1.2, SR1.21)} \\
				\hline
				{Funktion} & {Kunden får ett val om vilken föreställning i den nuvarande omgång som denne vill gå på.} \\
				\hline
				{Delkomponenter} & {Inga} \\
				\hline
				{Beroenden} & {BookingController} \\
				\hline
				{Gränssnitt} & {Ej applicerbart} \\
				\hline
				{Resurser} & {5.L4 customer\_layout.html } \\
				\hline
				{Källor} & {Inga} \\
				\hline
				{Process} & {Ej applicerbart} \\
				\hline
				{Data} & {@step - steg i bokning

@steps - array med vilka steg som finns

@reservation - bokningsdata} \\
				\hline
			\end{tabular} \end{table} \FloatBarrier


			\paragraph{5.V48 BookingController\#section.html}\

			\begin {table} [ht] \begin{tabular} {  p{3.5cm} p{9.6cm} }
				\hline
				{Typ} & {Vy} \\
				\hline
				{Syfte} & {Kunden ska kunna boka sin biljett via Kårspexets webbsida (SR1.2, SR1.24 i  plus)} \\
				\hline
				{Funktion} & {Kunden väljer vilken sektion som denne vill boka på den valda föreställningen.} \\
				\hline
				{Delkomponenter} & {Inga} \\
				\hline
				{Beroenden} & {BookingController} \\
				\hline
				{Gränssnitt} & {Ej applicerbart} \\
				\hline
				{Resurser} & {5.L4 customer\_layout.html} \\
				\hline
				{Källor} & {Inga} \\
				\hline
				{Process} & {Ej applicerbart} \\
				\hline
				{Data} & {@step - steg i bokning

@steps - array med vilka steg som finns

@reservation - bokningsdata} \\
				\hline
			\end{tabular} \end{table} \FloatBarrier


			\paragraph{5.V49 BookingController\#payment.html}\

			\begin {table} [ht] \begin{tabular} {  p{3.5cm} p{9.6cm} }
				\hline
				{Typ} & {Vy} \\
				\hline
				{Syfte} & {Kunden ska kunna boka sin biljett via Kårspexets webbsida (SR1.2, SR1.4 i delux)} \\
				\hline
				{Funktion} & {Kunden fyller i sin betalningsinformation och hur kunden vill hämta sin biljett.} \\
				\hline
				{Delkomponenter} & {Inga} \\
				\hline
				{Beroenden} & {BookingController} \\
				\hline
				{Gränssnitt} & {Ej applicerbart} \\
				\hline
				{Resurser} & {5.L4 customer\_layout.html} \\
				\hline
				{Källor} & {Inga} \\
				\hline
				{Process} & {Ej applicerbart} \\
				\hline
				{Data} & {@step - steg i bokning

@steps - array med vilka steg som finns

@reservation - bokningsdata} \\
				\hline
			\end{tabular} \end{table} \FloatBarrier


			\paragraph{5.V50 BookingController\#confirm.html}\

			\begin {table} [ht] \begin{tabular} {  p{3.5cm} p{9.6cm} }
				\hline
				{Typ} & {Vy} \\
				\hline
				{Syfte} & {Kunden ska kunna boka sin biljett via Kårspexets webbsida (SR1.2, SR1.26)} \\
				\hline
				{Funktion} & {Kunden bekräftar bokningen och alla betalningsuppgifter.} \\
				\hline
				{Delkomponenter} & {Inga} \\
				\hline
				{Beroenden} & {BookingController} \\
				\hline
				{Gränssnitt} & {Ej applicerbart} \\
				\hline
				{Resurser} & {5.L4 customer\_layout.html } \\
				\hline
				{Källor} & {Inga} \\
				\hline
				{Process} & {Ej applicerbart} \\
				\hline
				{Data} & {@step - steg i bokning

@steps - array med vilka steg som finns

@reservation - bokningsdata} \\
				\hline
			\end{tabular} \end{table} \FloatBarrier


			\paragraph{5.V51 BookingController\#cancel\_booking.html}\

			\begin {table} [ht] \begin{tabular} {  p{3.5cm} p{9.6cm} }
				\hline
				{Typ} & {Vy} \\
				\hline
				{Syfte} & {Kunden ska kunna avboka sin biljett via Kårspexets webbsida (SR1.2, SR1.3)} \\
				\hline
				{Funktion} & {Kunden avbokar sin bokning via en länk i sitt bekräftelsemail.} \\
				\hline
				{Delkomponenter} & {Inga} \\
				\hline
				{Beroenden} & {BookingController} \\
				\hline
				{Gränssnitt} & {Ej applicerbart} \\
				\hline
				{Resurser} & {Inga} \\
				\hline
				{Källor} & {Inga} \\
				\hline
				{Process} & {Ej applicerbart} \\
				\hline
				{Data} & {@step - steg i bokning

@steps - array med vilka steg som finns

@reservation - bokningsdata} \\
				\hline
			\end{tabular} \end{table} \FloatBarrier


			\paragraph{5.V52 BookingController\#destroy.html}\

			\begin {table} [ht] \begin{tabular} {  p{3.5cm} p{9.6cm} }
				\hline
				{Typ} & {Vy} \\
				\hline
				{Syfte} & {Kunden ska kunna avboka sin biljett via Kårspexets webbsida (SR1.2, SR1.3)} \\
				\hline
				{Funktion} & {Kunden terminerar bokningen från databasen.} \\
				\hline
				{Delkomponenter} & {Inga} \\
				\hline
				{Beroenden} & {BookingController} \\
				\hline
				{Gränssnitt} & {Ej applicerbart} \\
				\hline
				{Resurser} & {5.L4 customer\_layout.html} \\
				\hline
				{Källor} & {Inga} \\
				\hline
				{Process} & {Ej applicerbart} \\
				\hline
				{Data} & {@step - steg i bokning

@steps - array med vilka steg som finns

@reservation - bokningsdata} \\
				\hline
			\end{tabular} \end{table} \FloatBarrier


	\subsection{Övriga komponenter}



			\paragraph{5.X1 InheritedResources}\

			\begin {table} [ht] \begin{tabular} {  p{3.5cm} p{9.6cm} }
				\hline
				{Typ} & {Övrig komponent (Rails-gem)} \\
				\hline
				{Syfte} & {Abstraherar bort hantering av resurser för kontrollers som hanterar dessa för att minimera kodupprepning.} \\
				\hline
				{Funktion} & {Tillhandahåller alla de (eller valfria) standardactions för kontroller som hanterar resurser och ärver från denna klass.} \\
				\hline
				{Delkomponenter} & {Inga} \\
				\hline
				{Beroenden} & {Inga} \\
				\hline
				{Gränssnitt} & {Alla kontrollers som ärver från denna klass får standardactions för resurser ({\tt index}, {\tt show}, {\tt new}, {\tt create}, {\tt edit}, {\tt update} samt {\tt destroy}) fördefinierade om inget annat anges.

{\tt actions} - Hjälpmetod för att specificera vilka actions som ska fördefinieras.

Hjälpmetoder som görs tillgängliga i vyer:

{\tt resource} - Enskild instans av den representerade resursen (i {\tt show}, {\tt new}, {\tt create}, {\tt edit}, {\tt update}, {\tt destroy}).

{\tt collection} - En array innehållandes alla objekt av resursen (endast i {\tt index}).

{\tt resource\_class} - En referens till den modell (klassdefinitionen) som hanteras som resurs.} \\
				\hline
				{Resurser} & {InheritedResources::Base (ingår i Rails-gemet inherited\_resources).} \\
				\hline
				{Källor} & {inherited\_resources: https://github.com/josevalim/inherited\_resources} \\
				\hline
				{Process} & {Inga} \\
				\hline
				{Data} & {Instansvariabler av formatet @modellnamn (ex. @reservation för ReservationsController) respektive @modellnamn\_i\_plural för index-actions (ex. @reservations).} \\
				\hline
			\end{tabular} \end{table} \FloatBarrier


			\paragraph{5.X2 AuthLogic}\

			\begin {table} [ht] \begin{tabular} {  p{3.5cm} p{9.6cm} }
				\hline
				{Typ} & {Övrig komponent (Rails-gem)} \\
				\hline
				{Syfte} & {Hanterar autentisering och användarsessioner (SR1.1).} \\
				\hline
				{Funktion} & {Tillhandahåller funktioner för att autentisera användare utifrån existerande modeller.} \\
				\hline
				{Delkomponenter} & {UserSession.} \\
				\hline
				{Beroenden} & {Inga} \\
				\hline
				{Gränssnitt} & {Authlogic::Session::Base\#find - returnerar aktiv användarsession eller försöker initiera en ny om ingen redan finns

ActiveRecord::Base\#acts\_as\_authentic - instruerar Authlogic att den aktuella modellen representerar en typ av autentisering} \\
				\hline
				{Resurser} & {Authlogic (ingår i Rails-gemet authlogic), UserSession och User.} \\
				\hline
				{Källor} & {authlogic: https://github.com/binarylogic/authlogic} \\
				\hline
				{Process} & {Inga} \\
				\hline
				{Data} & {Inga} \\
				\hline
			\end{tabular} \end{table} \FloatBarrier


			\paragraph{5.X3 CanCan}\

			\begin {table} [ht] \begin{tabular} {  p{3.5cm} p{9.6cm} }
				\hline
				{Typ} & {Övrig komponent (Rails-gem)} \\
				\hline
				{Syfte} & {Definierar och hanterar användarroller (SR9.2).} \\
				\hline
				{Funktion} & {Tillhandahåller metoder för att bestämma vilka användarroller som finns samt vad dessa har rättigheter till. Definierar även hjälpmetoder för att testa rättigheter mot den inloggade användaren.} \\
				\hline
				{Delkomponenter} & {Ability} \\
				\hline
				{Beroenden} & {En metod {\tt ApplicationController\#current\_user} måste definieras så att den returnerar ett objekt.} \\
				\hline
				{Gränssnitt} & {{\tt can?} - returnerar huruvida användaren har rättigheter för att utföra en viss handling (tillgänglig i vyer och kontroller)

{\tt cannot?} - inversen av {\tt can?}

{\tt ApplicationController\#authorize!} - kastar en exception om användaren inte har rättighet att utföra den handling som angivits

{\tt ApplicationController\#load\_and\_authorize\_resource} - ladda in och autorisera alla standardactions för resurser automatiskt} \\
				\hline
				{Resurser} & {CanCan (ingår i Rails-gemet cancan) och Ability.} \\
				\hline
				{Källor} & {cancan: https://github.com/ryanb/cancan} \\
				\hline
				{Process} & {1. Rättigheter fördefinieras i Ability-modellen.

2. {\tt can?(:action, Object)} anropas (:action = någon fördefinierad action, Object är någon modell alternativt en modellinstans).

3. CanCan hämtar aktiv användare via {\tt ApplicationController\#current\_user}.

4. Användarens rättigheter undersöks av CanCan genom att skicka denna till Ability-modellen.

5. {\tt can?}-metoden returnerar huruvida rättigheter finns (eller saknas) för att utföra handlingen.} \\
				\hline
				{Data} & {Inga} \\
				\hline
			\end{tabular} \end{table} \FloatBarrier


			\paragraph{5.X4 ActiveRecord}\

			\begin {table} [ht] \begin{tabular} {  p{3.5cm} p{9.6cm} }
				\hline
				{Typ} & {Övrig komponent (Rails)} \\
				\hline
				{Syfte} & {ActiveRecord är en existerande baskomponent i Rails-ramverket som bland annat fungerar som ett gränssnitt till databasens tabeller (SR1.29). } \\
				\hline
				{Funktion} & {Tillhandahåller gränssnitt för sökningar och manipulation av objekt databasen. Implicit ingår databas och databasschema i varje modell.} \\
				\hline
				{Delkomponenter} & { Inga} \\
				\hline
				{Gränssnitt} & {Objektorienterat gränssnitt enligt active-record-patternen. En modell är en klass som ärver från ActiveRecord::Base och motsvarar en tabell i databasen. En instans av klassen motsvarar en rad i tabellen.

Följande är några av de metoder som definieras av {\tt ActiveRecord::Base} och är tillgängliga för alla modeller.

{\tt all} - hämta alla instanser av modellen (alla rader i databastabellen).

{\tt find} - gör diverse sökningar i databasen och returnerar de instanser som matchar.

{\tt valid?} - kontrollera att objektets data är korrekt. Vilka kontroller som görs definieras av varje modell.

{\tt save} - lagrar/uppdaterar en instans av ett modell.

{\tt update} - uppdaterar attributer för en instans i databastabellen.

{\tt destroy} - raderar objektet i databasen.

Dessutom finns accessors för varje attribut hos modellen. Dessa definieras dynamiskt.} \\
				\hline
				{Beroenden} & {Inga} \\
				\hline
				{Resurser} & {Inga} \\
				\hline
				{Källor} & {ActiveRecord: \url{http://ar.rubyonrails.org}} \\
				\hline
				{Process} & {Följande typscenario visar skapande och uppdatering av en instans av modellen {\tt M}.

{\tt m = M.new}

{\tt m.att1 = ``value''}

{\tt if m.valid?}

{\tt then m.save}} \\
				\hline
				{Data} & {Alla modeller har datafälten {\tt created\_at} respektive {\tt updated\_at} av typen {\tt timestamp}. Dessa uppdateras automatiskt av Rails vid skapande och uppdatering av objektet.

Alla modeller har även attributet {\tt id} som primärnyckel.

För övriga attribut se respektive modell. Enkla attribut listas, relationer till andra modeller beskrivs i delkomponenter.} \\
				\hline
			\end{tabular} \end{table} \FloatBarrier


\clearpage
\section{Genomförbarhet- och resursuppskattning}



	\subsection{Sammanfattning av slutsatser av en genomförbarhetsundersökning av arkitekturmodellen.}


	Arkitekturen Nyx kommer använda kallas Model-View-Controller, MVC. Alla komponenter under respektive punkt finns angivna nedan.

	I designen ser vi endast en del som kommer att vara mer komplicerad, nämligen stolsplaceringen. Det är dock fullt genomförbart, eftersom det inte är avancerade tekniker som används, utan att det är mycket data som ska sammankopplas.

	Modelleringen är redan gjord och simuleringen borde inte vara några problem, vi kommer nämligen använda rätt arkitektur från början.

	Slutsatsen är således att projektet är fullt genomförbart med den här arkitekturmodellen.

	\subsection{Komponentprioritering}



		\subsubsection{Modeller}


		\begin {table} [ht] \begin{tabular} {  p{4cm} p{2cm} }
			\hline
			{ Komponent } & { Prioritet } \\
			\hline
			{ 5.M1 Reservation } & { Standard } \\
			\hline
			{ 5.M2 ReservationCounter} & { Standard } \\
			\hline
			{ 5.M3 Placement} & { Standard } \\
			\hline
			{ 5.M4 Theater } & { Standard } \\
			\hline
			{ 5.M5 Section } & { Standard } \\
			\hline
			{ 5.M6 Seat} & { Standard } \\
			\hline
			{ 5.M7 Batch } & { Standard } \\
			\hline
			{ 5.M8 Show} & { Standard } \\
			\hline
			{ 5.M9 Pricing } & { Standard } \\
			\hline
			{ 5.M10 User } & { Standard } \\
			\hline
			{ 5.M11 MailTemplate} & { Standard } \\
			\hline
			{ 5.M12 UserSession } & { Standard } \\
			\hline
			{ 5.M13 Ability } & { Standard } \\
			\hline
		\end{tabular} \end{table} \FloatBarrier


		\subsubsection{Kontroller}


		\begin {table} [ht] \begin{tabular} {  p{4cm} p{2cm} }
			\hline
			{ Komponent } & { Prioritet } \\
			\hline
			{ 5.C1 ApplicationController } & { Standard } \\
			\hline
			{ 5.C2 BookingController } & { Standard } \\
			\hline
			{ 5.C3 SessionController } & { Standard } \\
			\hline
			{ 5.C4 AdminIndexController } & { Standard } \\
			\hline
			{ 5.C5 ReservationsController } & { Standard } \\
			\hline
			{ 5.C6 PlacementsController } & { Standard } \\
			\hline
			{ 5.C7 TheatersController } & { Standard } \\
			\hline
			{ 5.C8 SectionsController } & { Standard } \\
			\hline
			{ 5.C9 SeatsController } & { Standard } \\
			\hline
			{ 5.C10 BatchesController } & { Standard } \\
			\hline
			{ 5.C11 ShowsController } & { Standard } \\
			\hline
			{ 5.C12 PricingsController } & { Standard } \\
			\hline
			{ 5.C13 UsersController } & { Standard } \\
			\hline
			{ 5.C14 MailingController } & { Standard } \\
			\hline
			{ 5.C15 SalesController } & { Standard } \\
			\hline
		\end{tabular} \end{table} \FloatBarrier


		\subsubsection{Vyer}


		\begin {table} [ht] \begin{tabular} {  p{4cm} p{2cm} }
			\hline
			{ Komponent } & { Prioritet } \\
			\hline
			{ 5.L1 admin\_layout.html } & { Standard } \\
			\hline
			{ 5.L2 sales\_layout.html } & { Standard } \\
			\hline
			{ 5.L3 simple\_layout.html } & { Standard } \\
			\hline
			{ 5.L4 customer\_layout.html } & { Standard } \\
			\hline
			{ 5.V1 SessionController\#new.html } & { Standard } \\
			\hline
			{ 5.V2 AdminIndexController\#index.html } & { Plus } \\
			\hline
			{ 5.V3 AdminIndexController\#stat\_standard.html } & { Standard } \\
			\hline
			{ 5.V4 AdminIndexController\#stat\_plus.html } & { Plus } \\
			\hline
			{ 5.V5 AdminIndexController\#stat\_deluxe.html } & { Deluxe } \\
			\hline
			{ 5.V6 ReservationsController\#index.html } & { Standard } \\
			\hline
			{ 5.V7 ReservationsController\#new.html } & { Standard } \\
			\hline
			{ 5.V8 ReservationsController\#show.html } & { Standard } \\
			\hline
			{ 5.V9 ReservationsController\#edit.html } & { Standard } \\
			\hline
			{ 5.P1 ReservationsController\#\_form.html } & { Standard } \\
			\hline
			{ 5.V10 PlacementsController\#new.html } & { Standard } \\
			\hline
			{ 5.V11 PlacementsController\#edit.html } & { Standard } \\
			\hline
			{ 5.V12 TheatersController\#index.html } & { Standard } \\
			\hline
			{ 5.V13 TheatersController\#new.html } & { Standard } \\
			\hline
			{ 5.V14 TheatersController\#show.html } & { Standard } \\
			\hline
			{ 5.V15 TheatersController\#edit.html } & { Standard } \\
			\hline
			{ 5.P2 TheatersController\#\_form } & { Standard } \\
			\hline
			{ 5.V16 SectionsController\#index.html } & { Standard } \\
			\hline
			{ 5.V17 SectionsController\#new.html } & { Standard } \\
			\hline
			{ 5.V18 SectionsController\#show.html } & { Standard } \\
			\hline
			{ 5.V19 SectionsController\#edit.html } & { Standard } \\
			\hline
			{ 5.P3 SectionsController\#\_form } & { Standard } \\
			\hline
			{ 5.V20 SeatsController\#index.html } & { Standard } \\
			\hline
			{ 5.V21 BatchesController\#index.html } & { Standard } \\
			\hline
			{ 5.V22 BatchesController\#new.html } & { Standard } \\
			\hline
			{ 5.V23 BatchesController\#show.html } & { Standard } \\
			\hline
			{ 5.V24 BatchesController\#edit.html } & { Standard } \\
			\hline
			{ 5.P4 BatchesController\#\_form } & { Standard } \\
			\hline
			{ 5.V25 ShowsController\#index.html } & { Standard } \\
			\hline
			{ 5.V26 ShowsController\#new.html } & { Standard } \\
			\hline
			{ 5.V27 ShowsController\#show.html } & { Standard } \\
			\hline
			{ 5.V28 ShowsController\#edit.html } & { Standard } \\
			\hline
			{ 5.P5 ShowsController\#\_form } & { Standard } \\
			\hline
			{ 5.V29 PricingsController\#index.html } & { Standard } \\
			\hline
			{ 5.V30 UsersController\#index.html } & { Standard } \\
			\hline
			{ 5.V31 UsersController\#new.html } & { Standard } \\
			\hline
			{ 5.V32 UsersController\#show.html } & { Standard } \\
			\hline
			{ 5.V33 UsersController\#edit.html } & { Standard } \\
			\hline
			{ 5.P6 UsersController\#\_form } & { Standard } \\
			\hline
			{ 5.V34 MailingController\#index.html } & { Standard } \\
			\hline
			{ 5.V35 MailingController\#new.html } & { Standard } \\
			\hline
			{ 5.V36 MailingController\#show.html } & { Standard } \\
			\hline
			{ 5.V37 MailingController\#edit.html } & { Standard } \\
			\hline
			{ 5.V38 MailingController\#preview.html } & { Standard } \\
			\hline
			{ 5.P7 MailingController\#\_form } & { Standard } \\
			\hline
			{ 5.V39 SalesController\#index.html } & { Standard } \\
			\hline
			{ 5.V40 SalesController\#show\_choice.html } & { Standard } \\
			\hline
			{ 5.V41 SalesController\#seats.html } & { Standard } \\
			\hline
			{ 5.V42 SalesController\#confirm.html } & { Standard } \\
			\hline
			{ 5.V43 SalesController\#retrieve.html } & { Standard } \\
			\hline
			{ 5.V44 SalesController\#payment.html } & { Standard } \\
			\hline
			{ 5.V45 SalesController\#hand\_out.html } & { Standard } \\
			\hline
			{ 5.V46 SalesController\#search.html } & { Standard } \\
			\hline
			{ 5.V47 BookingController\#show\_choice.html } & { Standard } \\
			\hline
			{ 5.V48 BookingController\#section.html } & { Standard } \\
			\hline
			{ 5.V49 BookingController\#payment.html } & { Standard } \\
			\hline
			{ 5.V50 BookingController\#confirm.html } & { Standard } \\
			\hline
			{ 5.V51 BookingController\#cancel\_booking.html } & { Standard } \\
			\hline
			{ 5.V52 BookingController\#destroy.html } & { Standard } \\
			\hline
		\end{tabular} \end{table} \FloatBarrier


		\subsubsection{Övriga komponenter}


		\begin {table} [ht] \begin{tabular} {  p{4cm} p{2cm} }
			\hline
			{ Komponent } & { Prioritet } \\
			\hline
			{ 5.X1 InheritedResources} & { Standard } \\
			\hline
			{ 5.X2 AuthLogic } & { Standard } \\
			\hline
			{ 5.X3 CanCan} & { Standard } \\
			\hline
			{ 5.X4 ActiveRecord} & { Standard } \\
			\hline
		\end{tabular} \end{table} \FloatBarrier


	\subsection{Framtida projektuppgifter med tidsuppskattning}


	\begin {table} [ht] \begin{tabular} {  p{1cm} p{4cm} p {9cm} p{1cm} }
		\hline
		{ Nr } & { Benämning } & { Beskrivning } & { Tidsuppskattning } \\
		\hline
		{ 1 } & { Modellskelett } & { Skelett inklusive relationer och attribut } & { 5h } \\
		\hline
		{ 2 } & { Skelett för controllers } & { Definera tomma actions } & { 5h } \\
		\hline
		{ 3 } & { Routes} & { Skapa standardroutes för resurser } & { 2h } \\
		\hline
		{ 4 } & { Controller logik } & { Enkel controller logik för resurser } & { 15h } \\
		\hline
		{ 5 } & { Formulär } & { Grundläggande vyer (formulär) för att redigera omgångar/föreställningar } & { 10h } \\
		\hline
		{ 6 } & { Enkla vyer } & { Enkla vyer för index av omgång/föreställning } & { 3h } \\
		\hline
		{ 7 } & { Adminvyer } & { Lite-version av vyer för administrering av teatrar(enbart attributer) } & { 3h } \\
		\hline
		{ 8 } & { Redigeringsvy } & { Vyer för att redigera sektioner (inklusive polygonkordinater) } & { 8h } \\
		\hline
		{ 9 } & { Script för platser } & { Vy och javascript för utplacering av platser(seats) } & { 15h } \\
		\hline
		{ 10 } & { Adminlayout } & { Designa och implementera administratörslayouten } & { 8h } \\
		\hline
		{ 13 } & { BookingController } & { Implementera grundläggande stegnavigering med databeständighet } & { 20h } \\
		\hline
		{ 14 } & { Kundbokningslayout } & { Layouter för kundens bokningsprocess } & { 10h } \\
		\hline
		{ 15 } & { Bokningsvyer } & { Vyer för bokningsprocedyrens olika steg } & { 20h } \\
		\hline
		{ 12 } & { Vy för att redigering av bokningar } & { 5h } \\
		\hline
		{ 16 } & { Bokningsgränssnitt } & { Gränssnitt för placering av bokningar } & { 20h } \\
		\hline
		{ 11 } & { Enkelt bokningsfilter } & { Listning och en enkel filtring av bokningar } & { 10h } \\
		\hline
		{ 22 } & { Avancerat bokningsfilter } & { Avancerat filter av bokningar } & { 10h } \\
		\hline
		{ 17 } & { Prishantering } & { Hantering av priser (formulär och genering) } & { 15h } \\
		\hline
		{ 23 } & { Administrera mailmallar } & { Administreringsvy för mailmallar(skapa/redigera/ta bort) } & { 10h } \\
		\hline
		{ 24 } & { MailingController } & { MailingController - resurshantering } & { 3h } \\
		\hline
		{ 25 } & { Mailmallar } & { Generering av mail utifrån bokningar } & { 15h } \\
		\hline
		{ 18 } & { } & { Validerings- och övriga krav på omgångar, föreställningar och prissättningar } & { 10h } \\
		\hline
		{ 19 } & { } & { Validerings- och integritetskrav fö teatrar, sektioner och platser } & { 15h } \\
		\hline
		{ 20 } & { } & { Krav för bokningar(reservation, ResCounter) } & { 10h } \\
		\hline
		{ 21 } & { } & { Integritetskrav och övrig logik för placeringar och dess relationer } & { 10h } \\
		\hline
		{ 26 } & { Mailutskick } & { Mailutskick från bokningslistan till utskicken med data inlagd i vald mall } & { 15h } \\
		\hline
		{ 27 } & { } & { Förhandsgranskning av mailutskick - vy och logik } & { 10h } \\
		\hline
	\end{tabular} \end{table} \FloatBarrier


	\subsection{Beroenden mellan uppgifter}


	Vad gäller komponenterna och då speciellt vyer kan man se beroenden mellan vy och kontroller på följande sätt; AdminIndexController\#new.html har AdminIndexController som beroende.

	\subsection{Uppgiftsplanering}


	\begin {table} [ht] \begin{tabular} {  p{5cm} p{2cm} p{2cm} p{2cm} p{2cm} p{2cm} p{2cm} p{2cm} p{2cm} p{2cm} p{2cm} }
		\hline
		{ Namn } & { Vecka 1} & { Vecka 2} & { Vecka 3} & { Vecka 4} & { Vecka 5} & { Vecka 6} & { Vecka 7} & { Vecka 8} & { Vecka 9} & { Vecka 10 } \\
		\hline
		{ Projektmedlem 1 \emph{mögel} } & { nått } & { nått } & { } & {} & { } & {} & { } & {} & { } & { Testar } \\
		\hline
		{ Projektmedlem 2 \emph{tarandi}} & { nått } & { nått } & { } & {} & { } & {} & { } & {} & { } & { Testar } \\
		\hline
		{ Projektmedlem 3 \emph{perka}} & { nått } & { nått } & { } & {} & { } & {} & { } & {} & { } & { Testar } \\
		\hline
		{ Projektmedlem 4 \emph{kalle}} & { nått } & { nått } & { } & {} & { } & {} & { } & {} & { } & { Testar } \\
		\hline
		{ Projektmedlem 5 \emph{erik}} & { nått } & { nått } & { } & {} & { } & {} & { } & {} & { } & { Testar } \\
		\hline
		{ Projektmedlem 6 \emph{daniel}} & { nått } & { nått } & { } & {} & { } & {} & { } & {} & { } & { Testar } \\
		\hline
		{ Projektmedlem 7 \emph{andré}} & { nått } & { nått } & { } & {} & { } & {} & { } & {} & { } & { Testar } \\
		\hline
		{ Projektmedlem 8 \emph{lemming}} & { nått } & { nått } & { } & {} & { } & {} & { } & {} & { } & { Testar } \\
		\hline
		{ Projektmedlem 9 \emph{johan}} & { nått } & { nått } & { } & {} & { } & {} & { } & {} & { } & { Testar } \\
		\hline
		{ Projektmedlem 10 \emph{rasmus}} & { nått } & { nått } & { } & {} & { } & {} & { } & {} & { } & { Testar } \\
		\hline
		{ Projektmedlem 11 \emph{mia}} & { nått } & { nått } & { } & {} & { } & {} & { } & {} & { } & { Testar } \\
		\hline
	\end{tabular} \end{table} \FloatBarrier


	\subsection{Möjliga risker}


	Projektplaneringen visar att tiden kommer att räcka för att slutföra projektet och hittills har samarbetet fungerat väl. Vi har ingen nyckelkompetens som gör oss beroende av en enstaka projektmedlem. Den största risken ligger i att Kårspexet inte anser kraven uppfyllda eller kommer med ändringar i kraven. Då kommer det vara ont om tid för att korrigera dessa och det ger oss ingen marginal för fel. Vi kommer undvika detta genom ett nära samarbete med Kårspexet under implementationen. De kommer dessutom att ha en viktig roll under testningen som skall inledas i god tid.

	Risk att vi implementerar fel funktionalitet (annat än det efterfrågade) och därmed förlorar tid.

	Risk att vi inte hinner implementera alla funktioner som efterfrågas

	Risk att vi implementerar saker i fel ordning så att vi förlorar tid på att vänta på varandra. Här kommer nyckelkompetens in i beräkningarna.

	Risk att vår utvecklingsserver krånglar.

	Risk att det krånglar när vi ska installera vårt system på kårspexets server.

\clearpage
\section{Spårningsmatris mellan mjukvarukrav och strukturella krav}


Följande tabell listar mjukvarukrav från SRD:n. För en enklare överblick över vilka mjukvarukrav ett specifikt systemkrav härstammar ifrån, se de specifika kraven i sektion 5.

	\subsection{Funktionella krav}


	\begin {table} [ht] \begin{tabular} {  p{5cm} p{2cm} p{2cm} p{2cm} p{2cm} p{2cm} p{2cm} p{2cm} p{2cm} p{2cm} p{2cm} }
		\hline
		{Kravkälla} & {Krav} \\
		\hline
		{SR1.1} & {AR.M12, AR.M13, AR.C1, AR.C3, AR.L3, AR.V1, AR.X2} \\
		\hline
		{SR1.2} & {AR.M1, AR.M2, AR.C2, AR.L4, AR.V47, AR.V48, AR.V49, AR.V50, AR.V51, AR.V52} \\
		\hline
		{SR1.3} & {AR.M1, AR.C2, AR.V51, AR.V52} \\
		\hline
		{SR1.4} & {AR.V49} \\
		\hline
		{SR1.5} & {AR.M3, AR.C15, AR.L2, AR.V39, AR.V40, AR.V41, AR.V42, AR.V43, AR.V44, AR.V45, AR.V46} \\
		\hline
		{SR1.6} & {AR.C15, AR.V39, AR.V40, AR.V42, AR.V43, AR.V44, AR.V45, AR.V46} \\
		\hline
		{SR1.7} & {AR.M10, AR.C13, AR.V30, AR.V31, AR.V32, AR.V33, AR.P6} \\
		\hline
		{SR1.8} & {AR.C13, AR.V30, AR.V31, AR.V32, AR.V33, AR.P6} \\
		\hline
		{SR1.9} & {AR.M4, AR.M5, AR.M6, AR.C7, AR.C8, AR.C9, AR.V12, AR.V13, AR.V14, AR.V15, AR.P2} \\
		\hline
		{SR1.10} & {AR.M7, AR.M8, AR.C10, AR.C11, AR.V21, AR.V22, AR.V23, AR.V24, AR.V25, AR.V26, AR.V27, AR.V28, AR.P4, AR.P5} \\
		\hline
		{SR1.11} & {AR.M3, AR.C6, AR.V10, AR.V11, AR.V20} \\
		\hline
		{SR1.12} & {AR.V29} \\
		\hline
		{SR1.13} & {AR.C5, AR.V6, AR.V7, AR.V8, AR.V9, AR.P1} \\
		\hline
		{SR1.14} & {AR.C14, AR.V34, AR.V35, AR.V36, AR.V37, AR.V38, AR.P7} \\
		\hline
		{SR1.15} & {AR.C14, AR.V36} \\
		\hline
		{SR1.16} & {AR.C4, AR.V3} \\
		\hline
		{SR1.17} & {AR.C4, AR.V4} \\
		\hline
		{SR1.18} & {AR.C4, AR.V5} \\
		\hline
		{SR1.19} & {AR.L1, AR.V2} \\
		\hline
		{SR1.20} & {AR.V25, AR.V46} \\
		\hline
		{SR1.21} & {AR.V40, AR.V47} \\
		\hline
		{SR1.22} & {AR.M4, AR.M5, AR.V16, AR.V17, AR.V18, AR.V19, AR.P3} \\
		\hline
		{SR1.23} & {AR.M5, AR.V16, AR.V18} \\
		\hline
		{SR1.24} & {AR.M5, AR.V41, AR.V48} \\
		\hline
		{SR1.25} & {AR.M6, AR.C15, AR.L2, AR.V10, AR.V11, AR.V20, AR.V39, AR.V40, AR.V41, AR.V46} \\
		\hline
		{SR1.26} & {AR.V42, AR.V50} \\
		\hline
		{SR1.27} & {AR.M11, AR.C14} \\
		\hline
		{SR1.28} & {AR.L1, AR.L2, AR.L3} \\
		\hline
		{SR1.29} & {AR.X4} \\
		\hline
		{SR1.30} & {AR.M9, AR.C12} \\
		\hline
		{SR1.31} & {Tidsmätning loggas automatiskt av Rails.} \\
		\hline
	\end{tabular} \end{table} \FloatBarrier


	\subsection{Icke-funktionella krav}


	\begin {table} [ht] \begin{tabular} {  p{5cm} p{2cm} p{2cm} p{2cm} p{2cm} p{2cm} p{2cm} p{2cm} p{2cm} p{2cm} p{2cm} }
		\hline
		{Kravkälla} & {Krav} \\
		\hline
		{SR9.1} & {AR.M10} \\
		\hline
		{SR9.2} & {AR.M10, AR.M13, AR.L3, AR.V1, AR.X3} \\
		\hline
	\end{tabular} \end{table} \FloatBarrier


\clearpage
	\appendix

\end{document}

