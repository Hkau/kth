%%!TEX encoding = UTF-8 Unicode 
% Att vi vill ha ett dokument som ser ut som en teknisk artikel, på a4-papper, tvåsidigt med 10 punkters font.
\documentclass[a4paper,10pt,twoside]{article}
% sidmarginaler
\usepackage[inner=3cm,top=3cm,outer=2cm,bottom=3cm]{geometry}
% svenska avstavningsregler
\usepackage[swedish]{babel}
\usepackage[T1]{fontenc}
% teckenencoding
\usepackage[utf8x]{inputenc}

% mattesymboler
\usepackage{amssymb}

% fina kodlistings
\usepackage{fancyvrb}
\usepackage{listings}

% lite inställningar till listings-paketet, bland annat så att den bryter för långa rader
\lstset{
	% vilket språk vi använder i våra kodlistings, så att listings-paketet vet hur den ska highligta saker
	language=Java, 
	% huruvida vi ska ha syntax highlighting
	fancyvrb=true, 
	% hur stora tabstopp vi ska ha
	tabsize=4, 
	% huruvida vi ska tillåta andra tecken än a-z
	extendedchars=\true
	% hur breda listings vi vill ha (skriv exempelvis linewidth=0.5\textwidth för att få listings som bara tar upp halva bredden av sidan)
	linewidth=\textwidth, 
	% huruvida vi ska visa mellanslag
	showstringspaces=false, 
	% huruvida vi ska bryta rader som är för långa
	breaklines=true, 
	% huruvida den ska få bryta rader mitt i ord eller inte (true här betyder att den bara bryter mellan ord)
	breakatwhitespace=true, 
	% indentera radbrytningar automatiskt
	breakautoindent=true,
	% lägg in radnummer på vänster sida
	numbers=left, 
	% hur stora radnumren ska vara
	numberstyle=\tiny, 
	% hur långt det ska vara mellan radnumren och koden
	numbersep=8pt
}
% stoppa in fina hyperlänkar (som man kan klicka på) i tableofcontents
\usepackage{hyperref}
\hypersetup{pdfpagemode=FullScreen,  colorlinks=true, linkcolor=blue}

% Ett litet paket för fin pseudokod
\usepackage{algorithmic}
\usepackage{algorithm}

% stoppa in ditt namn här nedanför
\def\myName{Peter Boström}

% kurskod här (skriver in indans kod som default)
\def\courseCode{DD1340}
% kursnamn här (återigen inda som default)
\def\courseName{Introduktion till Datalogi}

% stoppa in numret på inlämningen här nedanför
\def\assignmentNumber{7}

% stoppa in datum för när du skriver den här inlämningen här
\def\writtenDate{2009-03-19}

% fina sidheaders/footers
\usepackage{fancyhdr}
% inställningar till fancyhdr
\pagestyle{fancy}\headheight 13pt
\fancyfoot{}
% sidhuvud, vänster sida, fyll i ditt namn här
\lhead{\courseCode - \myName}
% sidhuvud, höger sida, fyll i vilken uppgift detta gäller
\rhead{Inlämningsuppgift \assignmentNumber}
% sidnumrering på vänster sida för jämna sidnummer, höger sida för ojämna sidnummer
\fancyfoot[LE,RO]{\thepage}


\title{Hemuppgift \assignmentNumber -  \courseName}
\date{\writtenDate}
\author{\myName}
\begin{document}
\maketitle % skapa titelsida
	\thispagestyle{empty}\cfoot{}
\clearpage % ny sida
\thispagestyle{empty}\cfoot{}
\tableofcontents % innehållsförteckning
\clearpage
\setcounter{page}{1} % börja räkna från sidnummer ett på den här sidan.

\section{Grafuppgifter} % ny sektion
\subsection{Rita grafer}

Graferna finns bifogade längst bak.

Det går att konstruera en graf med 5 hörn, 4 kanter och 3 komponenter, förutsatt att kanten får gå från och till samma nod.

\subsection{Graf med 8 hörn}

Grafen finns bifogad längst bak.

\subsubsection{Djupet först-sökning}

DFS med start i hörnet 0:

0, 1, 4, 3, 5

Besöker 0.

Besöker 0:s första, 1

Besöker 1:s första obesökta: 4

Besöker 4:s första obesökta: 3

Besöker 3:s första obesökta: 5

Inga noder kvar med någon annan obesökt att besöka.

\subsubsection{Bredden först-sökning}

BFS med start i hörnet 0:

0, 1, 3, 4, 5

Besöker 0.

Besöker 0:s närmaste: 1, 3

Besöker 1:s närmaste som inte är besökta: 4

Besöker 3:s närmaste som inte är besökta: 5

\subsection{Representera med närhetsmatris eller närhetslistor}

\subsubsection{1000 hörn, 2000 kanter, sparsam med minne}

Närhetslistor, för att vara sparsam med minnet använder man närhetslistor så länge antalet kanter inte närmar sig antalet hörn i kvadrat. I detta fall hade det motsvarat 1 000 000 kanter. 2000 är väldigt mycket mindre, därför är listor att föredra.

\subsubsection{1000 hörn, 50 000 kanter, sparsam med minne}

Närhetslistor, då 50 000 fortfarande är väldigt mycket mindre än 1 000 000.

\subsubsection{Avgöra om 2 hörn är grannar på konstant tid}

Närhetsmatris, A[x][y] är alltid konstant.

\subsection {Varför är DFS $\Theta(n^2)$ för en sammanhängande graf med n hörn som representeras med en närhetsmatris}

DFS:en kommer att besöka ALLA noder. Från varje nod kommer den att leta igenom sina rader i tur och ordning efter en nod som inte är besökt, besöka den och dess undernoder, och så vidare, och sedan kommer den leta efter sin nästa obesökta nod. Dvs kommer den att gå igenom alla sina noder, för att se om de är grannar eller inte, och sedan kolla om de är besökta. I närhetsmatrisen kommer man behöva även gå igenom de noder som inte är grannar, eftersom man inte sparar de som är grannar, utan bara relationen mellan två stycken noder. Varje nod kommer på detta sätt behöva gå igenom alla sina noder, och finna att de antingen är grannar eller inte. n noder som i sin tur måste kolla mot n noder ger $Theta(n^2)$.

\section {Programmering med grafer}

Programmet slumpar fram grafer med samma antal kanter, slumpmässigt. Sedan djupet först-söker programmet tills alla hörn är besökta. Samtidigt samlar programmet upp information om antal kluster och hur många hörn som finns i största klustret.

\subsection {Resultat}

\begin{tabular}{llll} % fyra stycken kolumner med vänsterjustering
Storlek	&	Tid för DFS	&	Antal kluster	&	Största kluster	\\
\hline % horisontell linje
1000	&	20 ms	&	159 st	&	815 st	\\
5000	&	36 ms	&	786 st	&	4034 st \\
10000	&	68 ms	&	1584 st &	7968 st	\\
20000	&	197 ms	&	3242 st	&	16027 st	\\
40000	&	231 ms	&	6538 st &	31744 st	\\
80000	&	425 ms	&	12998 st	&	63673 st	\\
100 000	&	1137 ms	&	16405 st	&	79474 st	\\
\end{tabular}

\subsection{Val av datastruktur}

I det här fallet är valet av en närhetslista klart överlägset närhetsmatrisen. Eftersom antalet hörn är lika med antalet kanter så kommer varje nod vara kopplad till storleksordningen två kanter. Att gå igenom två kanter för att hitta vilka som gränsar till hörnet är klart överlägset att gå igenom lika många som antalet hörn, eftersom detta är storleksorningen upp till flera tusen gånger fler. Därför föreslås att problemet blir närmare linjärt än det är kvadratiskt, vilket (mer eller mindre) reflekteras i graferna.

\clearpage

% i appendix infogar du alla eventuella .java-filer du ska skicka med, och lite annat.
\appendix

\section{Källkod}
% här är källkod för klassen MyClass.java

GraphBuilder.java
ListGraph.java
ListGraphTest.java
ListGraphVertexIterator.java


\subsection{GraphBuilder.java}
\lstinputlisting{src/kth/csc/inda/GraphBuilder.java}

\subsection{ListGraph.java}
\lstinputlisting{src/kth/csc/inda/ListGraph.java}

\subsection{ListGraphTest.java}
\lstinputlisting{test/kth/csc/inda/ListGraphTest.java}

\subsection{ListGraphVertexIterator.java}
\lstinputlisting{src/kth/csc/inda/ListGraphVertexIterator.java}

% slut på dokumentet
\end{document}
