\documentclass{article}
\usepackage[utf8]{inputenc}
\begin{document}

\section*{Sort}

{{\def\dash{\raise2.1pt\hbox{\rule{5pt}{0.3pt}}\hspace{1pt}}\begin{tabbing}
{\it{/$\ast$$\ast$}}\\
{\it{\hspace{6pt}$\ast$\hspace{6pt}A\hspace{6pt}collection\hspace{6pt}of\hspace{6pt}sorting\hspace{6pt}algorithms\hspace{6pt}for\hspace{6pt}arrays\hspace{6pt}of\hspace{6pt}integers.}}\\
{\it{\hspace{6pt}$\ast$\hspace{6pt}}}\\
{\it{\hspace{6pt}$\ast$\hspace{6pt}@author\hspace{6pt}Stefan\hspace{6pt}Nilsson,\hspace{6pt}Peter\hspace{6pt}Boström}}\\
{\it{\hspace{6pt}$\ast$\hspace{6pt}@version\hspace{6pt}2008\dash{}11\dash{}17}}\\
{\it{\hspace{6pt}$\ast$/}}\\
{\textbf{public}}\hspace{6pt}{\textbf{class}}\hspace{6pt}Sort\\
\{\hspace{6pt}\\
\hspace{24pt}{\it{/$\ast$$\ast$}}\\
{\it{\hspace{30pt}$\ast$\hspace{6pt}Default\hspace{6pt}constructor.}}\\
{\it{\hspace{30pt}$\ast$/}}\\
\hspace{24pt}{\textbf{public}}\hspace{6pt}Sort()\\
\hspace{24pt}\{\\
\hspace{24pt}\}\\
\\
\hspace{24pt}{\it{/$\ast$$\ast$}}\\
{\it{\hspace{30pt}$\ast$\hspace{6pt}Sort\hspace{6pt}the\hspace{6pt}elements\hspace{6pt}in\hspace{6pt}ascending\hspace{6pt}order\hspace{6pt}using\hspace{6pt}selection\hspace{6pt}sort.}}\\
{\it{\hspace{30pt}$\ast$\hspace{6pt}This\hspace{6pt}algorithm\hspace{6pt}has\hspace{6pt}time\hspace{6pt}complexity\hspace{6pt}Theta(n$\ast$n),\hspace{6pt}where\hspace{6pt}n\hspace{6pt}is}}\\
{\it{\hspace{30pt}$\ast$\hspace{6pt}the\hspace{6pt}length\hspace{6pt}of\hspace{6pt}the\hspace{6pt}array.}}\\
{\it{\hspace{30pt}$\ast$\hspace{6pt}}}\\
{\it{\hspace{30pt}$\ast$\hspace{6pt}@param\hspace{12pt}v\hspace{18pt}An\hspace{6pt}array\hspace{6pt}of\hspace{6pt}integers.}}\\
{\it{\hspace{30pt}$\ast$/}}\\
\hspace{24pt}{\textbf{public}}\hspace{6pt}{\textbf{static}}\hspace{6pt}{\textbf{void}}\hspace{6pt}selectionSort({\textbf{int}}{[}{]}\hspace{6pt}v)\\
\hspace{24pt}\{\\
\hspace{54pt}{\textbf{int}}\hspace{6pt}n\hspace{6pt}=\hspace{6pt}v.length;\\
\hspace{54pt}{\textbf{for}}\hspace{6pt}({\textbf{int}}\hspace{6pt}i\hspace{6pt}=\hspace{6pt}0;\hspace{6pt}i\hspace{6pt}$<$\hspace{6pt}n\hspace{6pt}\dash{}\hspace{6pt}1;\hspace{6pt}i++)\hspace{6pt}\{\\
\hspace{78pt}{\it{//\hspace{6pt}find\hspace{6pt}index\hspace{6pt}m\hspace{6pt}of\hspace{6pt}min\hspace{6pt}element\hspace{6pt}in\hspace{6pt}v{[}i..n\dash{}1{]}\hspace{12pt}}}\\
\hspace{78pt}{\textbf{int}}\hspace{6pt}m\hspace{6pt}=\hspace{6pt}i;\\
\hspace{78pt}{\textbf{for}}\hspace{6pt}({\textbf{int}}\hspace{6pt}j\hspace{6pt}=\hspace{6pt}i\hspace{6pt}+\hspace{6pt}1;\hspace{6pt}j\hspace{6pt}$<$\hspace{6pt}n;\hspace{6pt}j++)\hspace{6pt}\{\\
\hspace{102pt}{\textbf{if}}\hspace{6pt}(v{[}j{]}\hspace{6pt}$<$\hspace{6pt}v{[}m{]})\\
\hspace{126pt}m\hspace{6pt}=\hspace{6pt}j;\\
\hspace{78pt}\}\\
\hspace{78pt}{\it{//\hspace{6pt}swap\hspace{6pt}v{[}i{]}\hspace{6pt}and\hspace{6pt}v{[}m{]}}}\\
\hspace{78pt}{\textbf{int}}\hspace{6pt}temp\hspace{6pt}=\hspace{6pt}v{[}i{]};\\
\hspace{78pt}v{[}i{]}\hspace{6pt}=\hspace{6pt}v{[}m{]};\\
\hspace{78pt}v{[}m{]}\hspace{6pt}=\hspace{6pt}temp;\\
\hspace{54pt}\}\\
\hspace{24pt}\}\\
\\
\hspace{24pt}{\it{/$\ast$$\ast$}}\\
{\it{\hspace{30pt}$\ast$\hspace{6pt}Sort\hspace{6pt}the\hspace{6pt}elements\hspace{6pt}in\hspace{6pt}ascending\hspace{6pt}order\hspace{6pt}using\hspace{6pt}insertion\hspace{6pt}sort.}}\\
{\it{\hspace{30pt}$\ast$\hspace{6pt}This\hspace{6pt}algorithm\hspace{6pt}has\hspace{6pt}time\hspace{6pt}complexity\hspace{6pt}Theta(n$\ast$n),\hspace{6pt}where\hspace{6pt}n\hspace{6pt}is}}\\
{\it{\hspace{30pt}$\ast$\hspace{6pt}the\hspace{6pt}length\hspace{6pt}of\hspace{6pt}the\hspace{6pt}array.}}\\
{\it{\hspace{30pt}$\ast$\hspace{6pt}}}\\
{\it{\hspace{30pt}$\ast$\hspace{6pt}@param\hspace{12pt}v\hspace{18pt}An\hspace{6pt}array\hspace{6pt}of\hspace{6pt}integers\hspace{6pt}to\hspace{6pt}be\hspace{6pt}sorted.}}\\
{\it{\hspace{30pt}$\ast$/}}\\
\hspace{24pt}{\textbf{public}}\hspace{6pt}{\textbf{static}}\hspace{6pt}{\textbf{void}}\hspace{6pt}insertionSort({\textbf{int}}{[}{]}\hspace{6pt}v)\\
\hspace{24pt}\{\\
\hspace{48pt}{\textbf{int}}\hspace{6pt}n\hspace{6pt}=\hspace{6pt}v.length;\\
\\
\hspace{48pt}{\textbf{for}}({\textbf{int}}\hspace{6pt}i\hspace{6pt}=\hspace{6pt}1;\hspace{6pt}i\hspace{6pt}$<$\hspace{6pt}n;\hspace{6pt}++i)\\
\hspace{48pt}\{\\
\hspace{72pt}{\textbf{for}}({\textbf{int}}\hspace{6pt}j\hspace{6pt}=\hspace{6pt}0;\hspace{6pt}j\hspace{6pt}$<$\hspace{6pt}i;\hspace{6pt}++j)\\
\hspace{72pt}\{\\
\hspace{96pt}{\textbf{if}}(v{[}j{]}\hspace{6pt}$>$\hspace{6pt}v{[}i{]})\\
\hspace{96pt}\{\\
\hspace{120pt}{\textbf{int}}\hspace{6pt}temp\hspace{6pt}=\hspace{6pt}v{[}i{]};\\
\hspace{120pt}{\textbf{for}}({\textbf{int}}\hspace{6pt}k\hspace{6pt}=\hspace{6pt}i;\hspace{6pt}k\hspace{6pt}$>$\hspace{6pt}j;\hspace{6pt}k\dash{}\dash{})\\
\hspace{120pt}\{\\
\hspace{144pt}v{[}k{]}\hspace{6pt}=\hspace{6pt}v{[}k\dash{}1{]};\\
\hspace{120pt}\}\\
\hspace{120pt}v{[}j{]}\hspace{6pt}=\hspace{6pt}temp;\\
\hspace{96pt}\}\\
\hspace{72pt}\}\\
\hspace{48pt}\}\\
\hspace{24pt}\}\\
\}
\end{tabbing}}}


\section*{Uppgift 2.3}

	$log(n)$ växer långsammare än $n^{c}$ som växer långsammare än $c^{n}$, förutsatt att $c>1$.

	\[n^{2.5} = \Theta(n^{2.5})\]
	\[\sqrt{2n} = \Theta(n^{1/2})\]
	\[n + 10 = \Theta(n)\]
	\[10^{n} = \Theta(10^{n})\]
	\[100^{n} = \Theta(100^{n})\]
	\[n^{2}log(n) = \Theta(n^{2}log(n))\]

	Följer är funktionerna i sjunkande ordning:

	\[100^{n}\]
	\[10^{n}\]
	
	Överstående funktioner är exponentiella, vilka växer snabbare än polynom. Sedan följer polynom av sjunkande grad.

	\[n^{2.5}\]
	\[n^{2}log(n)\]

	$n^{2}log(n)$ liknar polynomet $n^{2}$, men växer långsammare än $n^{2.5}$ eftersom $log(n)$ som logaritm växer långsammare än $n^{0.5}$ som polynom.

	\[n+10\]
	\[n^{1/2}\]

\section*{Komplexitet}

	\[\frac{n(n-1)}{2} = \frac{n^{2}-n}{2}\]

	O($n^{3}$) sant, $n^{3}$ är av högre grad än $n^{2}$ som är högsta graden av polynom.

	O($n^{2}$) sant, $n^{2}$ är högsta termen i funktionen.

	$\Theta(n^{3})$ är falskt, $\Theta(n^{2})$ hade stämt då $n^{2}$ är termen av högst grad.

	$\Omega(n)$ sant eftersom det ligger under funktionens komplexitet $\Theta(n^{2})$.\newline

\section*{Bestäm $f(n)$ där $O(f(n)) \neq n$ och $\Omega(f(n)) \neq n$}

	\[f(n) = |\frac{1}{sin(x)}|\]

	\[f(n) = |tan(x)|\]

	\[f(n) = ((-1)^{n}+1) * n^{2}\]


\section*{Uppgift 2.6}

	Detta gäller eftersom förutom de två for-looparna, vilka ger $O(n^{3})$ så är algoritmen för att beräkna summan av a[i] till a[j] av komplexiteten $O(n)$ vilket medför att algoritmen är:

	\[O(n^{3})\]

	Algoritmen är oberoende av indata, och kommer alltid att köras likadant för samma n. Algoritmen är kubisk, därför gäller även att:

	\[O(f(n)) = \Omega(f(n)) = \Theta(f(n)) = \Theta(n^{3})\]

	\subsection *{Mindre komplex algoritm}

	\indent \indent for i=1, 2, ... n-1 \newline

	\indent \indent b[i, i+1] = a[i] + a[i+1];\newline

	\indent \indent for j=i+2, i+3 ... n \newline

	\indent \indent \indent b[i, j] = b[i, j-1] + a[j]; \newline

	\indent \indent Endfor

	Endfor \newline

	Först fylls första relevanta värdet i för varje kolumn. Sedan fylls kolumnen i rad efter rad, där a[j] läggs till på värdet som var i tidigare raden.

	Eftersom summan har blivit utbytt inuti den andra for-loopen mot en konstant operation, bestämmer nästlade for-looparna ordningen, vilken blir $O(n^{2})$.

	Följer gör att hela algoritmen är av ordningen:

	\[O(n^{2})\]

\end{document}
