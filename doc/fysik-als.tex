\def\theauthor{Martin Frost} % TODO: stoppa in ditt namn här
\def\homeworknumber{17} % TODO: stoppa in vilken hemläxa det är här
\def\coursename{Godtycklighetslära} % TODO: stoppan in kursnamn här
\def\course{DD4711} % TODO: stoppa in kurskod här
\def\thedate{2009-11-06}

\documentclass[a4paper,10pt]{article}
\usepackage[inner=3cm,top=3cm,outer=2cm,bottom=3cm]{geometry}
\usepackage[swedish]{babel}
%\usepackage[T1]{fontenc}
\usepackage[utf8]{inputenc}

\title{Hemläxa \homeworknumber\ - \course\ \coursename}
\date{\thedate}
\author{\theauthor}

\begin{document}
\maketitle % skapa titelsida
	\thispagestyle{empty}
\newpage % ny sida
\thispagestyle{empty}
\tableofcontents % innehållsförteckning
\newpage

% TODO: stoppa in faktiskt innehåll här.

\section{Kanalanpassning efter våglängd} % (fold)

\begin{center}

\begin{tabular}[c]{|l|l|}
	\hline
	Våglängd & Kanal \\
	\hline
	404,7nm & 626 \\
	435,8nm & 686 \\
	546,1nm & 896 \\
	578,0nm & 960 \\
	\hline
	Andra & diffraktionsordningen \\
	\hline
	809,4nm & 1931 \\
	871,6nm & 1515 \\
	1092nm & 1931 \\
	\hline

\end {tabular}

\end {center}

\vspace{10pt}

Vi fick två toppar för 578 nanometer som sammanföll till mindre toppar i kanal 975 och 960. Vi valde topp 960 efter rådfrågning då den gav mindre residual vid anpassning (mindre fel).

Det största felet vi fick var 0.8 (nm).

\subsection

\subsection{En godtycklig undersektion} % (fold)

544,2
550,0
556.5 *
563,1
569.9 *
576,7
583,5 *
590,4
597,7
605.3 *
612,9
620.3 *
627,7
635.6 *
643,4
652,4
661,3
669,8
678.0 *
686,1
695.6 *
705,1
714,6
725,2
733,6
743,1
753,1
764,2
773,2
783,8
Lorem ipsum dolor sit amet, consectetur adipisicing elit, sed do eiusmod tempor incididunt ut labore et dolore magna aliqua. Ut enim ad minim veniam, quis nostrud exercitation ullamco laboris nisi ut aliquip ex ea commodo consequat. Duis aute irure dolor in reprehenderit in voluptate velit esse cillum dolore eu fugiat nulla pariatur. Excepteur sint occaecat cupidatat non proident, sunt in culpa qui officia deserunt mollit anim id est laborum.



\end{document}
