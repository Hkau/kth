\documentclass[a4paper,10pt,twoside]{article}
% includes (fold)
\usepackage[inner=3cm,top=3cm,outer=2cm,bottom=3cm]{geometry}
\usepackage[swedish]{babel}
\usepackage[T1]{fontenc}
\usepackage{moreverb}
\usepackage{amssymb}
\usepackage{fancyhdr}
\usepackage{color}
\definecolor{dark-blue}{rgb}{0, 0, 0.6}
\usepackage{hyperref}
\hypersetup{
  colorlinks=true, 
  linkcolor=dark-blue,
  urlcolor=dark-blue
}
% includes (end)

% defines (fold)
\def\names{André, Martin, Peter, Rasmus} % ...och namn på alla andra som skrivit...
\def\theauthor{Peter Boström\\890224-0814\\pbos@kth.se\and Martin Frost\\840326-7118\\blame@kth.se\and André Gräsman\\890430-3214\\grasman@kth.se \and Rasmus Göransson\\850908-8517\\rasmusgo@kth.se}
\def\homeworknumber{Chefsintervju} % fyll i vilket kursmoment det handlar om här
\def\coursename{Organisation och Kunskapsintensivt Arbete}
\def\course{ME1010}

\title{\homeworknumber\\ \course\ - \coursename}
\date{2010-04-28} % TODO
\author{\theauthor}
% defines (end)

\begin{document}
% fancyheaders (fold)
\fancypagestyle{plain}{
  \headheight 13pt
  \fancyfoot{}
  \lhead{\course\ -- \homeworknumber}
  \rhead{\names}
  \fancyfoot[LE,RO]{\thepage}
}
\pagestyle{plain}
\fancypagestyle{empty}{
  \fancyhead{}
  \fancyfoot{}
}
% fancyheaders (end) 

\maketitle % skapa titelsida
\thispagestyle{empty}
\newpage

\thispagestyle{empty}
\tableofcontents % innehållsförteckning
\newpage
\setcounter{page}{1}

\section{Uppgift} % (fold)
\label{sec:uppgift}

Vi är en grupp på fyra teknologer som har fått uppgiften att intervjua en chef eller ledare på ett företag. Uppgiften är ett delmoment i kursen Organisation och kunskapsintensivt arbete. Kursen handlar om organisation, ledarskap och gruppsykologi. Vi har själva valt chef att ta kontakt med och bestämt inriktning på intervjun.

% section uppgift (end)

\section{Förberedelser och utförande} % (fold)
\label{sec:förberedelser_och_utförande}

Vi började med att ta kontakt med en chef som en i gruppen tidigare hade haft kontakt med angående sommarjobb. Först hade vi ett kort samtal för att se om möjligheten till intervju fanns. Vi skickade sedan ett utförligt mail med vidare information och utkast till intervjufrågor. Tillsammans bestämde vi en tid för intervju på plats på företaget. Därefter satte vi oss i gruppen och diskuterade vad vi ville få ut av intervjun. Främst var vi intresserade av hennes personliga reflektioner och upplevelser kring chefsskap, samt vad det innebär att vara en bra chef. Vi formulerade hur vi skulle ställa våra frågor för att få personliga reflektioner snarare än direkta svar på explicita frågor. Detta gjorde vi genom att formulera frågorna på ett öppet sätt och med alternativa formuleringar med olika perspektiv.

På intervjun presenterade vi oss och berättade om kursen, dess syfte och vår uppgift. Vi kom överrens om att inte använda namn på varken chef eller företag i rapporten. Under intervjun tog alla anteckningar och turades om att ställa frågor. Vi spelade även in under intervjun för att ha möjlighet att lyssna ifall vi missat något. Stämningen under intervjun blev snabbt avslappnad och ledde till en öppen stämning. Vi höll en passiv roll under intervjun så att vi fick långa och reflekterande svar. Vi fick alla med oss mycket material från intervjun.


% section förberedelser_och_utförande (end)

\section{Intervjun} % (fold)
\label{sec:intervjun}

Vi har intervjuat en kvinnlig chef på ett IT-konsultföretag i Stockholm. På IT-konsultföretaget blev hon för två år sedan chef tillfälligt under ett halvår. Nu vid årsskiftet (2010) började hon permanent som konsultgruppchef för 35 personer.

I grunden har hon en examen i farkostteknik från KTH. Efter examen började hon arbeta på Ericsson och blev snabbt tillfrågad om att bli projektledare. Inom loppet av fem år drev hon flera stora projekt. När hennes avdelning senare skulle flyttas till USA valde hon att stanna hemma. Under sin tid på Ericsson hade hon haft kontakt med det IT-konsultföretag som hon idag jobbar på. Eftersom de visste att hon tänkte sluta erbjöd de henne anställning. Hon började som konsult men blev sedan kundansvarig. Kundansvarig inom företaget innebär att sälja in projekt hos kunden, driva dem och hålla kundkontakten.

Vi upplever att kundansvarig syftar mer på en roll än en titel. Rollen innefattar både att vara projektledare, säljare samt att hålla fortlöpande kontakt med kunden. Vi tror att titeln kundansvarig har valts för att undvika maktladdningen som projektledare kan innebära. Istället för att vara ledare för en grupp är man kundansvarig inom gruppen.

Från KTH hade hon ingen utbildning i projektledning, men när hon arbetade som projektledare på Ericsson gick hon samtidigt kurser genom företaget. Ericsson var frikostiga med bra kurser och seminarier inom ledarskap och specifikt situationsanpassat ledarskap. Hon tyckte det var väldigt bra att arbeta och läsa kurser parallellt. Detta gav henne många verktyg att applicera i sitt arbete. Hon lärde sig även genom nätverk av kollegor och medarbetare i liknande roller.

\subsection{Chefsrollen} % (fold)
\label{sub:chefsrollen}
Hon ser sig själv som en struktuerad och socialt ansvarstagande person.Hon trodde inte att hon skulle bli chef, men hon har alltid irriterat sig på dåliga chefer. Däremot tycker hon att det kan vara svårt att leva upp till den bild som hon har av en bra chef, men inte i projektledarrollen, då man har tydliga mål som alla arbetar mot.

Hennes bild av en bra chef är en person som kan ta hänsyn till att det händer saker i folks privatliv. Man måste ta sig tid att lyssna och ha en förmåga att stötta sina medarbetare. Att verkligen lyssna är något som är svårt och som hon har fått arbeta på. Det är lätt att vara flera steg framåt i tankarna, men det är viktigt att kunna vara närvarande i stunden. Som chef handlar arbetet till stor del om kommunikation. Därför är det viktigt att uttrycka sig med “diplomatisk klokskap” och situationsanpassning. Man måste tänka på vad man säger och vilka konsekvenser ens uttalande kan innebära. Det är viktigt att gilla att arbeta med människor och att våga vara personlig och ge av sig själv. 

Som chef rekryterar hon alla konsulter till sin grupp. Att anställa rätt personligheter är viktigt för att hålla en bra stämning inom gruppen. Det är också viktigt att rekrytera personer med varierade kunskaper som kompletterar varandra. Hon säljer även in uppdrag till kunderna, därför behövs bra bedömningsförmåga för att bedöma uppgiftens svårighetsgrad och vad kunden egentligen är ute efter. Internt gäller det att tillsätta personal med kompetens som passar till uppgiften. En stor del av jobbet handlar om att placera rätt person på rätt plats. 

% subsection chefsrollen (end)

\subsection{Företagskultur och struktur} % (fold)
\label{sub:företagskultur_och_struktur}
Företaget styrs internt av mänskliga värderingar, som exempelvis att man ska ha roligt på jobbet. De har fasta löner för att inte vara internt tävlingsinriktade. Fokus ska ligga på uppgiften och samarbete istället för personlig vinning. Det är viktigt att kommunicera direkt person till person. Det är viktigt att alla bemöter varandra med ömsesidig respekt, oavsett position. Det finns inte så många vertikala positioner inom företaget och alla fyller en viktig funktion. Alla chefer på företaget är kundansvariga.

Vi tolkar det som att företagets struktur är platt och har stort fokus på individen och personliga relationer. Det här passar in i modellen “Human Relations School” som berättar om hur fokus ligger på att nyttja individens förmåga. Detta till skillnad från “The Classical School of Management” som fokuserar på att optimera varje moment. “The Classical School of Management" anser att det finns ett optimalt sätt att utföra en uppgift oavsett vem som utför uppgiften. “Human Relations School" leder till plattare hierarkier, då man tillåter individen större eget ansvar och inte kontrollerar arbetet i detaljnivå. Modellen bygger också på en mer mänsklig syn och uppmanar till sociala relationer mellan individer. Företagets värderingar som till stor del handlar om mänskliga värderingar passar därför väl in i denna modell.

Företagets värderingar hålls hårt på. Det är svårt att införa vissa typer av förändringar som t.ex. att förändra företagets kommunikationsflöde. Detta eftersom kommunikationsflödet finns beskrivet inom värderingarna som vill uppmuntra till direkt kommunikation. Införande av nya system kan även vara trögt, då man behöver lära upp personal. Förändringar man vill införa måste därför motiveras väl då det finns system som redan fungerar. Andra typer av förändringar som inte berör dessa värderingar är mycket lättare att införa, t.ex. marknadsanpassningar som ständigt görs.

En gång varannan vecka har hon möte med sin chef för att stämma av hur arbetet har fortskridit och vad som ska göras härnäst. På företaget har de ett start-på-veckan-möte där alla i ledningsroll rapporterar kort vad de gör och hur det går, men det är inte bara då man då har kontakt med sina chefer. Hon har även daglig kontakt med sin chef och sina kollegor i liknande roller. Ledningsgruppen har dessutom en wiki på prov för att dela information, istället för att skicka mail. Kontakt med kund, exempelvis i form av möten, har hon varje vecka. Var tredje vecka har hon ett möte tillsammans med sin konsultgrupp.

Ännu ett exempel som passar in i “Human Relations School” är hur företaget hanterar feedback. Personligen tycker vår chef att ett bra sätt att få feedback är att själv lämna feedback, eller att uttrycklingen be om det. De ger ofta varandra feedback naturligt efter möten med kunder. Detta passar i modellen eftersom feedback inte bara sker uppifrån och ned utan även i andra riktningar. Framför allt lämnas feedback i sidled mellan personer i liknande roller.

% subsection företagskultur_och_struktur (end)

\subsection{Balans mellan arbete och privatliv} % (fold)
\label{sub:balans_mellan_arbete_och_privatliv}
Hon är nöjd med hur hennes arbetssituation ser ut just nu, men tycker att det är hektiskt som chef. Det är svårt att styra arbetstiderna, men hon tycker att det var lättare när barnen var små och behövde hämtas på dagis. Då hade hon en tid att passa och kunde lämna arbetet vid dagisgrinden. Hon tycker att det är lättare om man håller på med aktiviteter, så att man har fasta tider och kan släppa arbetet lättare. Man får anstränga sig för att verkligen bara göra det man har framför sig; t.ex. att komma hem och klappa katten och då bara tänka på att klappa katten. Hon försöker bli färdig med det som ligger i huvudet när hon arbetar innan hon går hem på kvällen. När hon kommer hem så lägger hon undan mobilen och försöker hålla sig ifrån datorn. Hon har satt upp lediga stunder då man inte får arbeta. Fredagkvällar och lördagar är heliga och hon försöker att inte arbeta på söndagar.

Dock kan det bli lite väl intensivt som chef och det hade gärna kunnat få vara lite lugnare. Hon är mån om att saker blir bra, men genom prestation så skapar hon mer arbete åt sig själv. Däremot så är arbetet intressant då det ändå är varierat. Det framstod att hon var mer intresserad av de intressanta arbetsuppgifterna än att göra karriär. I framtiden skulle hon även kunna tänka sig att göra något helt annat någon gång. Vi tror att man behöver visioner för att vara lycklig, men måste inte nödvändigtvis få dem uppfyllda. Det kan vara väldigt skönt att ha en tanke bara för tankens skull. Hon tänker ibland att hon ska sova till klockan tio på morgonen, men gör det ändå inte.

% subsection balans_mellan_arbete_och_privatliv (end)
% section intervjun (end)

\section{Reflektioner} % (fold)
\label{sec:reflektion}

Innan vi gjorde intervjun hade vi en del förväntningar på vilka svar vi skulle få på våra frågor.

En av förväntningarna vi hade var att gränsen mellan arbete och privatliv skulle vara ganska flytande, och därmed svår att upprätthålla. Det är lätt att ta hem arbetsuppgifterna för att de inte behöver göras på plats. På den här punkten stämde vår föreställning. Hon tyckte att det var svårt att släppa jobbet när hon inte hade fasta tider att gå efter, som att hämta barnen på dagis.


En sak som vi inte hade tänkt på, och som kom ganska oväntat, var att de inte använder sig av provisionsbaserade löner. Vi hade någonstans väntat oss att en provisionsbaserad lön skulle fungerat som en morot till att göra ett bra jobb. Istället förklarade hon att de vill undvika intern konkurrens och uppmuntra till samarbete. I efterhand kan vi koppla detta till ett resonemang om att lönehöjningar bara är kortsiktigt motiverande. Detta resonemang togs upp på en föreläsning i kursen. Man vänjer sig fort med en högre inkomst och slutar se den som en belöning.

I frågan om intern utbildning hade vi en uppfattning om att det är ganska vanligt att företag låter sina anställda gå kurser för att fortbilda sig inom sitt område. Det är dessutom bra för företaget att utbilda sina anställda samtidigt som de arbetar, eftersom de anställda då lättare kan relatera sitt lärande till sina uppgifter på företaget. De lär sig kursen på ett sådant sätt som passar in på företaget. Vår chef berättade att hon på Ericsson fått mycket intern utbildning inom ledarskap, vilket stämde bra överrens med vår föreställning.
% section reflektion (end)

\end{document}