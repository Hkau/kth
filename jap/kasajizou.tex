\documentclass[a4paper] {article}
\usepackage[utf8]{inputenc}
\title {Japanska FK2}
\author {Peter Boström \\ pbos@kth.se}

\begin {document}
\maketitle

\section*{Kasajizou}

Det var en gång för länge sedan, då en gammal man och en gammal kvinna levde inuti berget och tillverkade hemmagjorde bambuhattar. Det var dagen innan nyårsafton. Ett nytt år höll på att börja. Men, eftersom de inte hade några pengar så hade de inte heller några riskakor. De sålde därför bambuhattar för att försöka köpa riskakor.

Gubben tog med sig bambuhattarna och begav sig ner till staden för att försöka sälja dem. Men ingen ville köpa hans hattar och gubben blev väldigt ledsen. Så han gick tillbaka längs den långa bergsvägen. Det snöade kraftigt på vägen hemåt.

"Aa! Det är herr Jizo!"

Mitt i snön stod det sex stycken Jizo-statyer. Gubben sa, "Herr Jizo, är det inte väldigt kallt?"

Jizo-statyn sa ingenting.

"Varsågod, här får du en bambuhatt att använda." Gubben tog en bambuhatt och täckte över toppen på statyns huvud.

"Ett, två, tre, fyra, fem."

Det var totalt fem stycken. Men en av Jizo-statyerna saknade bambuhatt. Gubben tog av sig sin egna hatt.

"Den här bambuhatten är gammal, men varsågod", sa han samtidigt som han satte på statyn hatten.

När han återvänt hem så berättade han om Jizo-statyerna för gumman. Då sa hon "Det var väl en fin gärning som du gjort."

Senare på kvällen hörde gubben nåns röst.

"Gamla man, gamla man." Gubben öppnade dörren och blev samtidigt väldigt förvånad.

Framför honom stod de sex Jizo-herrarna från tidigare. De hade med sig hur många riskakor för nyår som helst.

Följande morgon var nyårsdagen. Gubben och gumman åt väldigt många riskakor. De var väldigt lyckliga.

\end{document}

