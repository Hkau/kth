\documentclass[a4paper,10pt,twoside]{article}
\usepackage[inner=3cm,top=3cm,outer=2cm,bottom=3cm]{geometry}
\usepackage[swedish]{babel}
\usepackage[T1]{fontenc}
\usepackage[utf8]{inputenc}
\usepackage{moreverb}
\usepackage{amssymb}
\usepackage{fancyhdr}
\usepackage{fancyvrb}
\usepackage{listings}
\lstset{language=Java, fancyvrb=true, tabsize=4}
\usepackage{algorithmic}
\usepackage{algorithm}
\newcommand{\Ordo}[1]{$\mathcal{O}(#1)$}
\pagestyle{fancy}\headheight 13pt
\fancyfoot{}
\lhead{DD1340 - Peter Boström}
\rhead{Inlämningsuppgift 4}
\fancyfoot[LE,RO]{\thepage}
\title{Hemuppgift 4 - DD1340 Introduktion till Datalogi}
\date{2009-02-08}
\author{Peter Boström}
\begin{document}
\maketitle % skapa titelsida
	\thispagestyle{empty}\cfoot{}
\newpage % ny sida
\setcounter{page}{1} % börja räkna från sidnummer ett på den här sidan.

\section{Tidskomplexitet} % ny sektion

Uppgiften är att beräkna tidskomplexiteten för \emph{medelfallen} av insättning, borttagning och sökning efter element i diverse datastrukturer, inkl. hash table.

	\subsection{Osorterad vektor}

		\subsubsection*{Sökning}

		\Ordo{n} eftersom elementet kan ligga var som helst i vektorn måste alla element gås igenom. Medelfallet förväntas hittas i mitten. Dvs. måste man gå igenom $\frac{n}{2}$ element, vilket är \Ordo{n}.

		\subsubsection*{Insättning}

		\Ordo{1} eftersom det endast är att placera elementet i slutet av listan. Räknar inte med omallokering av listan, utan antar att det finns plats för elementet.

		\subsubsection*{Borttagning}

		\Ordo{n} eftersom hälften av listan måste kopieras ett steg till vänster då medelfallet är att elementet tas bort från mitten. \Ordo{\frac{n}{2}} = \Ordo{n}.

	\subsection{Sorterad vektor}

		\subsubsection*{Sökning}

		Binärsökning ger \Ordo{log(n)}.

		\subsubsection*{Insättning}

		\Ordo{n}. Att hitta var elementet ska placeras är \Ordo{log(n)}. Det ska genomsnittligt placeras i mitten av listan vilket kräver att hälften av alla element flyttas ett steg till höger. Att flytta hälften av alla element är \Ordo{n}.

		\subsubsection*{Borttagning}

		\Ordo{n} av samma anledning som insättning. Oavsett om sökning efter elementet som ska tas bort ingår i borttagningen eller inte är operationen \Ordo{n} eftersom \Ordo{log(n)} < \Ordo{n}.

	\subsection{Osorterad enkellänkad lista}

		\subsubsection*{Sökning}

		\Ordo{n}. Alla element måste gås igenom tills det hittas. Medelfallet hittas i mitten, vilket är \Ordo{n}.

		\subsubsection*{Insättning}

		\Ordo{1}. Att bryta en länk och stoppa in ett annat element är konstant. Eftersom listan är osorterad spelar det ingen roll var utan elementet läggs till längst fram eller i slutet.

		\subsubsection*{Borttagning}

		\Ordo{1}. Av samma anledning som insättning.

	\subsection{Sorterad enkellänkad lista}

		\subsubsection*{Sökning}

		Fortfarande \Ordo{n} eftersom man måste gå igenom hälften av alla element för att komma till mitten. Går inte att binärsöka.

		\subsubsection*{Insättning}

		\Ordo{n}. Hitta platsen där elementet ska sitta är \Ordo{n}. Att sedan sätta in elementet är \Ordo{1}. Tillsammans blir det \Ordo{n}.

		\subsubsection*{Borttagning}

		\Ordo{n} av samma anledning som insättning. Om det inte ingår att söka upp elementet eftersom det tidigare hittades med en iterering så är operationen konstant, \Ordo{1}

	\subsection{Hashtabell}

	Tidskomplexiteten för hashtabellen antar att antalet element är ungefär lika med hashtabellens storlek.

	Analysen antar även att hashfunktionen har uniform distribution.

		\subsubsection*{Sökning}

		\Ordo{1}. Vid medelfallet finns inga kollisioner vid hashen mod storlek. Detta gör att hashen hittas på konstant tid. I värre mindre ovanliga fall får man gå igenom två-tre element, men det är mindre vanligt vid uniform distribution. Medelfallet är konstant.

		\subsubsection*{Insättning}

		\Ordo{1}. Fortfarande inga kollisioner för medelfallet. Att lägga till elementet i en tom lista är konstant, annars är det litet nog.

		\subsubsection*{Borttagning}

		\Ordo{1}. Fortfarande inga kollisioner för medelfallet. Att ta bort elementet i en tom lista är konstant, annars är det litet nog.


\newpage
\section{Källkod}
	\subsection*{HashStringDictionary.java}
		\VerbatimInput{src/HashStringDictionary.java}
	\subsection*{HashStringDictionaryTest.java}
		\VerbatimInput{src/HashStringDictionaryTest.java}

\end{document}

