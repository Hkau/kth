\documentclass[a4paper, twoside, 11pt, titlepage]{article}

\usepackage{bds/bds}

\usepackage[utf8]{inputenc} % -- använd denna "när det funkar", dvs på skolans nya datorer + linux, ibland på windows
\usepackage[swedish,english]{babel}
\usepackage{placeins}

\project{Bokningssystem för Kårspexet}
\author{
	\small
	Arvidsson, Kalle -- kallear@kth.se\\
	Boström, Peter -- pbos@kth.se\\
	Eklund, Erik -- eekl@kth.se\\
	Gräsman, André -- grasman@kth.se\\
	Göransson, Rasmus -- rasmusgo@kth.se\\
	Hagsten, Per -- hagsten@kth.se\\
	Hallberg, Victor -- victorha@kth.se\\
	Modée, Anna Maria -- ammodee@kth.se\\
	Nyberg, Daniel -- dnyb@kth.se\\
	Stjernberg, Johan -- stjer@kth.se\\
	Tarandi, Andreas -- taran@kth.se
	}

\version{0.1}
\title{Architectural Design Document}

\begin{document}
\maketitle

\clearpage
\thispagestyle{empty}
\mbox{}
\newpage

\selectlanguage{english}
\begin{abstract}
	This document describes the architectural design of Nyx' booking system for Kårspexet. It aims to give Nyx' developers  a more defined picture of what needs to be delivered during the software development phase. 
Within, the reader will find a brief overview of the system's architecture, external interfaces, and what methods will be used while implementing the system. A full, highly detailed description is given for all the system's components, from its models, views, and controllers, to its layouts, partials and external components. The project's feasibility is also included, with a likely time plan for the system's development phase. Lastly, a matrix is given that lets the reader trace requirements back to the System Requirements Document.
\end{abstract}
\selectlanguage{swedish}

\newpage

\setcounter{page}{1}

\startfooter

\clearpage
\section*{Ändringslogg}


\begin {table} [ht] \begin{tabular} { p{2.6cm} p{12.5cm} }
	\hline
	\sffamily\textbf{Version} & \sffamily\textbf{Ändringar } \\
	\hline
	\sffamily\textbf{0.1} & Första sammanställd version av dokumentet.  \\
	\hline
\end{tabular} \end{table} \FloatBarrier


\clearpage
\section*{Dokumentversioner}


Dokumentet har genererats från följande deldokument.

\textbf{ADD/abstract} version: \emph{4}.

\textbf{ADD/Ändringslogg} version: \emph{3}.

\textbf{Gruppmedlemmar} version: \emph{3}.

\textbf{ADD/Introduktion} version: \emph{2}.

\textbf{ADD/Introduktion/Syfte} version: \emph{5}.

\textbf{ADD/Introduktion/Mjukvarans omfattning} version: \emph{3}.

\textbf{ADD/Introduktion/Definitioner akronymer och förkortningar} version: \emph{13}.

\textbf{ADD/Introduktion/Källor} version: \emph{5}.

\textbf{ADD/Introduktion/Dokumentöversikt} version: \emph{9}.

\textbf{ADD/Systemöverblick} version: \emph{8}.

\textbf{ADD/Systemkontext} version: \emph{9}.

\textbf{ADD/Systemdesign} version: \emph{2}.

\textbf{ADD/Systemdesign/Designmetod} version: \emph{19}.

\textbf{ADD/Systemdesign/Uppdelningsbeskrivning} version: \emph{11}.

\textbf{ADD/Komponentbeskrivning} version: \emph{19}.

\textbf{ADD/Komponentbeskrivning/Model} version: \emph{62}.

\textbf{ADD/Komponentbeskrivning/Controller} version: \emph{24}.

\textbf{ADD/Komponentbeskrivning/View} version: \emph{47}.

\textbf{ADD/Komponentbeskrivning/Övriga\_komponenter} version: \emph{15}.

\textbf{ADD/Genomförbarhet- och resursuppskattning} version: \emph{22}.

\textbf{ADD/Spårningsmatris mellan mjukvarukrav och strukturella krav} version: \emph{2}.

\textbf{ADD/appendix} version: \emph{1}.

\clearpage
\section*{Gruppmedlemmar}


Projektgruppen \textbf{Nyx} består av följande medlemmar.

\textbf{Kalle Arvidsson} -- 890601-2490, kallear@kth.se

\textbf{Peter Boström} -- 890224-0814, pbos@kth.se

\textbf{Erik Eklund} -- 880816-0454, eekl@kth.se 

\textbf{André Gräsman} -- 890430-3214, grasman@kth.se 

\textbf{Rasmus Göransson} -- 850908-8517, rasmusgo@kth.se 

\textbf{Per Hagsten} -- 870529-0115, hagsten@kth.se

\textbf{Victor Hallberg} -- 890121-0057, victorha@kth.se

\textbf{Anna Maria Modée} -- 871120-0363, ammodee@kth.se 

\textbf{Daniel Nyberg} -- 900104-4495, dnyb@kth.se 

\textbf{Johan Stjernberg} -- 890315-0533, stjer@kth.se

\textbf{Andreas Tarandi} -- 890416-0317, taran@kth.se

\clearpage \tableofcontents \clearpage

\clearpage
\section{Introduktion}



	\subsection{Syfte}


	Dokumentets syfte är att specificera detaljer kring produktens arkitektur. Den är speciellt skriven för att underlätta arbetet inom Nyx, men skall också kunna läsas av de som skall underhålla produkten åt kunden Kårspexet. Dokumentet redogör för vilka komponenter som finns, systemdesignen samt en uppskattning av Nyx resurser och genomförbarhet.

	\subsection{Mjukvarans omfattning}


	Produkten som Nyx utvecklar består av ett webbaserat biljettbokningssystem med ett enkelt användargränssnitt för besökare och administrationsverktyg för Kårspexets personal. Administrationsverktygen består av tre gränssnitt; ett för säljare, ett för ekonomiansvariga och ett för administratörer.

	\subsection{Definitioner akronymer och förkortningar}


	VAR SNÄLL OCH TA BORT STRÄCKET FRAMFÖR AKRONYMER OM DU ANVÄNDER DIG AV DEM.

	\textbf{Modell} (Railsmodel) \emph{En datastruktur i MVC arkitekturen som laddar, sparar och hanterar data genom att vanligtvis arbete mot en databas. Modellen kan innehålla grundläggande logik för att opererea på dadan.}

	\textbf{Vy} (Railsview) \emph{En komponent i MVC arkitekturen som renderar innehållet från en modell till ett interaktivbart användargränssnitt. Flera vyer är ofta kopplade till samma modell fast med olika syften.}

	\textbf{Kontroll} (Railscontroller) \emph{todo.}

	\textbf{Partial} (Railspartial) \emph{todo.}

	\textbf{Action} (Rails action) \emph{todo.}

	\textbf{Active record} \emph{Teknik för att kommunicera med databaser i objekt-orienterade språk. Objekten i databasen kopplas till objekt i programmet.}

	- \textbf{Algoritm} \emph{Inom matematik och datorvetenskap är detta en begränsad uppsättning tydliga instruktioner för att utföra en uppgift.}

	\textbf{Apache} \emph{Syftar i detta dokument på webbservern Apache HTTP Server.}

	\textbf{Apache HTTP Server} \emph{Världens mest använda webbserver. Är gratis att använda. [1.3.1]}

	- \textbf{Arbetsminne} \emph{Även kallat primärminne. En fysisk komponent i en dator. Används för att lagra program och data då programmet körs.}

	- \textbf{Bandbredd} \emph{I vardagligt tal en storhet för hur mycket information som kan överföras på en viss tid. Vanlig enhet är Mbit/sekund.}

	- \textbf{Bit} (Binary Digit) \emph{Den grundläggande enhet som datorer arbetar med. En bit kan anta ett utav två möjliga värden (ofta angivna som 0 eller 1).}

	- \textbf{Byte} \emph{En vanlig enhet för informationsmängd i datasammanhang. En byte är ett paket bestående av åtta bitar.}

	- \textbf{CentOS} \emph{Ett operativsystem baserat på Red Hat Enterprise Linux som är gratis att använda. [1.3.2]}

	\textbf{Databas} \emph{En databas är en samling information ordnad på ett sådant sätt att informationen i den effektivt går att hitta.}

	- \textbf{Firefox} (Mozilla Firefox) \emph{En webbläsare som går att köra på de populäraste operativsystemen.}

	- \textbf{Foreign key} \emph{Inom relationsdatabaser en begränsning som kräver att information på ett ställe finns definierat tidigare på ett annat ställe (kolumner i tabeller).}

	\textbf{Gem} \emph{Term för Rails-bibliotek som ger utökad funktionalitet.}

	- \textbf{GHz} \emph{Enhet för antalet miljarder svängningar per sekund. ``G'' är binärt prefix för $10^{9}$. ``Hz'' är förkortning för Hertz.}

	\textbf{Gränssnitt} \emph{Utformningen av kommunikationen mellan en mjukvarumodul och användare eller annan mjuk-/hårdvara.}

	\textbf{GUI} (Graphical User Interface) \emph{Se gränssnitt.}

	- \textbf{HTML} (Hyper Text Markup Language) \emph{Ett språk och webbstandard som används för att beskriva strukturering av text, bilder och annan media på en webbsida.}

	\textbf{HTTP} (HyperText Transfer Protocol) \emph{Ett standardiserat protokoll som definierar hur kommunikation över webben sker.}

	- \textbf{HTTPS} (HTTP Secure) \emph{En kombination av HTTP och SSL/TLS med syftet att förhindra avlyssning av HTTP-trafiken}

	- \textbf{Hårdvara} \emph{Även kallat Maskinvara. Ett samlingsnamn för en dators fysiska komponenter.}

	- \textbf{Internet Explorer} \emph{En webbläsare utvecklad av Microsoft för operativsystemet Windows.}

	- \textbf{InvalidAuthenticyToken} \emph{Ett fel som kan uppstå i rails om användaren backar på sidor innehållande formulär. Uppstår på grund av Rails skydd mot XSS.}

	- \textbf{KiB} (kibibyte) \emph{$2^{10}$ byte.}

	- \textbf{Linux} \emph{Unix-liknande operativsystem. Linux är fri mjukvara.}

	- \textbf{MiB} (mebibyte) \emph{$2^{20}$ byte.}

	- \textbf{Mib} (mebibit) \emph{$2^{20}$ bit.}

	- \textbf{Mjukvara} \emph{Även kallat programvara. En organiserad samling av data och maskininstruktioner.}

	\textbf{MVC} (Model-View-Controller) \emph{Se Model-View-Controller.}

	\textbf{Model-View-Controller} \emph{Ett koncept som bygger på att separera data (modeller), logik (kontroller) och användarinterface (vyer).}

	\textbf{MySQL} \emph{En typ av relationsdatabas baserad på SQL-standarden. Ett relationsdatabas hanteringssystem där flera användare kan arbeta med flera databaser.}

	- \textbf{Passenger} \emph{I Rails-sammanhang en modul som gör det möjligt att köra Ruby on Rails på webbservern Apache.}

	- \textbf{Processor} \emph{Den komponent i en dator som utför beräkningar efter instruktioner.}

	- \textbf{RDoc} (Ruby Doc) \emph{Verktyg för att generera dokumentation för Ruby-källkod i HTML-format.}

	\textbf{Ruby} \emph{Ett objektorienterat programmeringsspråk.}

	\textbf{Rails} (Ruby On Rails) \emph{Ett abstrakt mjukvarubibliotek med öppen källkod för utveckling av webbapplikationer.}

	- \textbf{Sjöslaget} \emph{Årligen återkommande studentfest på Finlandsfärja.}

	\textbf{SQL} (Structured Query Language) \emph{Ett språk designat för att interagera med databaser.}

	- \textbf{SQL-injection} \emph{En metod för att förändra eller komma åt data i en databas genom att ange strängar i användarinterfacet som förändrar betydelsen av en SQL-fråga.}

	- \textbf{SHA-1} \emph{Ett sätt att spara strängar så de inte står i klartext. Det går inte att återskapa strängen efter omkodning.}

	- \textbf{SSL/TLS} (Secure Socket Layer/Transport Layer Security) \emph{Ett kryptografiskt protokoll för att sätta upp säkra kommunikationskanaler över internet.}

	- \textbf{Testkod} \emph{Kod som används för att testa funktioner i programmet så att det returnerar förväntat svar för att försäkra sig om att mjukvaran fungerar på ett tillfredsställande sätt.}

	- \textbf{Tutorial} \emph{En metod för att överföra kunskap som ofta används vid inlärning.}

	- \textbf{URD} (User Requirements Document) \emph{Dokument inom PSS050 standarden där användarens krav specificeras.}

	- \textbf{Webbapplikation} \emph{Samlingsnamn för mjukvara som användare kommer åt via en webbläsare.}

	\textbf{Webbläsare} \emph{Ett program som hämtar, tolkar och återger webbsidor kodade exempelvis som HTML.}

	\textbf{Webbserver} \emph{Program som körs på en server och distribuerar webbsidor och/eller andra filer som en webbläsare begär via HTTP-protokollet.}

	- \textbf{Webbsida} \emph{En fil, innehållandes exempelvis HTML, avsedd att visas av en webbläsare.}

	- \textbf{XSS} (Cross site scripting) \emph{En teknik som utnyttjar svagheter i en webbsida genom att låta en auktoriserad användare accessa en länk som modifierar sidan på ett sätt som den auktoriserade användaren inte önskar [1.3.3].}

	\subsection{Källor}


	Referenser till de källor som använts i dokumentet är listade här under. En och samma källa kan refereras vid flera ställen i texten. En referens är på formatet [Sektion.Rubrik.Löpnummer]. Exempelvis är [4.2.1] den första (1) referensen för rubriken ``Uppdelningsbeskrivning'' (2) under sektion ``Systemdesign'' (4).

	\textbf{Apache HTTP Server}

	\url{http://httpd.apache.org/}

	Hänvisning till källan görs från referenserna: [1.3.1].

	\textbf{CentOS}

	\url{http://www.centos.org/}

	Hänvisning till källan görs från referenserna: [1.3.2].

	\textbf{Cross-site-scripting}

	\url{http://en.wikipedia.org/wiki/Cross-site_scripting}

	Hänvisning till källan görs från referenserna: [1.3.3].

	\textbf{Ruby on Rails Guides: Getting Started with Rails}

	\url{http://guides.rubyonrails.org/getting_started.html}

	Hänvisning till källan görs från referenserna: [4.2.1].

	\subsection{Dokumentöversikt}


	\emph{1.5. Overview of the document. Similar to SRD Section 1.5, but describes the ADD.}

	\emph{Again it need not be assumed that readership on the customer side exists. In practise, this}

	\emph{document may again be company confidential to the development team.}

	Detta dokument inleds med en kort introduktion. Efter denna följer sektion 2 som ger en överblick av hela systemet. Detta bör vara utgångspunkten för läsare som inte redan är bekanta med Nyx system.

	I sektion 3 beskrivs systemets gränssnitt mot omgivningen. Sektion 4 beskriver systemets interna design, som sedan gås igenom i detalj i sektion 5 där alla delkomponenter beskrivs.

	I sektion 6 görs en bedömning av projektets genomförbarhet utifrån uppskattningar av behov och tillgänglighet av resurser, främst arbetstid för t ex programmering och testning. Detta inkluderar en riskbedömning.

	Sektion 7 kopplar krav från SRD:n till ADD:n. För varje krav listas de komponenter vars direkta syfte är att uppfylla kravet.

\clearpage
\section{Systemöverblick}


\emph{2 System Overview. Summarises: (i) the system context (how it fits into an existing}

\emph{framework of other packages and systems), and (ii) the system design. More detailed}

\emph{descriptions of (i) and (ii) are given in Sections 3 and 4 below.}

Nyx biljettbokningssystem kommer att ersätta det biljettbokningssystemet som för närvarande används av Kårspexet. Det innebär att vårt system kommer användas på Kårspexets webbplats och då vi endast utvecklar biljettbokningssystemet kommer vårt system behöva passa in på den befintliga webbplatsen. Dock ska systemet inte interagera med det befintliga systemet, annat än med länkar och därmed behöver vi bara anpassa vårt system visuellt.

Externa system som kommer användas av vårt system är MySQL, för att få tillgång till en databas, samt Apache HTTP Server, för att hantera HTTP-kommunikationen med användarens webbläsare. 

Nyx valde att utveckla applikationen i dessa system eftersom vi hade mycket positiv erfarenhet av dessa inom gruppen. Dessutom så arbetar Ruby on Rails väldigt väl mot MySQL så vi får därmed mycket funktionalitet på köpet när vi använder dessa tillsammans. Anledningen till att Apache valdes var för att det är välanvänt, väldokumenterat samt att det är öppen källkod vilket underlättar det för oss.

\clearpage
\section{Systemkontext}


\emph{3 System Context. Gives a detailed description of the system context, with relevant}

\emph{diagrams. Defines the external interfaces of the product under development to these other}

\emph{systems.}

\emph{3.n External interface definition. Provides an interface definition to each separate}

\emph{external component type or physical component.}

Systemet är designat för att köras på en webbserver där alla systemets komponenter befinner sig internt  på servern, som används via ett webbläsarfönster. Apache anropas som i sin tur kör igång Ruby som i förväg har laddat in Rails biblioteket, applikationen och övriga komponenter. Där behandlas anropet och Rails kommer att kommunicera med MySQL och hämtar relaterad data. Rails förbereder sedan resultatet och skickar tillbaka detta till Apache som vidarebefordrar det till klienten som i sin tur presenterar innehållet i webbläsaren.

	\subsection{MySQL}


	Applikationen kommer använda sig av en MySQL databas för att hantera data för det olika användargrupperna. Databasen anropas och relevant data läses in och modifieras från Ruby applikationens olika vyer.

	\subsection{Apache HTTP Server}


	Servern kommer att köra Apache HTTP Server för att sköta kommunikationen mellan servern och applikationen. När en användare ansluter tar Apache emot anropet och skickar det vidare till applikationen. Apache kommer även ta emot resultat från applikationen och vidarebefordrar det till användaren.

\clearpage
\section{Systemdesign}


\emph{4. System Design. Provides an overview of the design techniques used, especially any in-}

\emph{house or non-standard methods, project specific methods, or non-standard interpretation}

\emph{of standard languages/methods such as UML.}

	\subsection{Designmetod}



		\subsubsection{Model-View-Controller}


		Nyx har valt att följa designprincipen Model-View-Controller för utvecklig av systemet. MVC metoden är en metod för att separera modeller, logik och det visuella interfacet i olika komponenter. Modellerna (\emph{Models}) hanterar och lagrar data som är relevant för systemet. De ser även till att ingen ogiltlig data sparas i databasen. Kontroller (\emph{Controllers}) sköter all logik och är ett mellanlager mellan modellerna och interfacet användaren ser. Vyerna (\emph{Views}) i sin tur renderar det interface som användaren ser och sköter postning tillbaka till kontrollerna.

		\subsubsection{Komponentbeskrivningar}


		Komponentbeskrivningarna under punkt fem (5) är upplagda enligt följande format.

		Alla komponenter identifieras av 5.Tn där T är någon av bokstäverna MVCLPX, som står för vilken typ komponenten är, och n är ett sekventiellt tal från ett (1) och uppåt.

		I vissa fall används noteringen \emph{@namn} för variabelnamn. Detta indikerar att variabeln är en instansvariabel.

			\paragraph{Komponentrubriker}\

			Alla komponenter har en tabell med följande innehåll.

			\begin {table} [ht] \begin{tabular} { p{2.6cm} p{12.5cm} }
				\hline
				Typ & Kontroll/Modell/Layout/Vy/Partiell vy/Övrig komponent  \\
				\hline
				Syfte & Specificerar vilket syfte komponenten fyller. Referenser till SR-krav.  \\
				\hline
				Funktion & Anger vilka olika funktioner komponenten bidrar med.  \\
				\hline
				Delkomponenter & Eventuellt andra komponenter som är en del av/ingår i denna.  \\
				\hline
				Beroenden & Krav för användandet av denna komponent.  \\
				\hline
				Gränssnitt & Publika metoder som går att anropa. Inkluderar metodnamn och kortare beskrivning av vad metoden gör. Om delar av gränssnittet kräver att användaren är inloggad/admin/accountant/sales ska detta specificeras här.  \\
				\hline
				Resurser & Vilka resurser (modeller/klasser/gems) som komponenten utnyttjar direkt (ej indirekt).  \\
				\hline
				Källor & Referenser till information om eventuellt använda gem och andra externa komponenter.  \\
				\hline
				Process & Lista möjliga arbetsflöden (övergångar).  \\
				\hline
				Data & Tillgängliga instansvariabler för/som används av komponenten (de som är relevanta från ett externt perspektiv).  \\
				\hline
			\end{tabular} \end{table} \FloatBarrier


			\paragraph{Komponenttyper}\


			\subparagraph{\emph{M - Model (\emph{Modell})}}\

				\emph{Beskriver en modell i MVC-modellen.}

				\textbf{Specifieringar och tillägg till komponentrubrikerna från 4.2.1}

				\begin {table} [ht] \begin{tabular} { p{2.6cm} p{12.5cm} }
					\hline
					Syfte & Beskriver hur ett objekt av typen fungerar samt vilka begränsningar som ställs.  \\
					\hline
					Delkomponenter & Vilka relationer till andra modeller som finns (motsvarande Rails has\_many- och belongs\_to-nyckelord).  \\
					\hline
					Beroenden & Krav för att objekt av denna modell ska vara giltiga (valideringskrav), inklusive relationer.  \\
					\hline
					Gränssnitt & Av Nyx definierade publika metoder som går att anropa och som antingen returnerar information (utöver standard-getters/setters) eller påverkar objektet, ej standardmetoder från ActiveRecord:Base. Inkluderar metodnamn, eventuella parametrar respektive returdata samt kortare beskrivning av vad metoden gör.  \\
					\hline
					Resurser & ActiveRecord::Base samt de eventuella resurser (modeller/gem) som modellen utnyttjar direkt (ej indirekt).  \\
					\hline
					Källor & Referenser till information om eventuellt använda gem, externa bibliotek, etc.  \\
					\hline
					Process & Eventuella valideringssteg eller andra processer för manipulation av objektet (inkluderar ej skapande/sparande/borttagning av objekt).  \\
					\hline
					Data & Lista med namn, typ och beskrivning av modellens samtliga attribut. Attributnamn följer lowercase\_with\_underscore-namngivning.  \\
					\hline
				\end{tabular} \end{table} \FloatBarrier


			\subparagraph{\emph{C - Controller (\emph{Kontroll})}}\

				\emph{Beskriver en kontroller i MVC-modellen.}

				\textbf{Specificeringar och tillägg till komponentrubrikerna från 4.2.1}

				\begin {table} [ht] \begin{tabular} { p{2.6cm} p{12.5cm} }
					\hline
					Delkomponenter & I de flesta fall ej applicerbart.  \\
					\hline
					Gränssnitt & Listar och beskriver alla actions. Inkluderar vilka anropsparametrar (@params[]@) som utnyttjas. Actions som kräver att användaren är inloggad eller innehar specifika roller (admin/accountant/sales) nämner detta här.  \\
					\hline
					Resurser & ApplicationController samt de eventuella resurser (modeller/gem) som kontrollern utnyttjar direkt (ej indirekt).  \\
					\hline
					Källor & Referenser till information om eventuellt använda gem, externa bibliotek, etc.  \\
					\hline
					Process & Listar möjliga arbetsflöden (övergångar) mellan actions (new > create, edit > update, index > show/destroy, etc.).  \\
					\hline
					Data & Instansvariabler som görs tillgängliga för vyerna och som är relevanta från ett externt perspektiv.  \\
					\hline
				\end{tabular} \end{table} \FloatBarrier


			\subparagraph{\emph{V - View (\emph{Vy})}}\

				\emph{Beskriver en vy i MVC-modellen.}

			\subparagraph{\emph{L - Layout}}\

				\emph{Beskriver en layout i MVC-modellen.}

			\subparagraph{\emph{P - Partial (\emph{Partiell vy})}}\

				\emph{Beskriver en partiell vy i MVC-modellen.}

			\subparagraph{\emph{X - Övriga/externa komponenter}}\

				\emph{Beskriver övriga och externa komponenter.}

				Komponenter som beskrivs här kan vara logiska komponenter (exempelvis inloggningssystemet) eller externa komponenter (exempelvis Ruby-gem).

	\subsection{Uppdelningsbeskrivning}


	\emph{4.2 Decomposition description. Gives the top level view of the systems design,}

	\emph{preferably with diagrams. Shows the major components which will be described in detail}

	\emph{in Section 5. Identifies control and data flow between components.}

	Systemet består av tre komponenttyper enligt MVC-modellen. [4.2.1]

	>> Modeller

	>> Vyer

	>> Kontroller

	\begin{figure}[ht] \centering \includegraphics[width=0.8\textwidth]{mvc.png} \end{figure} \FloatBarrier

	\^ misplaced image? how do I make this good? -- Peter

		\subsubsection{Modeller}


		Modeller representerar den data som applikationen använder sig av och består av och innehåller regler för hur denna data får manipuleras. I Nyx fall används modellerna främst för att bestämma hur tabeller i databasen ska interageras med. Det är här större delen av logiken i applikationen ligger.

		Bokningssystemet har följande typer av modeller:

		>> Administrativa modeller

		>> Bokningar

		>> Föreställningar

		\subsubsection{Vyer}


		Vyer representerar användargränssnittet till applikationen. Detta är olika webbsidor som visar representationer av data som finns i applikationen. De hanterar även förfrågningar som görs till systemet.

		Bokningssystemet har följande typer av vyer:

		>> Administration

		>> Ekonomi

		>> Kund

		>> Login

		>> Säljare

		\subsubsection{Kontroller}


		Kontroller kopplar samman modeller och vyer. I vårt fall är kontrollerna ansvariga för att hantera de inkommande förfrågningarna som kommer från användares webbläsare. De gör förfrågningar vidare till modellerna för data och skickar vidare till vyerna för att kunna presenteras för användaren.

		Bokningssystemet har följande typer av kontroller:

		>> Administration

		>> Ekonomiansvarig

		>> Försäljning

		>> Kund

\clearpage
\section{Komponentbeskrivning}


\emph{5. Component Description.}

\emph{Gives detailed component information according to a fixed template. Components may}

\emph{be top level components, identified in Section 4.2, or subcomponents of these. Preferably}

\emph{use a component identification scheme which is easy to read/follow/remember.}

\emph{5.n. [Component identifier] Fill in name here.}

\emph{5.n.1. Type. Could be a module, an input/output/temporary file, a program, a class, a}

\emph{script, a web page, etc.}

\emph{5.n.2. Purpose. Describe the purpose of the component, and relate this to a numbered}

\emph{software requirement in the SRD.}

\emph{5.n.3. Function. Describe the functionality of the component, including its interface}

\emph{properties (call and return types) and logical behaviour.}

\emph{5.n.4. Subordinates. List the immediate subcomponents of the component, using defined}

\emph{component identifiers.}

\emph{5.n.5. Dependencies. Describe the logical preconditions for using this component, e.g.}

\emph{files and/or objects that must exist.}

\emph{5.n.6. Interfaces. Define the control and data flow to and from the object. Gives a}

\emph{detailed picture of its context in the overall system architecture.}

\emph{5.n.7. Resources. List any resources required by the component, such as external}

\emph{components external subsystems, hardware, etc.}

\emph{5.n.8. References. Reference any external documents needed to understand the}

\emph{component.}

\emph{5.n.9. Processing. Describe the control and data flow betwen immediate subcomponents}

\emph{of this component. If the component has no immediate subcomponents (i.e. it is fully}

\emph{decomposed) then outline the method of processing used by the component to perform its}

\emph{task (e.g. with pseudo-code, state diagrams, etc).}

\emph{5.n.10. Data. Describe in detail (where possible) the local data values and data structures}

\emph{belonging to (local in scope) this component. Otherwise give an outline description.}

Komponenterna är nummrerade med en bokstav och en siffra: 5.Xn

Där X är en bokstav som representerar vilken typ av komponent det är och n är ett löpnummer som är unikt för den typen av komponenter. Bokstäverna som används är:

>> M - Model (Modell)

>> C - Controller (Kontroll)

>> V - View (Vy)

>> L - Layout

>> P - Partial (Partiell vy)

>> X - Övriga/externa komponenter

	\subsection{Modeller}


	\emph{TODO skriv om följande stycke så att varje mening inte börjar med 'modellerna'.}

	\emph{Se även designmetod, det mesta står även där, kanske ska detta stycke tas bort.}

	Modellerna är datastrukturer som i de flesta fall är kopplade till databasen. Isåfall ärver modellen @ActiveRecord::Base@, se ActiveRecord (5.X4). Modellerna implementerar också grundläggande logik relaterad till datan.

	Modellernas funktion är alltså att spara och hämta respektive data från databasen, samt hålla datan temporärt. Funktion har utelämnats i de flesta modeller.

	Modellerna är kopplade till relaterade modeller genom ActiveRecord, för varje modell beskrivs relationerna under delkomponenter.

	Modellernas gränssnitt och process beskrivs i ActiveRecord.

	Modellernas data inkluderar, förutom vad som specificeras för respektive modell, även det som specificeras i ActiveRecord.

	Alla modeller med databaskoppling har ActiveRecord som implicit resurs.

	SR1.29 (användande av databas) implementeras av alla modeller tillsammans.

	Skapande av databasschema och ingår i modellernas implementation.


		\subsubsection{Bokning}



			\paragraph{5.M1 Reservation}\

			\begin {table} [ht] \begin{tabular} {  p{3.5cm} p{9.6cm} }
				\hline
				Typ & Modell  \\
				\hline
				Syfte & Hålla information om en bokning (SR1.2, SR1.3).  \\
				\hline
				Funktion &   \\
				\hline
			\end{tabular} \end{table} \FloatBarrier
			\vspace{6mm}

			|Delkomponenter|Tillhör en Show.

			Har flera Placements.

			Har flera ReservationCounters.|

			\begin {table} [ht] \begin{tabular} {  p{3.5cm} p{9.6cm} }
				\hline
				Gränssnitt &   \\
				\hline
			\end{tabular} \end{table} \FloatBarrier
			\vspace{6mm}

			|Beroenden|Måste referera till existerande Show.

			Måste ha minst en ReservationCounter.

			Om inte @creator@ existerar måste @customer\_reservation@ vara giltig.|

			\begin {table} [ht] \begin{tabular} {  p{3.5cm} p{9.6cm} }
				\hline
				Resurser & grouped\_validations för att hantera valideringar  \\
				\hline
			\end{tabular} \end{table} \FloatBarrier
			\vspace{6mm}

			|Källor|grouped\_validations: \url{http://rubydoc.info/gems/grouped_validations/0.2.2/file/README.rdoc|}

			|Process|Valideringsgrupp @customer\_reservation@

			@email@ måste vara giltig.

			@adress@ måste existera.

			@post\_code@ måste vara giltig.

			@post\_town@ måste vara giltig.

			@paymentoption@ måste finnas.

			@deliverymethod@ måste finnas.|

			|Data|@name : string@

			@phone : string@

			@email : string@

			@adress : string@

			@post\_code : decimal@

			@post\_town : string@

			@comment : text@

			@paymentoption : {plusgiro, kontant, kort} (integer)@

			@deliverymethod : {brev, uthämtning} (integer)@

			@paid : boolean@

			@paid\_sum : decimal@

			@placed : boolean@

			@retrieved : boolean@

			@cost : decimal@

			@hash\_key : string@ - identifierar bokningen vid avbokning.

			@creator : integer@|

			\paragraph{5.M2 ReservationCounter}\

			\begin {table} [ht] \begin{tabular} {  p{3.5cm} p{9.6cm} }
				\hline
				Typ & Modell  \\
				\hline
				Syfte & Håller information om antal platser i en viss sektion, med ett visst pris för en bokning (SR1.2).  \\
				\hline
				Funktion &   \\
				\hline
			\end{tabular} \end{table} \FloatBarrier
			\vspace{6mm}

			|Delkomponenter|Tillhör en Reservation.

			Tillhör en Pricing.

			Tillhör en Section.|

			\begin {table} [ht] \begin{tabular} {  p{3.5cm} p{9.6cm} }
				\hline
				Gränssnitt &   \\
				\hline
				Beroenden & Måste ha existerande Reservation, Pricing och Section, samt Antal > 0.  \\
				\hline
				Resurser &   \\
				\hline
				Källor &   \\
				\hline
				Process &   \\
				\hline
				Data & @antal: decimal@  \\
				\hline
			\end{tabular} \end{table} \FloatBarrier


			\paragraph{5.M3 Placement}\

			\begin {table} [ht] \begin{tabular} {  p{3.5cm} p{9.6cm} }
				\hline
				Typ & Modell  \\
				\hline
				Syfte & En boknings placeringar (sittplatser) (SR1.5, SR1.11).  \\
				\hline
				Funktion &   \\
				\hline
			\end{tabular} \end{table} \FloatBarrier
			\vspace{6mm}

			|Delkomponenter|Tillhör en Reservation.

			Tillhör en Seat.

			Tillhör en Show.

			Tillhör en User.|

			\begin {table} [ht] \begin{tabular} {  p{3.5cm} p{9.6cm} }
				\hline
				Gränssnitt &   \\
				\hline
				Beroenden & Måste ha existerande Reservation, Show och Seat  \\
				\hline
				Resurser &   \\
				\hline
				Källor &   \\
				\hline
				Process &   \\
				\hline
				Data &   \\
				\hline
			\end{tabular} \end{table} \FloatBarrier


		\subsubsection{Föreställningar}



			\paragraph{5.M4 Theater}\

			\begin {table} [ht] \begin{tabular} {  p{3.5cm} p{9.6cm} }
				\hline
				Typ & Modell  \\
				\hline
				Syfte & Modell för teater (SR1.9, SR1.22).  \\
				\hline
				Funktion &   \\
				\hline
			\end{tabular} \end{table} \FloatBarrier
			\vspace{6mm}

			|Delkomponenter|Har flera Sections.

			Har flera Seats, genom Sections.

			Har flera Batches.

			Har flera Shows, genom Batches.|

			\begin {table} [ht] \begin{tabular} {  p{3.5cm} p{9.6cm} }
				\hline
				Gränssnitt &   \\
				\hline
				Beroenden & Måste ha internalname, publicname och picture.  \\
				\hline
				Resurser &   \\
				\hline
				Källor &   \\
				\hline
				Process &   \\
				\hline
			\end{tabular} \end{table} \FloatBarrier
			\vspace{6mm}

			|Data|@internal\_name : string@

			@public\_name : string@

			@description : string@

			@comment : string@

			@picture : string@ - bild eller referens till bild (?) för att visa sektioner och sittplatser.|

			\paragraph{5.M5 Section}\

			\begin {table} [ht] \begin{tabular} {  p{3.5cm} p{9.6cm} }
				\hline
				Typ & Modell  \\
				\hline
				Syfte & Modell för sektion (SR1.9, SR1.22--SR1.24).  \\
				\hline
				Funktion &   \\
				\hline
			\end{tabular} \end{table} \FloatBarrier
			\vspace{6mm}

			|Delkomponenter|Tillhör en Theater.

			Har flera Seats.|

			\begin {table} [ht] \begin{tabular} {  p{3.5cm} p{9.6cm} }
				\hline
				Gränssnitt &   \\
				\hline
				Beroenden & Måste tillhöra en existerande Theater  \\
				\hline
				Resurser &   \\
				\hline
				Källor &   \\
				\hline
				Process &   \\
				\hline
				Data &   \\
				\hline
			\end{tabular} \end{table} \FloatBarrier


			\paragraph{5.M6 Seat}\

			\begin {table} [ht] \begin{tabular} {  p{3.5cm} p{9.6cm} }
				\hline
				Typ & Modell  \\
				\hline
				Syfte & Modell för sittplats i en teaters sektion (SR1.9, SR1.25).  \\
				\hline
				Funktion &   \\
				\hline
			\end{tabular} \end{table} \FloatBarrier
			\vspace{6mm}

			|Delkomponenter|Tillhör en Section.

			Har flera Placements.

			Har flera Placement\_locks.|

			\begin {table} [ht] \begin{tabular} {  p{3.5cm} p{9.6cm} }
				\hline
				Gränssnitt &   \\
				\hline
				Beroenden & Måste referera existerande Section  \\
				\hline
				Resurser &   \\
				\hline
				Källor &   \\
				\hline
				Process &   \\
				\hline
			\end{tabular} \end{table} \FloatBarrier
			\vspace{6mm}

			|Data|@number : integer@ - stolsnummer

			@row : integer@ (x,y)-koordinat i teaterns bild eller dylikt.|

			\paragraph{5.M7 Batch}\

			\begin {table} [ht] \begin{tabular} {  p{3.5cm} p{9.6cm} }
				\hline
				Typ & Modell  \\
				\hline
				Syfte & Håller information om en omgång (SR1.10).  \\
				\hline
				Funktion &   \\
				\hline
			\end{tabular} \end{table} \FloatBarrier
			\vspace{6mm}

			|Delkomponenter|Tillhör en Theater.

			Har flera Shows.|

			\begin {table} [ht] \begin{tabular} {  p{3.5cm} p{9.6cm} }
				\hline
				Gränssnitt &   \\
				\hline
			\end{tabular} \end{table} \FloatBarrier
			\vspace{6mm}

			|Beroenden|Måste referera existerande Theater

			description får inte vara tom.

			Synlighet och startdatum måste finnas.|

			\begin {table} [ht] \begin{tabular} {  p{3.5cm} p{9.6cm} }
				\hline
				Resurser &   \\
				\hline
				Källor &   \\
				\hline
				Process &   \\
				\hline
			\end{tabular} \end{table} \FloatBarrier
			\vspace{6mm}

			|Data|@description : text@

			@visible : boolean@ - Synlighet, huruvida kunden kan se och boka biljetter till omgången.

			@start\_date@|

			\paragraph{5.M8 Show}\

			\begin {table} [ht] \begin{tabular} {  p{3.5cm} p{9.6cm} }
				\hline
				Typ & Modell  \\
				\hline
				Syfte & Data för en enskild föreställning (SR1.10).  \\
				\hline
				Funktion &   \\
				\hline
			\end{tabular} \end{table} \FloatBarrier
			\vspace{6mm}

			|Delkomponenter|Tillhör en Batch.

			Tillhör en Theater, genom Batch.|

			\begin {table} [ht] \begin{tabular} {  p{3.5cm} p{9.6cm} }
				\hline
				Gränssnitt &   \\
				\hline
				Beroenden & Måste referera existerande Batch.  \\
				\hline
				Resurser &   \\
				\hline
				Källor &   \\
				\hline
				Process &   \\
				\hline
				Data & @datetime : datetime@  \\
				\hline
			\end{tabular} \end{table} \FloatBarrier


			\paragraph{5.M9 Pricing}\

			\begin {table} [ht] \begin{tabular} {  p{3.5cm} p{9.6cm} }
				\hline
				Typ & Modell  \\
				\hline
				Syfte & I varje Batch har varje Section i teatern normalpris och studentpris. Detta sköts av Pricing-modellen (SR?).  \\
				\hline
				Funktion &   \\
				\hline
			\end{tabular} \end{table} \FloatBarrier
			\vspace{6mm}

			|Delkomponenter|Tillhör en Batch.

			Tillhör en Section.|

			\begin {table} [ht] \begin{tabular} {  p{3.5cm} p{9.6cm} }
				\hline
				Gränssnitt &   \\
				\hline
			\end{tabular} \end{table} \FloatBarrier
			\vspace{6mm}

			|Beroenden|Referenser till Batch och Section måste vara giltiga.

			@normal\_price@ och @student\_price@ måste vara icke-negativa heltal.|

			\begin {table} [ht] \begin{tabular} {  p{3.5cm} p{9.6cm} }
				\hline
				Resurser &   \\
				\hline
				Källor &   \\
				\hline
				Process &   \\
				\hline
			\end{tabular} \end{table} \FloatBarrier
			\vspace{6mm}

			|Data|@normal\_price@

			@student\_price@|

		\subsubsection{Administrativa och övriga modeller}



			\paragraph{5.M10 User}\

			\begin {table} [ht] \begin{tabular} {  p{3.5cm} p{9.6cm} }
				\hline
				Typ & Modell  \\
				\hline
				Syfte & För att hålla koll på de olika roller användare kan logga in som (SR1.7, SR9.1, SR9.2).  \\
				\hline
				Funktion &   \\
				\hline
				Delkomponenter & Tillhör en Ability.  \\
				\hline
				Gränssnitt &   \\
				\hline
			\end{tabular} \end{table} \FloatBarrier
			\vspace{6mm}

			|Beroenden|@username@ måste vara unikt.

			@password@ måste finnas.|

			\begin {table} [ht] \begin{tabular} {  p{3.5cm} p{9.6cm} }
				\hline
				Resurser &   \\
				\hline
				Källor &   \\
				\hline
				Process &   \\
				\hline
			\end{tabular} \end{table} \FloatBarrier
			\vspace{6mm}

			|Data|@username : string@

			@password : string@

			@valid\_to : date@|

			\paragraph{5.M11 MailTemplate}\

			\begin {table} [ht] \begin{tabular} {  p{3.5cm} p{9.6cm} }
				\hline
				Typ & Modell  \\
				\hline
				Syfte & För att hantera och mailmallar för färdigformatterade mail (SR1.27).  \\
				\hline
				Funktion &   \\
				\hline
				Delkomponenter &   \\
				\hline
				Gränssnitt &   \\
				\hline
				Beroenden & @content@ måste finnas.  \\
				\hline
				Resurser & ActiveRecord::Base  \\
				\hline
				Källor &   \\
				\hline
				Process &   \\
				\hline
			\end{tabular} \end{table} \FloatBarrier
			\vspace{6mm}

			|Data|@description :string@

			@title :string@

			@content :string@|

			\paragraph{5.M12 UserSession}\

			\begin {table} [ht] \begin{tabular} {  p{3.5cm} p{9.6cm} }
				\hline
				Typ & Modell  \\
				\hline
				Syfte & (Se 5.K2 AuthLogic, samt dess dokumentation) (SR1.1)  \\
				\hline
				Funktion & Håller koll på de sessioner som finns inloggade  \\
				\hline
				Delkomponenter & Har en User  \\
				\hline
			\end{tabular} \end{table} \FloatBarrier
			\vspace{6mm}

			|Gränssnitt|@find@ Letar upp efterfrågad session om den finns.

			@user@ |

			\begin {table} [ht] \begin{tabular} {  p{3.5cm} p{9.6cm} }
				\hline
				Beroenden &   \\
				\hline
				Resurser & AuthLogic::Base::Session  \\
				\hline
				Källor & AuthLogic: https://github.com/binarylogic/authlogic  \\
				\hline
				Process &   \\
				\hline
				Data &   \\
				\hline
			\end{tabular} \end{table} \FloatBarrier


			\paragraph{5.M13 Ability}\

			\begin {table} [ht] \begin{tabular} {  p{3.5cm} p{9.6cm} }
				\hline
				Typ & Modell  \\
				\hline
				Syfte & (Se 5.K3 CanCan, samt dokumentationen) (SR1.1, SR9.2)  \\
				\hline
				Funktion & Håller koll på vilka rättigheter varje användare har.  \\
				\hline
				Delkomponenter & Har flera Users \emph{(osäker på denna)}  \\
				\hline
				Gränssnitt & @can@ Returnerar om användaren har rättighet att göra den efterfrågade handlingen.  \\
				\hline
				Beroenden &   \\
				\hline
				Resurser & ActiveRecord::Base, CanCan, User  \\
				\hline
				Källor & CanCan: https://github.com/ryanb/cancan  \\
				\hline
				Process &   \\
				\hline
				Data &   \\
				\hline
			\end{tabular} \end{table} \FloatBarrier


	\subsection{Kontroller}



			\paragraph{5.C1 ApplicationController}\

			\begin {table} [ht] \begin{tabular} {  p{3.5cm} p{9.6cm} }
				\hline
				Typ & Kontroller  \\
				\hline
				Syfte & Rails-applikationens baskontroller som bland annat ska hantera autentisering (SR1.1).  \\
				\hline
				Funktion & Alla kontrollers ärver funktionerna som denna klass definierar, vilket inkluderar autentisering.  \\
				\hline
				Delkomponenter & Inga  \\
				\hline
				Beroenden & Inga  \\
				\hline
			\end{tabular} \end{table} \FloatBarrier
			\vspace{6mm}

			|Gränssnitt|ApplicationController tillhandahåller inga actions utan enbart filter och hjälpmetoder åt övriga kontrollers. De hjälpmetoder som definieras är:

			@current\_user@ - Returnerar en User-instans för den aktiva användaren (försöker logga in via Authlogic om nödvändigt).

			@current\_user\_session@ - Returnerar en UserSession-instans för den aktiva inloggningen (försöker logga in via Authlogic om nödvändigt).

			@require\_user@ - Hjälpmetod som kastar en exception om den anropas och användaren inte är autentiserad.|

			\begin {table} [ht] \begin{tabular} {  p{3.5cm} p{9.6cm} }
				\hline
				Resurser & AuthLogic och CanCan.  \\
				\hline
				Källor & Inga  \\
				\hline
				Process & Filter definerade med @before\_filter@ resp. @after\_filter@ körs före resp. efter den anropade kontrollerns action.  \\
				\hline
				Data & Inga  \\
				\hline
			\end{tabular} \end{table} \FloatBarrier


		\subsubsection{Kund}



			\paragraph{5.C2 BookingController}\

			\begin {table} [ht] \begin{tabular} {  p{3.5cm} p{9.6cm} }
				\hline
				Typ & Kontroller  \\
				\hline
				Syfte & Hanterar bokningsproceduren för kunden (SR1.2, SR1.3).  \\
				\hline
				Funktion & Skapar en ny bokning och sparar den eller avbokar en existerande bokning. Bokningsprocessen delas upp i flera privata metoder som anropas beroende på vilket steg i bokningen man befinner sig i.  \\
				\hline
				Delkomponenter & Inga  \\
				\hline
				Beroenden & Inga  \\
				\hline
			\end{tabular} \end{table} \FloatBarrier
			\vspace{6mm}

			|Gränssnitt|@new@ - påbörjar en bokning

			@create@ - sparar data i användarsessionen allt eftersom och vid sista steget lagrar en bokning i databasen

			@cancel@ - presenterar en bokning från en hash\_key med möjlighet till avbokning via ett formulär

			@destroy@ - tar bort en bokning.|

			\begin {table} [ht] \begin{tabular} {  p{3.5cm} p{9.6cm} }
				\hline
				Resurser & ApplicationController, Reservation, ReservationCounter, Batch och Show.  \\
				\hline
				Källor & Inga  \\
				\hline
			\end{tabular} \end{table} \FloatBarrier
			\vspace{6mm}

			|Process|new > (create)+ > create

			cancel > destroy|

			|Data|@@step@ - steg i bokning

			@@steps@ - array med vilka steg som finns

			@@reservation@ - bokningsdata|

		\subsubsection{Inloggning}



			\paragraph{5.C3 SessionController}\

			\begin {table} [ht] \begin{tabular} {  p{3.5cm} p{9.6cm} }
				\hline
				Typ & Kontroller  \\
				\hline
				Syfte & Hantera inloggning och sessioner (SR1.1).  \\
				\hline
				Funktion & Skapar ny session vid inloggning, tar bort session vid utloggning.  \\
				\hline
				Delkomponenter & Inga  \\
				\hline
				Beroenden & Inga  \\
				\hline
			\end{tabular} \end{table} \FloatBarrier
			\vspace{6mm}

			|Gränssnitt|@new@ - presenterar inloggningsformuläret

			@create@ - tar emot data från formuläret och skapar en ny UserSession

			@destroy@ - loggar ut|

			\begin {table} [ht] \begin{tabular} {  p{3.5cm} p{9.6cm} }
				\hline
				Resurser & ApplicationController och UserSession.  \\
				\hline
				Källor & Inga  \\
				\hline
			\end{tabular} \end{table} \FloatBarrier
			\vspace{6mm}

			|Process|new > create

			destroy > new|

			\begin {table} [ht] \begin{tabular} {  p{3.5cm} p{9.6cm} }
				\hline
				Data & Inga  \\
				\hline
			\end{tabular} \end{table} \FloatBarrier


		\subsubsection{Admin}



			\paragraph{5.C4 AdminIndexController}\

			\begin {table} [ht] \begin{tabular} {  p{3.5cm} p{9.6cm} }
				\hline
				Typ & Kontroller  \\
				\hline
				Syfte & Generera data för översikt och statistik (SR1.16, SR1.17, SR1.18).  \\
				\hline
				Funktion & Presenterar en översikt samt statistiksidorna för administratören och ekonomichefen.  \\
				\hline
				Delkomponenter & Inga  \\
				\hline
				Beroenden & Autentiserad som administratör eller ekonomichef.  \\
				\hline
			\end{tabular} \end{table} \FloatBarrier
			\vspace{6mm}

			|Gränssnitt|@index@ - översikt

			@statistics@ - statistik|

			\begin {table} [ht] \begin{tabular} {  p{3.5cm} p{9.6cm} }
				\hline
				Resurser & ApplicationController, Reservation, Batch och Show.  \\
				\hline
				Källor & Inga  \\
				\hline
				Process & Inga  \\
				\hline
			\end{tabular} \end{table} \FloatBarrier
			\vspace{6mm}

			|Data|@@unplaced@ - antal betalade bokningar att placera

			@@shows@ - tabell med information om aktuella föreställningar (antal bokade platser mm)

			@@stats@ - tabell över föreställningar, sektionsgrupper(prisklass), med antal ordinarie, studenter, gratis, platser kvar|

			\paragraph{5.C5 ReservationsController}\

			\begin {table} [ht] \begin{tabular} {  p{3.5cm} p{9.6cm} }
				\hline
				Typ & Kontroller  \\
				\hline
				Syfte & Administrering av bokningar (SR1.13).  \\
				\hline
				Funktion & Redigering av existerande bokningar samt skapande och borttagning. Ekonomichef kan bara läsa information och registrera betalningar.  \\
				\hline
				Delkomponenter & Inga  \\
				\hline
				Beroenden & Autentiserad som administratör eller ekonomichef.  \\
				\hline
			\end{tabular} \end{table} \FloatBarrier
			\vspace{6mm}

			|Gränssnitt|Ärver resursactions från InheritedResources med undantag för @show@.

			@index@ - lista bokningar med ev. filter

			@new@, @edit@ - formulär för att redigera alla attributer hos en bokning bortsett från stolsplacering|

			\begin {table} [ht] \begin{tabular} {  p{3.5cm} p{9.6cm} }
				\hline
				Resurser & ApplicationController, InheritedResources och Reservation.  \\
				\hline
				Källor & Inga  \\
				\hline
				Process & Standardprocesser för resurser.  \\
				\hline
			\end{tabular} \end{table} \FloatBarrier
			\vspace{6mm}

			|Data|Se InheritedResources.

			@@filters@ - array med möjliga filter samt indikationer för vilka som är aktiva (@index@)|

			\paragraph{5.C6 PlacementsController}\

			\begin {table} [ht] \begin{tabular} {  p{3.5cm} p{9.6cm} }
				\hline
				Typ & Kontroller  \\
				\hline
				Syfte & Tilldelning av stolar till webbokningar (SR1.11).  \\
				\hline
				Funktion & Hanterar stolsplaceringar för en given bokning med stöd för AJAX-anrop.  \\
				\hline
				Delkomponenter & Inga  \\
				\hline
				Beroenden & Autentiserad som administratör.  \\
				\hline
			\end{tabular} \end{table} \FloatBarrier
			\vspace{6mm}

			|Gränssnitt|@new@, @edit@ - presenterar aktuella placeringar och status för alla stolar med möjlighet att placera

			@test\_multiple@ [AJAX] - validerar placeringar, skapar temporära låsningar och returnerar dess IDn

			@update\_multiple@ - skapar och uppdaterar nya/existerande placeringar

			@destroy@ - tar bort existerande placeringar för given bokning|

			\begin {table} [ht] \begin{tabular} {  p{3.5cm} p{9.6cm} }
				\hline
				Resurser & ApplicationController och Placement.  \\
				\hline
				Källor & Inga  \\
				\hline
			\end{tabular} \end{table} \FloatBarrier
			\vspace{6mm}

			|process|new > [test\_multiple]+ > update\_multiple

			edit > [test\_multiple]+ > update\_multiple

			edit > destroy|

			|Data|@@reservation@ - den aktulla bokningsinstansen

			@@placements@ - placeringar

			@@theater@ - aktuell teater (bild mm)

			@@show@ - information om föreställningen (datum mm)

			@@seats@ - stolar med positioner och placeringsstatus (upptagen, tillfälligt låst, ledig, vald för denna bokning)|

			\paragraph{5.C7 TheatersController}\

			\begin {table} [ht] \begin{tabular} {  p{3.5cm} p{9.6cm} }
				\hline
				Typ & Kontroller  \\
				\hline
				Syfte & Hantera teatrar (SR1.9).  \\
				\hline
				Funktion & Hantera teatrar, inklusive möjligheten att kopiera existerande. Plus: administrera sektioner och platser.  \\
				\hline
				Delkomponenter & SectionsController och SeatsController.  \\
				\hline
				Beroenden & Autentiserad som administratör.  \\
				\hline
			\end{tabular} \end{table} \FloatBarrier
			\vspace{6mm}

			|Gränssnitt|Ärver resursactions från InheritedResources.

			@new@ - formulär för att skapa ny teater, inkluderar val att kopiera en existerande teater.|

			\begin {table} [ht] \begin{tabular} {  p{3.5cm} p{9.6cm} }
				\hline
				Resurser & ApplicationController, InheritedResources och Theater.  \\
				\hline
				Källor & Inga  \\
				\hline
				Process & Standardprocesser för resurser.  \\
				\hline
			\end{tabular} \end{table} \FloatBarrier
			\vspace{6mm}

			|Data|Se InheritedResources.

			@@theaters@ - existerande teatrars attributer (@new@)|

			\paragraph{5.C8 SectionsController}\

			\begin {table} [ht] \begin{tabular} {  p{3.5cm} p{9.6cm} }
				\hline
				Typ & Kontroller  \\
				\hline
				Syfte & Hantera sektioner (SR1.9).  \\
				\hline
				Funktion & Visa och redigera sektionsindelningar av en existerande teater.  \\
				\hline
				Delkomponenter & Inga  \\
				\hline
				Beroenden & Autentiserad som administratör, nästlad via TheaterController.  \\
				\hline
			\end{tabular} \end{table} \FloatBarrier
			\vspace{6mm}

			|Gränssnitt|Ärver resursactions från InheritedResources med undantag för @show@.

			@index@ - visar sektioner för aktuell teater.|

			\begin {table} [ht] \begin{tabular} {  p{3.5cm} p{9.6cm} }
				\hline
				Resurser & ApplicationController, InheritedResources, Theater och Section.  \\
				\hline
				Källor & Inga  \\
				\hline
				Process & Standardprocesser för resurser.  \\
				\hline
			\end{tabular} \end{table} \FloatBarrier
			\vspace{6mm}

			|Data|Se InheritedResources.

			@@theater@ - aktuell teater|

			\paragraph{5.C9 SeatsController}\

			\begin {table} [ht] \begin{tabular} {  p{3.5cm} p{9.6cm} }
				\hline
				Typ & Kontroller  \\
				\hline
				Syfte & Hantera stolarnas positioner och sektionstillhörighet (SR1.9).  \\
				\hline
				Funktion & Visa och redigera en given teaters stolar och deras positioner på salongsskissen. AJAX används för att hantera stolar utan att ladda om sidan.  \\
				\hline
				Delkomponenter & Inga  \\
				\hline
				Beroenden & Autentiserad som administratör, nästlad via TheaterController.  \\
				\hline
			\end{tabular} \end{table} \FloatBarrier
			\vspace{6mm}

			|Gränssnitt|@index@ - presenterar gränssnittet för att lägga till och redigera teaterns stolar

			@create@ [AJAX] - sparar en ny stol

			@update@ [AJAX] - uppdaterar en existerande stol

			@destroy@ [AJAX] - tar bort en existerande stol|

			\begin {table} [ht] \begin{tabular} {  p{3.5cm} p{9.6cm} }
				\hline
				Resurser & ApplicationController, Theater och Seat.  \\
				\hline
				Källor & Inga  \\
				\hline
				Process & index > [create/update]+  \\
				\hline
			\end{tabular} \end{table} \FloatBarrier
			\vspace{6mm}

			|Data|@@seats@ - existerande stolar

			@@theater@ - aktuell teater|

			\paragraph{5.C10 BatchesController}\

			\begin {table} [ht] \begin{tabular} {  p{3.5cm} p{9.6cm} }
				\hline
				Typ & Kontroller  \\
				\hline
				Syfte & Hantera omgångar (SR1.10).  \\
				\hline
				Funktion & Visa och redigera omgångar.  \\
				\hline
				Delkomponenter & ShowsController och PricingsController.  \\
				\hline
				Beroenden & Autentiserad som administratör.  \\
				\hline
				Gränssnitt & Ärver resursactions från InheritedResources med undantag för @show@.  \\
				\hline
				Resurser & ApplicationController, InheritedResources och Batch.  \\
				\hline
				Källor & Inga  \\
				\hline
				Process & Standardprocesser för resurser.  \\
				\hline
				Data & Se InheritedResources.  \\
				\hline
			\end{tabular} \end{table} \FloatBarrier


			\paragraph{5.C11 ShowsController}\

			\begin {table} [ht] \begin{tabular} {  p{3.5cm} p{9.6cm} }
				\hline
				Typ & Kontroller  \\
				\hline
				Syfte & Hantera föreställningar (SR1.10).  \\
				\hline
				Funktion & Hanterar föreställningar som hör till en given omgång.  \\
				\hline
				Delkomponenter & Inga  \\
				\hline
				Beroenden & Autentiserad som administratör, nästlad via BatchesController.  \\
				\hline
			\end{tabular} \end{table} \FloatBarrier
			\vspace{6mm}

			|Gränssnitt|Ärver resursactions från InheritedResources med undantag för @show@.

			@index@ - listar alla föreställningar för given omgång|

			\begin {table} [ht] \begin{tabular} {  p{3.5cm} p{9.6cm} }
				\hline
				Resurser & ApplicationController, InheritedResources och Show.  \\
				\hline
				Källor & Inga  \\
				\hline
				Process & Standardprocesser för resurser.  \\
				\hline
			\end{tabular} \end{table} \FloatBarrier
			\vspace{6mm}

			|Data|Se InheritedResources.

			@@batch@ - aktuell omgång

			@@batches@ - existerande omgångar|

			\paragraph{5.C12 PricingsController}\

			\begin {table} [ht] \begin{tabular} {  p{3.5cm} p{9.6cm} }
				\hline
				Typ & Kontroller  \\
				\hline
				Syfte & Hantering av prisklasser (SR1.30).  \\
				\hline
				Funktion & Administrerng av priser på omgångsnivå. Varje sektion har ett pris för varje rabattklass. Alla pris-/sektions-kombinationer för given omgång administreras på en gång, oavsett om de redan finns eller inte.  \\
				\hline
				Delkomponenter & Inga  \\
				\hline
				Beroenden & Autentiserad som administratör, nästlad via BatchesController.  \\
				\hline
			\end{tabular} \end{table} \FloatBarrier
			\vspace{6mm}

			|Gränssnitt|@index@ - visar formulär för att välja priser för alla sektioner för den aktuella omgången

			@update\_multiple@ - lagrar informationen med datan från @index@|

			\begin {table} [ht] \begin{tabular} {  p{3.5cm} p{9.6cm} }
				\hline
				Resurser & ApplicationController, Batch, Section och Pricing.  \\
				\hline
				Källor & Inga  \\
				\hline
				Process & index > update\_multiple > index  \\
				\hline
			\end{tabular} \end{table} \FloatBarrier
			\vspace{6mm}

			|Data|@@batch@ - aktuell omgång

			@@sections@ - sektioner i omgångens teater

			@@pricings@ - priser för varje sektion- och rabattklasskombination|

			\paragraph{5.C13 UsersController}\

			\begin {table} [ht] \begin{tabular} {  p{3.5cm} p{9.6cm} }
				\hline
				Typ & Kontroller  \\
				\hline
				Syfte & Hantera användare och deras lösenord (SR1.7, SR1.8).  \\
				\hline
				Funktion & Administrering av bokningssystemets användare, inklusive lösenordsändringar och giltighetstid för säljare.  \\
				\hline
				Delkomponenter & Inga  \\
				\hline
				Beroenden & Autentiserad som administratör.  \\
				\hline
				Gränssnitt & Ärver resursactions från InheritedResources.  \\
				\hline
				Resurser & ApplicationController, InheritedResources och User.  \\
				\hline
				Källor & Inga  \\
				\hline
				Process & Standardprocesser för resurser.  \\
				\hline
				Data & Se InheritedResources.  \\
				\hline
			\end{tabular} \end{table} \FloatBarrier


			\paragraph{5.C14 MailingController}\

			\begin {table} [ht] \begin{tabular} {  p{3.5cm} p{9.6cm} }
				\hline
				Typ & Kontroller  \\
				\hline
				Syfte & Hantera manuella och automatiska utskick av mail (SR1.14, SR1.15, SR1.27).  \\
				\hline
				Funktion & Hantera mallar för mail och utskick av mail till förvalda bokningars kontaktpersoner. Nyckelord i mallarna byts ut mot motsvarande bokningsdata vid utskick. Plus: val av mall och editering vid utskick i samma steg med hjälp av AJAX (@index@).  \\
				\hline
				Delkomponenter & Inga  \\
				\hline
				Beroenden & Autentiserad som administratör.  \\
				\hline
			\end{tabular} \end{table} \FloatBarrier
			\vspace{6mm}

			|Gränssnitt|Ärver resursactions från InheritedResources

			@prepare@ - formulär för att skriva och skicka mail utan mall

			@preview@ - förhandsgranska mail innan utskick

			@send@ - skickar mail|

			\begin {table} [ht] \begin{tabular} {  p{3.5cm} p{9.6cm} }
				\hline
				Resurser & ApplicationController, InheritedResources, Reservation och Mail.  \\
				\hline
				Källor & Inga  \\
				\hline
			\end{tabular} \end{table} \FloatBarrier
			\vspace{6mm}

			|Process|Standardprocesser för resurser.

			ReservationsController\#index > index > preview > send - Utskick av existerande mall till valda mottagare.

			ReservationsController\#index > index > new > create > index > preview > send - Utskick av ny mall till valda mottagare.

			ReservationsController\#index > index > prepare > send - Utskick av mail utan mall till valda mottagare.|

			|Data|Se InheritedResources.

			@@reciepts = session[:reciepts]@ - mailmottagare (semipermanent)

			@@mail = session[:mail]@ - den aktuella mailmallen, kan vara temporär för att möjliggöra utskick utan mall|

		\subsubsection{Ekonomichef}


		Ekonomichefen har åtkomst till AdminIndexController (för översikt och statistik) respektive ReservationsController (för att redigera existerande bokningars betalningsstatus) per SR1.19.

		\subsubsection{Säljare}



			\paragraph{5.C15 SalesController}\

			\begin {table} [ht] \begin{tabular} {  p{3.5cm} p{9.6cm} }
				\hline
				Typ & Kontroller  \\
				\hline
				Syfte & Hantering av säljarens gränssnitt och funktioner (SR1.5, SR1.6, SR1.25).  \\
				\hline
				Funktion & Möjlighet att genomköra kontantköp (med nya bokningar) samt ta betalt för, och lämna ut existerande bokningars biljetter. @new@ delas upp i flera privata metoder som anropas beroende på vilket steg i bokningen man befinner sig i.  \\
				\hline
				Delkomponenter & Inga  \\
				\hline
				Beroenden & Autentiserad som säljare.  \\
				\hline
			\end{tabular} \end{table} \FloatBarrier
			\vspace{6mm}

			|Gränssnitt|@index@ - startsida med länkar till ny resp. sök bokning

			@new@ - påbörjar ett nytt kontantköp

			@create@ - sparar data i användarsessionen allt eftersom och vid sista steget lagrar en bokning i databasen

			@find@ - sök efter existerande bokning

			@place@ - stolsplacering av existerande bokning, görs via PlacementsController

			@retrieve@ - presentera vilka biljetter som ska hämtas

			@payment@ - presenterar hur mycket som ska tas betalt

			@finalize@ - uppdaterar bokningens status till betald och utlämnad samt instruerar säljaren om att lämna ut biljetten, går ej att ångra|

			\begin {table} [ht] \begin{tabular} {  p{3.5cm} p{9.6cm} }
				\hline
				Resurser & ApplicationController, Reservation, ReservationCounter och Placement.  \\
				\hline
				Källor & Inga  \\
				\hline
			\end{tabular} \end{table} \FloatBarrier
			\vspace{6mm}

			|Process|index > new > [place]+ > create > retrieve > payment > finalize > index - nytt kontantköp

			index > find > retrieve > finalize > index - lämna ut betald biljett

			index > find > retrieve > payment > finalize > index - lämna ut obetald biljett

			index > find > place > retrieve > payment > finalize > index - lämna ut oplacerad biljett|

			|Data|@@step@ - steg i bokning

			@@steps@ - array med vilka steg som finns

			@@reservation = session[:reservation]@ - bokningsdata|

	\subsection{Vyer}



		\subsubsection{Layouter}



			\paragraph{5.L1 admin\_layout.html}\

			\begin {table} [ht] \begin{tabular} {  p{3.5cm} p{9.6cm} }
				\hline
				Typ & Layout  \\
				\hline
				Syfte & Alla administratörs- och ekonomichefssidor skall ha en gemensam layout (SR1.19, SR1.28)  \\
				\hline
				Funktion & Renderar en layout för administratörens och ekonomichefens sidor, inklusive meny. De länkar i menyn som visas är beroende på vem som är inloggad, ty ekonomichefen ska inte ha tillgång till alla verktyg som administratören har.  \\
				\hline
				Delkomponenter & Inga  \\
				\hline
				Beroenden & Inga  \\
				\hline
				Gränssnitt & Ej applicerbart  \\
				\hline
				Resurser & Inga  \\
				\hline
				Källor & Inga  \\
				\hline
				Process & Inga  \\
				\hline
				Data & @menu@ – Array med alla menylänkar  \\
				\hline
			\end{tabular} \end{table} \FloatBarrier


			\paragraph{5.L2 sales\_layout.html}\

			\begin {table} [ht] \begin{tabular} {  p{3.5cm} p{9.6cm} }
				\hline
				Typ & Layout  \\
				\hline
				Syfte & Säljaren ska ha ett enkelt gränssnitt (SR1.5, SR1.5, SR1.25, SR1.28)  \\
				\hline
				Funktion & Alla sidor som säljaren använder skall ha samma layout.  \\
				\hline
				Delkomponenter & Inga  \\
				\hline
				Beroenden & Inga  \\
				\hline
				Gränssnitt & Ej applicerbart  \\
				\hline
				Resurser & Inga  \\
				\hline
				Källor & Inga  \\
				\hline
				Process & Inga  \\
				\hline
				Data & Inga  \\
				\hline
			\end{tabular} \end{table} \FloatBarrier


			\paragraph{5.L3 simple\_layout.html}\

			\begin {table} [ht] \begin{tabular} {  p{3.5cm} p{9.6cm} }
				\hline
				Typ & Layout  \\
				\hline
				Syfte & Kårspexets personal behöver en layout till sin inloggning (SR1.1, SR9.2, SR1.28)  \\
				\hline
				Funktion & Inloggninssidan ska ha en simpel layout med en ruta där användaren kan logga in.  \\
				\hline
				Delkomponenter & Inga  \\
				\hline
				Beroenden & Inga  \\
				\hline
				Gränssnitt & Ej applicerbart  \\
				\hline
				Resurser & Inga  \\
				\hline
				Källor & Inga  \\
				\hline
				Process & Inga  \\
				\hline
				Data &   \\
				\hline
			\end{tabular} \end{table} \FloatBarrier


			\paragraph{5.L4 customer\_layout.html}\

			\begin {table} [ht] \begin{tabular} {  p{3.5cm} p{9.6cm} }
				\hline
				Typ & Layout  \\
				\hline
				Syfte & Alla bokningsprocessens steg ska ha samma layout (SR1.2)  \\
				\hline
				Funktion & Layout ska matcha Kårspexets nuvarande webbsida, och vara användarvänlig för kunden.  \\
				\hline
				Delkomponenter & Inga  \\
				\hline
				Beroenden & Inga  \\
				\hline
				Gränssnitt & Ej applicerbart  \\
				\hline
				Resurser & Inga  \\
				\hline
				Källor & Inga  \\
				\hline
				Process & Inga  \\
				\hline
				Data & @menu@ – Array med alla menylänkar  \\
				\hline
			\end{tabular} \end{table} \FloatBarrier


		\subsubsection{Inloggning}



			\paragraph{5.V1 SessionController\#new.html}\

			\begin {table} [ht] \begin{tabular} {  p{3.5cm} p{9.6cm} }
				\hline
				Typ & Vy  \\
				\hline
				Syfte & Användaren ska kunna logga in på systemet (SR1.1, SR9.2)  \\
				\hline
				Funktion & Renderar inloggningsformuläret.  \\
				\hline
				Delkomponenter & Inga  \\
				\hline
				Beroenden & SessionController  \\
				\hline
				Gränssnitt & Ej applicerbart  \\
				\hline
				Resurser & 5.L3 simple\_layout.html  \\
				\hline
				Källor & Inga  \\
				\hline
				Process & Ej applicerbart  \\
				\hline
				Data & Inga  \\
				\hline
			\end{tabular} \end{table} \FloatBarrier


		\subsubsection{Admin}



			\paragraph{5.V2 AdminIndexController\#index.html}\

			\begin {table} [ht] \begin{tabular} {  p{3.5cm} p{9.6cm} }
				\hline
				Typ & Vy  \\
				\hline
				Syfte & Administratören behöver en snabb, första översiktssida (SR1.19)  \\
				\hline
				Funktion & Ger en översikt direkt efter inloggning.  \\
				\hline
				Delkomponenter & Inga  \\
				\hline
				Beroenden & AdminIndexController  \\
				\hline
				Gränssnitt & Ej applicerbart  \\
				\hline
				Resurser & 5.L1 admin\_layout.html  \\
				\hline
				Källor & Inga  \\
				\hline
				Process & Inga  \\
				\hline
			\end{tabular} \end{table} \FloatBarrier
			\vspace{6mm}

			|Data|@unplaced - antal betalade bokningar att placera

			@shows - tabell med information om aktuella föreställningar (antal bokade platser mm)|

			\paragraph{5.V3 AdminIndexController\#stat\_standard.html}\

			\begin {table} [ht] \begin{tabular} {  p{3.5cm} p{9.6cm} }
				\hline
				Typ & Vy  \\
				\hline
				Syfte & Administratören och ekonomichefen ska ha möjlighet att se statistik (SR1.16)  \\
				\hline
				Funktion & Visar statistik i form av rådata i en tabell.  \\
				\hline
				Delkomponenter & Inga  \\
				\hline
				Beroenden & AdminIndexController  \\
				\hline
				Gränssnitt & Ej applicerbart  \\
				\hline
				Resurser & 5.L1 admin\_layout.html  \\
				\hline
				Källor & Inga  \\
				\hline
				Process & Inga  \\
				\hline
				Data & @stats - tabell över föreställningar, sektionsgrupper(prisklass), med antal ordinarie, studenter, gratis, platser kvar  \\
				\hline
			\end{tabular} \end{table} \FloatBarrier


			\paragraph{5.V4 AdminIndexController\#stat\_plus.html}\

			\begin {table} [ht] \begin{tabular} {  p{3.5cm} p{9.6cm} }
				\hline
				Typ & Vy  \\
				\hline
				Syfte & Administratören och ekonomichefen ska ha möjlighet att se organiserad statistik (SR1.17)  \\
				\hline
				Funktion & Visar statistik i form av ett flertal tabeller under olika rubriker.  \\
				\hline
				Delkomponenter & Inga  \\
				\hline
				Beroenden & AdminIndexController  \\
				\hline
				Gränssnitt & Ej applicerbart  \\
				\hline
				Resurser & 5.L1 admin\_layout.html  \\
				\hline
				Källor & Inga  \\
				\hline
				Process & Ej applicerbart  \\
				\hline
				Data & @stats - tabell över föreställningar, sektionsgrupper(prisklass), med antal ordinarie, studenter, gratis, platser kvar  \\
				\hline
			\end{tabular} \end{table} \FloatBarrier


			\paragraph{5.V5 AdminIndexController\#stat\_deluxe.html}\

			\begin {table} [ht] \begin{tabular} {  p{3.5cm} p{9.6cm} }
				\hline
				Typ & Vy  \\
				\hline
				Syfte & Administratören och ekonomichefen ska ha möjlighet att se organiserad, grafisk statistik (SR1.18)  \\
				\hline
				Funktion & Visar statistik i form av en interaktiv graf.  \\
				\hline
				Delkomponenter & Inga  \\
				\hline
				Beroenden & AdminIndexController  \\
				\hline
				Gränssnitt & Ej applicerbart  \\
				\hline
				Resurser & 5.L1 admin\_layout.html  \\
				\hline
				Källor & Inga  \\
				\hline
				Process & Ej applicerbart  \\
				\hline
				Data & @stats - tabell över föreställningar, sektionsgrupper(prisklass), med antal ordinarie, studenter, gratis, platser kvar  \\
				\hline
			\end{tabular} \end{table} \FloatBarrier


			\paragraph{5.V6 ReservationsController\#index.html}\

			\begin {table} [ht] \begin{tabular} {  p{3.5cm} p{9.6cm} }
				\hline
				Typ & Vy  \\
				\hline
				Syfte & Administratören ska kunna hantera bokningar (SR1.13)  \\
				\hline
				Funktion & Administratören får en överblick över de val han kan göra angående bokningar.  \\
				\hline
				Delkomponenter & Inga  \\
				\hline
				Beroenden & ReservationsController  \\
				\hline
				Gränssnitt & Ej applicerbart  \\
				\hline
				Resurser & 5.L1 admin\_layout.html  \\
				\hline
				Källor & Inga  \\
				\hline
				Process & Ej applicerbart  \\
				\hline
			\end{tabular} \end{table} \FloatBarrier
			\vspace{6mm}

			|Data|@reservations

			@filters - array med möjliga filter samt indikationer för vilka som är aktiva (index)|

			\paragraph{5.V7 ReservationsController\#new.html}\

			\begin {table} [ht] \begin{tabular} {  p{3.5cm} p{9.6cm} }
				\hline
				Typ & Vy  \\
				\hline
				Syfte & Administratören ska kunna hantera bokningar (SR1.13)  \\
				\hline
				Funktion & Administratören ska kunna skapa en ny bokning från sitt gränssnitt.  \\
				\hline
				Delkomponenter & AD5.P1  \\
				\hline
				Beroenden & ReservationsController  \\
				\hline
				Gränssnitt & Ej applicerbart  \\
				\hline
				Resurser & 5.L1 admin\_layout.html  \\
				\hline
				Källor & Inga  \\
				\hline
				Process & Ej applicerbart  \\
				\hline
				Data & @reservation  \\
				\hline
			\end{tabular} \end{table} \FloatBarrier


			\paragraph{5.V8 ReservationsController\#show.html}\

			\begin {table} [ht] \begin{tabular} {  p{3.5cm} p{9.6cm} }
				\hline
				Typ & Vy  \\
				\hline
				Syfte & Administratören ska kunna hantera bokningar (SR1.13)  \\
				\hline
				Funktion & Administratören ska kunna visa att en bokning ligger i databasen genom att söka efter den med sökfunktionen eller filter.  \\
				\hline
				Delkomponenter & Inga  \\
				\hline
				Beroenden & ReservationsController  \\
				\hline
				Gränssnitt & Ej applicerbart  \\
				\hline
				Resurser & 5.L1 admin\_layout.html  \\
				\hline
				Källor & Inga  \\
				\hline
				Process & Ej applicerbart  \\
				\hline
				Data & @reservation  \\
				\hline
			\end{tabular} \end{table} \FloatBarrier


			\paragraph{5.V9 ReservationsController\#edit.html}\

			\begin {table} [ht] \begin{tabular} {  p{3.5cm} p{9.6cm} }
				\hline
				Typ & Vy  \\
				\hline
				Syfte & Administratören ska kunna hantera bokningar (SR1.13)  \\
				\hline
				Funktion & Administratören ska kunna göra ändringar i en bokning, till exempel ändra antalet biljetter, ändra betalstatus, placeringsstatus, och så vidare.  \\
				\hline
				Delkomponenter & AD5.P1  \\
				\hline
				Beroenden & ReservationsController  \\
				\hline
				Gränssnitt & Ej applicerbart  \\
				\hline
				Resurser & 5.L1 admin\_layout.html  \\
				\hline
				Källor & Inga  \\
				\hline
				Process & Ej applicerbart  \\
				\hline
				Data & @reservation  \\
				\hline
			\end{tabular} \end{table} \FloatBarrier


			\paragraph{5.P1 ReservationsController\#\_form.html}\

			\begin {table} [ht] \begin{tabular} {  p{3.5cm} p{9.6cm} }
				\hline
				Typ & Partiell vy  \\
				\hline
				Syfte & Administratören ska kunna hantera bokningar (SR1.13)  \\
				\hline
				Funktion & Renderar ett formulär för att redigera Reservations-objekt.  \\
				\hline
				Delkomponenter & Inga  \\
				\hline
				Beroenden & ReservationsController  \\
				\hline
				Gränssnitt & Ej applicerbart  \\
				\hline
				Resurser & Reservation  \\
				\hline
				Källor & Inga  \\
				\hline
				Process & Ej applicerbart  \\
				\hline
				Data & @reservation  \\
				\hline
			\end{tabular} \end{table} \FloatBarrier


			\paragraph{5.V10 PlacementsController\#new.html}\

			\begin {table} [ht] \begin{tabular} {  p{3.5cm} p{9.6cm} }
				\hline
				Typ & Vy  \\
				\hline
				Syfte & Administratören ska kunna placera ut bokningar (SR1.11, SR1.25)  \\
				\hline
				Funktion & Administratören ska kunna placera ut platser i den korrekta sektionen som kunden har bokat.  \\
				\hline
				Delkomponenter & Inga  \\
				\hline
				Beroenden & PlacementsController  \\
				\hline
				Gränssnitt & Ej applicerbart  \\
				\hline
				Resurser & 5.L1 admin\_layout.html  \\
				\hline
				Källor & Inga  \\
				\hline
				Process & Ej applicerbart  \\
				\hline
			\end{tabular} \end{table} \FloatBarrier
			\vspace{6mm}

			|Data|@reservation - den aktulla bokningsinstansen

			@placements - placeringar

			@theater - aktuell teater (bild mm)

			@show - information om föreställningen (datum mm)

			@seats - stolar med positioner och placeringsstatus (upptagen, tillfälligt låst, ledig, vald för denna bokning)|

			\paragraph{5.V11 PlacementsController\#edit.html}\

			\begin {table} [ht] \begin{tabular} {  p{3.5cm} p{9.6cm} }
				\hline
				Typ & Vy  \\
				\hline
				Syfte & Administratören ska kunna placera ut bokningar (SR1.11, SR1.25)  \\
				\hline
				Funktion & Administratören ska kunna ändra placeringen av en bokning, så länge som den inte har hämtats ut av kunden.  \\
				\hline
				Delkomponenter & Inga  \\
				\hline
				Beroenden & PlacementsController  \\
				\hline
				Gränssnitt & Ej applicerbart  \\
				\hline
				Resurser & 5.L1 admin\_layout.html  \\
				\hline
				Källor & Inga  \\
				\hline
				Process & Ej applicerbart  \\
				\hline
			\end{tabular} \end{table} \FloatBarrier
			\vspace{6mm}

			|Data|@reservation - den aktulla bokningsinstansen

			@placements - placeringar

			@theater - aktuell teater (bild mm)

			@show - information om föreställningen (datum mm)

			@seats - stolar med positioner och placeringsstatus (upptagen, tillfälligt låst, ledig, vald för denna bokning)|

			\paragraph{5.V12 TheatersController\#index.html}\

			\begin {table} [ht] \begin{tabular} {  p{3.5cm} p{9.6cm} }
				\hline
				Typ & Vy  \\
				\hline
				Syfte & Administratören ska kunna hantera teatrar (SR1.9)  \\
				\hline
				Funktion & Administratören får en överblick över de teatrar som finns i databasen och vad han kan göra med dem.  \\
				\hline
				Delkomponenter & Inga  \\
				\hline
				Beroenden & TheatersController  \\
				\hline
				Gränssnitt & Ej applicerbart  \\
				\hline
				Resurser & 5.L1 admin\_layout.html  \\
				\hline
				Källor & Inga  \\
				\hline
				Process & Ej applicerbart  \\
				\hline
				Data & @theaters - existerande teatrars attributer   \\
				\hline
			\end{tabular} \end{table} \FloatBarrier


			\paragraph{5.V13 TheatersController\#new.html}\

			\begin {table} [ht] \begin{tabular} {  p{3.5cm} p{9.6cm} }
				\hline
				Typ & Vy  \\
				\hline
				Syfte & Administratören ska kunna hantera teatrar (SR1.9)  \\
				\hline
				Funktion & Administratören ska genom ett speciellt gränssnitt kunna skapa en ny teater.  \\
				\hline
				Delkomponenter & AD5.P3  \\
				\hline
				Beroenden & TheatersController  \\
				\hline
				Gränssnitt & Ej applicerbart  \\
				\hline
				Resurser & 5.L1 admin\_layout.html  \\
				\hline
				Källor & Inga  \\
				\hline
				Process & Ej applicerbart  \\
				\hline
				Data & @theater  \\
				\hline
			\end{tabular} \end{table} \FloatBarrier


			\paragraph{5.V14 TheatersController\#show.html}\

			\begin {table} [ht] \begin{tabular} {  p{3.5cm} p{9.6cm} }
				\hline
				Typ & Vy  \\
				\hline
				Syfte & Administratören ska kunna hantera teatrar (SR1.9)  \\
				\hline
				Funktion & Administratören ska kunna se vilka teatrar som finns.  \\
				\hline
				Delkomponenter & Inga  \\
				\hline
				Beroenden & TheatersController  \\
				\hline
				Gränssnitt & Ej applicerbart  \\
				\hline
				Resurser & 5.L1 admin\_layout.html  \\
				\hline
				Källor & Inga  \\
				\hline
				Process & Ej applicerbart  \\
				\hline
				Data & @theater  \\
				\hline
			\end{tabular} \end{table} \FloatBarrier


			\paragraph{5.V15 TheatersController\#edit.html}\

			\begin {table} [ht] \begin{tabular} {  p{3.5cm} p{9.6cm} }
				\hline
				Typ & Vy  \\
				\hline
				Syfte & Administratören ska kunna hantera teatrar (SR1.9)  \\
				\hline
				Funktion & Administratören ska kunna göra ändringar till en teater, till exempel om en sektion är under reparation så ska kunder inte kunna boka platser i den.  \\
				\hline
				Delkomponenter & AD5.P3  \\
				\hline
				Beroenden & TheatersController  \\
				\hline
				Gränssnitt & Ej applicerbart  \\
				\hline
				Resurser & 5.L1 admin\_layout.html  \\
				\hline
				Källor & Inga  \\
				\hline
				Process & Ej applicerbart  \\
				\hline
				Data & @theater  \\
				\hline
			\end{tabular} \end{table} \FloatBarrier


			\paragraph{5.P2 TheatersController\#\_form}\

			\begin {table} [ht] \begin{tabular} {  p{3.5cm} p{9.6cm} }
				\hline
				Typ & Partiell vy  \\
				\hline
				Syfte & Administratören ska kunna hantera teatrar (SR1.9)  \\
				\hline
				Funktion & Renderar ett formulär för att redigera Theaters-objekt.  \\
				\hline
				Delkomponenter & Inga  \\
				\hline
				Beroenden & TheatersController  \\
				\hline
				Gränssnitt & Ej applicerbart  \\
				\hline
				Resurser & Theater  \\
				\hline
				Källor & Inga  \\
				\hline
				Process & Ej applicerbart  \\
				\hline
				Data & @theater  \\
				\hline
			\end{tabular} \end{table} \FloatBarrier


			\paragraph{5.V16 SectionsController\#index.html}\

			\begin {table} [ht] \begin{tabular} {  p{3.5cm} p{9.6cm} }
				\hline
				Typ & Vy  \\
				\hline
				Syfte & Administratören och säljaren ska kunna välja sektionsplaceringen (SR1.22, SR1.23)  \\
				\hline
				Funktion & Administratören och säljaren ska få en överblick över de sektioner som platser kan placeras ut i och hur detta kan göras.  \\
				\hline
				Delkomponenter & Inga  \\
				\hline
				Beroenden & SectionsController  \\
				\hline
				Gränssnitt & Ej applicerbart  \\
				\hline
				Resurser & 5.L1 admin\_layout.html  \\
				\hline
				Källor & Inga  \\
				\hline
				Process & Ej applicerbart  \\
				\hline
			\end{tabular} \end{table} \FloatBarrier
			\vspace{6mm}

			|Data|@sections

			@theater - aktuell teater|

			\paragraph{5.V17 SectionsController\#new.html}\

			\begin {table} [ht] \begin{tabular} {  p{3.5cm} p{9.6cm} }
				\hline
				Typ & Vy  \\
				\hline
				Syfte & Administratören och säljaren ska kunna välja sektionsplaceringen (SR1.22)  \\
				\hline
				Funktion & Administratören och säljaren ska kunna välja en sektion att placera en boknings platser i.  \\
				\hline
				Delkomponenter & AD5.P4  \\
				\hline
				Beroenden & SectionsController  \\
				\hline
				Gränssnitt & Ej applicerbart  \\
				\hline
				Resurser & 5.L1 admin\_layout.html  \\
				\hline
				Källor & Inga  \\
				\hline
				Process & Ej applicerbart  \\
				\hline
			\end{tabular} \end{table} \FloatBarrier
			\vspace{6mm}

			|Data|@section

			@theater - aktuell teater|

			\paragraph{5.V18 SectionsController\#show.html}\

			\begin {table} [ht] \begin{tabular} {  p{3.5cm} p{9.6cm} }
				\hline
				Typ & Vy  \\
				\hline
				Syfte & Administratören och säljaren ska kunna välja sektionsplaceringen (SR1.22, SR1.23 (i plus))  \\
				\hline
				Funktion & Administratören och säljaren ska kunna se alla sektioner som är valbara.  \\
				\hline
				Delkomponenter & Inga  \\
				\hline
				Beroenden & SectionsController  \\
				\hline
				Gränssnitt & Ej applicerbart  \\
				\hline
				Resurser & 5.L1 admin\_layout.html  \\
				\hline
				Källor & Inga  \\
				\hline
				Process & Ej applicerbart  \\
				\hline
				Data & @theater - aktuell teater  \\
				\hline
			\end{tabular} \end{table} \FloatBarrier


			\paragraph{5.V19 SectionsController\#edit.html}\

			\begin {table} [ht] \begin{tabular} {  p{3.5cm} p{9.6cm} }
				\hline
				Typ & Vy  \\
				\hline
				Syfte & Administratören och säljaren ska kunna välja sektionsplaceringen (SR1.22)  \\
				\hline
				Funktion & Administratören ska kunna ändra vilka sektioner som är valbara.  \\
				\hline
				Delkomponenter & AD5.P4  \\
				\hline
				Beroenden & SectionsController  \\
				\hline
				Gränssnitt & Ej applicerbart  \\
				\hline
				Resurser & 5.L1 admin\_layout.html  \\
				\hline
				Källor & Inga  \\
				\hline
				Process & Ej applicerbart  \\
				\hline
			\end{tabular} \end{table} \FloatBarrier
			\vspace{6mm}

			|Data|@section

			@theater - aktuell teater|

			\paragraph{5.P3 SectionsController\#\_form}\

			\begin {table} [ht] \begin{tabular} {  p{3.5cm} p{9.6cm} }
				\hline
				Typ & Partiell vy  \\
				\hline
				Syfte & Administratören och säljaren ska kunna välja sektionsplaceringen (SR1.22)  \\
				\hline
				Funktion & Renderar ett formulär för att redigera Sections-objekt.  \\
				\hline
				Delkomponenter & Inga  \\
				\hline
				Beroenden & SectionsController  \\
				\hline
				Gränssnitt & Ej applicerbart  \\
				\hline
				Resurser & Section  \\
				\hline
				Källor & Inga  \\
				\hline
				Process & Ej applicerbart  \\
				\hline
			\end{tabular} \end{table} \FloatBarrier
			\vspace{6mm}

			|Data|@section

			@theater - aktuell teater|

			\paragraph{5.V20 SeatsController\#index.html}\

			\begin {table} [ht] \begin{tabular} {  p{3.5cm} p{9.6cm} }
				\hline
				Typ & Vy  \\
				\hline
				Syfte & Administratören ska kunna hantera stolsplacering (SR1.11, SR1.25)  \\
				\hline
				Funktion & Administratören ska kunna identifiera och placera ut stolar för en given teater.  \\
				\hline
				Delkomponenter & Inga  \\
				\hline
				Beroenden & SeatsController  \\
				\hline
				Gränssnitt & Ej applicerbart  \\
				\hline
				Resurser & 5.L1 admin\_layout.html  \\
				\hline
				Källor & Inga  \\
				\hline
				Process & Ej applicerbart  \\
				\hline
			\end{tabular} \end{table} \FloatBarrier
			\vspace{6mm}

			|Data|@seats

			@theater - aktuell teater|

			\paragraph{5.V21 BatchesController\#index.html}\

			\begin {table} [ht] \begin{tabular} {  p{3.5cm} p{9.6cm} }
				\hline
				Typ & Vy  \\
				\hline
				Syfte & Administratören skall kunna lägga till omgångar (SR1.10)  \\
				\hline
				Funktion & Ge en översikt på alla omgångar.  \\
				\hline
				Delkomponenter & Inga  \\
				\hline
				Beroenden & BatchesController  \\
				\hline
				Gränssnitt & Ej applicerbart  \\
				\hline
				Resurser & 5.L1 admin\_layout.html  \\
				\hline
				Källor & Inga  \\
				\hline
				Process & Ej applicerbart  \\
				\hline
				Data & @batches  \\
				\hline
			\end{tabular} \end{table} \FloatBarrier


			\paragraph{5.V22 BatchesController\#new.html}\

			\begin {table} [ht] \begin{tabular} {  p{3.5cm} p{9.6cm} }
				\hline
				Typ & Vy  \\
				\hline
				Syfte & Administratören skall kunna lägga till omgångar (SR1.10)  \\
				\hline
				Funktion & Skapa en ny omgång.  \\
				\hline
				Delkomponenter & AD5.3.36  \\
				\hline
				Beroenden & BatchesController  \\
				\hline
				Gränssnitt & Ej applicerbart  \\
				\hline
				Resurser & 5.L1 admin\_layout.html  \\
				\hline
				Källor & Inga  \\
				\hline
				Process & Ej applicerbart  \\
				\hline
				Data & @batch  \\
				\hline
			\end{tabular} \end{table} \FloatBarrier


			\paragraph{5.V23 BatchesController\#show.html}\

			\begin {table} [ht] \begin{tabular} {  p{3.5cm} p{9.6cm} }
				\hline
				Typ & Vy  \\
				\hline
				Syfte & Administratören skall kunna lägga till omgångar (SR1.10)  \\
				\hline
				Funktion & Visa information om en omgång, till exemple alla föreställningar som hör till omgången.  \\
				\hline
				Delkomponenter & Inga  \\
				\hline
				Beroenden & BatchesController  \\
				\hline
				Gränssnitt & Ej applicerbart  \\
				\hline
				Resurser & 5.L1 admin\_layout.html  \\
				\hline
				Källor & Inga  \\
				\hline
				Process & Ej applicerbart  \\
				\hline
				Data & @batch  \\
				\hline
			\end{tabular} \end{table} \FloatBarrier


			\paragraph{5.V24 BatchesController\#edit.html}\

			\begin {table} [ht] \begin{tabular} {  p{3.5cm} p{9.6cm} }
				\hline
				Typ & Vy  \\
				\hline
				Syfte & Administratören skall kunna lägga till omgångar (SR1.10)  \\
				\hline
				Funktion & Ändra omgången och vilka föreställningar som ingår i den.  \\
				\hline
				Delkomponenter & AD5.3.36  \\
				\hline
				Beroenden & BatchesController  \\
				\hline
				Gränssnitt & Ej applicerbart  \\
				\hline
				Resurser & 5.L1 admin\_layout.html  \\
				\hline
				Källor & Inga  \\
				\hline
				Process & Ej applicerbart  \\
				\hline
				Data & @batch  \\
				\hline
			\end{tabular} \end{table} \FloatBarrier


			\paragraph{5.P4 BatchesController\#\_form}\

			\begin {table} [ht] \begin{tabular} {  p{3.5cm} p{9.6cm} }
				\hline
				Typ & Partiell vy  \\
				\hline
				Syfte & Administratören skall kunna lägga till omgångar (SR1.10)  \\
				\hline
				Funktion & Renderar ett formulär för att redigera Batches-objekt.  \\
				\hline
				Delkomponenter & Inga  \\
				\hline
				Beroenden & BatchesController  \\
				\hline
				Gränssnitt & Ej applicerbart  \\
				\hline
				Resurser & Batch  \\
				\hline
				Källor & Inga  \\
				\hline
				Process & Ej applicerbart  \\
				\hline
				Data & @batch  \\
				\hline
			\end{tabular} \end{table} \FloatBarrier


			\paragraph{5.V25 ShowsController\#index.html}\

			\begin {table} [ht] \begin{tabular} {  p{3.5cm} p{9.6cm} }
				\hline
				Typ & Vy  \\
				\hline
				Syfte & Administratören skall kunna hantera föreställningar (SR1.10, SR1.20)  \\
				\hline
				Funktion & Ger översikt på alla föreställningar som hör till en vald omgång.  \\
				\hline
				Delkomponenter & Inga  \\
				\hline
				Beroenden & ShowsController  \\
				\hline
				Gränssnitt & Ej applicerbart  \\
				\hline
				Resurser & 5.L1 admin\_layout.html  \\
				\hline
				Källor & Inga  \\
				\hline
				Process & Ej applicerbart  \\
				\hline
			\end{tabular} \end{table} \FloatBarrier
			\vspace{6mm}

			|Data|@shows

			@batch - aktuell omgång

			@batches - existerande omgångar|

			\paragraph{5.V26 ShowsController\#new.html}\

			\begin {table} [ht] \begin{tabular} {  p{3.5cm} p{9.6cm} }
				\hline
				Typ & Vy  \\
				\hline
				Syfte & Administratören skall kunna hantera föreställningar (SR1.10)  \\
				\hline
				Funktion & Lägga till nya föreställningar till en omgång.  \\
				\hline
				Delkomponenter & AD5.3.41  \\
				\hline
				Beroenden & ShowsController  \\
				\hline
				Gränssnitt & Ej applicerbart  \\
				\hline
				Resurser & 5.L1 admin\_layout.html  \\
				\hline
				Källor & Inga  \\
				\hline
				Process & Ej applicerbart  \\
				\hline
			\end{tabular} \end{table} \FloatBarrier
			\vspace{6mm}

			|Data|@show

			@batch - aktuell omgång

			@batches - existerande omgångar|

			\paragraph{5.V27 ShowsController\#show.html}\

			\begin {table} [ht] \begin{tabular} {  p{3.5cm} p{9.6cm} }
				\hline
				Typ & Vy  \\
				\hline
				Syfte & (Administratören skall kunna hantera föreställningar SR1.10)  \\
				\hline
				Funktion & Visa information om en vald föreställning, till exempel antal lediga platser  \\
				\hline
				Delkomponenter & Inga  \\
				\hline
				Beroenden & ShowsController  \\
				\hline
				Gränssnitt & Ej applicerbart  \\
				\hline
				Resurser & 5.L1 admin\_layout.html  \\
				\hline
				Källor & Inga  \\
				\hline
				Process & Ej applicerbart  \\
				\hline
			\end{tabular} \end{table} \FloatBarrier
			\vspace{6mm}

			|Data|@batch - aktuell omgång

			@batches - existerande omgångar|

			\paragraph{5.V28 ShowsController\#edit.html}\

			\begin {table} [ht] \begin{tabular} {  p{3.5cm} p{9.6cm} }
				\hline
				Typ & Vy  \\
				\hline
				Syfte & Administratören skall kunna hantera föreställningar (SR1.10)  \\
				\hline
				Funktion & Ändra information på en vald föreställning, till exempel datum.  \\
				\hline
				Delkomponenter & AD5.3.41  \\
				\hline
				Beroenden & ShowsController  \\
				\hline
				Gränssnitt & Ej applicerbart  \\
				\hline
				Resurser & 5.L1 admin\_layout.html  \\
				\hline
				Källor & Inga  \\
				\hline
				Process & Ej applicerbart  \\
				\hline
			\end{tabular} \end{table} \FloatBarrier
			\vspace{6mm}

			|Data|@show

			@batch - aktuell omgång

			@batches - existerande omgångar|

			\paragraph{5.P5 ShowsController\#\_form}\

			\begin {table} [ht] \begin{tabular} {  p{3.5cm} p{9.6cm} }
				\hline
				Typ & Partiell vy  \\
				\hline
				Syfte & Administratören skall kunna hantera föreställningar (SR1.10)  \\
				\hline
				Funktion & Renderar ett formulär för att redigera Shows-objekt.  \\
				\hline
				Delkomponenter & Inga  \\
				\hline
				Beroenden & ShowsController  \\
				\hline
				Gränssnitt & Ej applicerbart  \\
				\hline
				Resurser & Show  \\
				\hline
				Källor & Inga  \\
				\hline
				Process & Ej applicerbart  \\
				\hline
			\end{tabular} \end{table} \FloatBarrier
			\vspace{6mm}

			|Data|@show

			@batch - aktuell omgång

			@batches - existerande omgångar|

			\paragraph{5.V29 PricingsController\#index.html}\

			\begin {table} [ht] \begin{tabular} {  p{3.5cm} p{9.6cm} }
				\hline
				Typ & Vy  \\
				\hline
				Syfte & Administratören och ekonomichefen ska kunna registrera betalningar (SR1.12)  \\
				\hline
				Funktion & Administratören och ekonomichefen behöver kunna registrera om en betalning är gjord, så att administratören eller eventuellt säljaren kan placera ut bokningens platser.  \\
				\hline
				Delkomponenter & Inga  \\
				\hline
				Beroenden & PricingsController  \\
				\hline
				Gränssnitt & Ej applicerbart  \\
				\hline
				Resurser & 5.L1 admin\_layout.html  \\
				\hline
				Källor & Inga  \\
				\hline
				Process & Ej applicerbart  \\
				\hline
			\end{tabular} \end{table} \FloatBarrier
			\vspace{6mm}

			|Data|@batch - aktuell omgång

			@sections - sektioner i omgångens teater

			@pricings - priser för varje sektion- och rabattklasskombination|

			\paragraph{5.V30 UsersController\#index.html}\

			\begin {table} [ht] \begin{tabular} {  p{3.5cm} p{9.6cm} }
				\hline
				Typ & Vy  \\
				\hline
				Syfte & Administratören ska kunna hantera systemets användarnas konton (SR1.7, SR1.8)  \\
				\hline
				Funktion & Administratören får en överblick över de användarne som finns i systemet, och deras rättigheter.  \\
				\hline
				Delkomponenter & Inga  \\
				\hline
				Beroenden & UsersController  \\
				\hline
				Gränssnitt & Ej applicerbart  \\
				\hline
				Resurser & 5.L1 admin\_layout.html  \\
				\hline
				Källor & Inga  \\
				\hline
				Process & Ej applicerbart  \\
				\hline
				Data & @users  \\
				\hline
			\end{tabular} \end{table} \FloatBarrier


			\paragraph{5.V31 UsersController\#new.html}\

			\begin {table} [ht] \begin{tabular} {  p{3.5cm} p{9.6cm} }
				\hline
				Typ & Vy  \\
				\hline
				Syfte & Administratören ska kunna hantera systemets användarnas konton (SR1.7, SR1.8)  \\
				\hline
				Funktion & Administratören ska kunna skapa en ny användare och sätta dennnes rättigheter i systemet.  \\
				\hline
				Delkomponenter & AD5.P7  \\
				\hline
				Beroenden & UsersController  \\
				\hline
				Gränssnitt & Ej applicerbart  \\
				\hline
				Resurser & 5.L1 admin\_layout.html  \\
				\hline
				Källor & Inga  \\
				\hline
				Process & Ej applicerbart  \\
				\hline
				Data & @user  \\
				\hline
			\end{tabular} \end{table} \FloatBarrier


			\paragraph{5.V32 UsersController\#show.html}\

			\begin {table} [ht] \begin{tabular} {  p{3.5cm} p{9.6cm} }
				\hline
				Typ & Vy  \\
				\hline
				Syfte & Administratören ska kunna hantera systemets användarnas konton (SR1.7, SR1.8)  \\
				\hline
				Funktion & Administratören ska kunna granska en enskild användare och dennes rättigheter.  \\
				\hline
				Delkomponenter & Inga  \\
				\hline
				Beroenden & UsersController  \\
				\hline
				Gränssnitt & Ej applicerbart  \\
				\hline
				Resurser & 5.L1 admin\_layout.html  \\
				\hline
				Källor & Inga  \\
				\hline
				Process & Ej applicerbart  \\
				\hline
				Data & @user  \\
				\hline
			\end{tabular} \end{table} \FloatBarrier


			\paragraph{5.V33 UsersController\#edit.html}\

			\begin {table} [ht] \begin{tabular} {  p{3.5cm} p{9.6cm} }
				\hline
				Typ & Vy  \\
				\hline
				Syfte & Administratören ska kunna hantera systemets användarnas konton (SR1.7, SR1.8)  \\
				\hline
				Funktion & Administratören ska kunna ändra en användares rättigheter eller lösenord, till exempel säljare som inte ska säljare längre.  \\
				\hline
				Delkomponenter & AD5.P7  \\
				\hline
				Beroenden & UsersController  \\
				\hline
				Gränssnitt & Ej applicerbart  \\
				\hline
				Resurser & 5.L1 admin\_layout.html  \\
				\hline
				Källor & Inga  \\
				\hline
				Process & Ej applicerbart  \\
				\hline
				Data & @user  \\
				\hline
			\end{tabular} \end{table} \FloatBarrier


			\paragraph{5.P6 UsersController\#\_form}\

			\begin {table} [ht] \begin{tabular} {  p{3.5cm} p{9.6cm} }
				\hline
				Typ & Partiell vy  \\
				\hline
				Syfte & Administratören ska kunna hantera systemets användarnas konton (SR1.7, SR1.8)  \\
				\hline
				Funktion & Renderar ett formulär för att redigera User-objekt.  \\
				\hline
				Delkomponenter & Inga  \\
				\hline
				Beroenden & UsersController  \\
				\hline
				Gränssnitt & Ej applicerbart  \\
				\hline
				Resurser & User  \\
				\hline
				Källor & Inga  \\
				\hline
				Process & Ej applicerbart  \\
				\hline
				Data & @user  \\
				\hline
			\end{tabular} \end{table} \FloatBarrier


			\paragraph{5.V34 MailingController\#index.html}\

			\begin {table} [ht] \begin{tabular} {  p{3.5cm} p{9.6cm} }
				\hline
				Typ & Vy  \\
				\hline
				Syfte & Administratören ska kunna hantera mailutskick (SR1.14)  \\
				\hline
				Funktion & Ge en överiskt på bland annat sparade mailmallar.  \\
				\hline
				Delkomponenter & Inga  \\
				\hline
				Beroenden & MailingController  \\
				\hline
				Gränssnitt & Ej applicerbart  \\
				\hline
				Resurser & 5.L1 admin\_layout.html  \\
				\hline
				Källor & Inga  \\
				\hline
				Process & Ej applicerbart  \\
				\hline
			\end{tabular} \end{table} \FloatBarrier
			\vspace{6mm}

			|Data|@mailing

			@reciepts = session[:reciepts] - mailmottagare (semipermanent)

			@mail = session[:mail] - den aktuella mailmallen, kan vara temporär för att möjliggöra utskick utan mall|

			\paragraph{5.V35 MailingController\#new.html}\

			\begin {table} [ht] \begin{tabular} {  p{3.5cm} p{9.6cm} }
				\hline
				Typ & Vy  \\
				\hline
				Syfte & Administratören ska kunna hantera mailutskick (SR1.14)  \\
				\hline
				Funktion & Skapa en ny mailmall för att skicka till en eller flera mottagare.  \\
				\hline
				Delkomponenter & AD5.3.56  \\
				\hline
				Beroenden & MailingController  \\
				\hline
				Gränssnitt & Ej applicerbart  \\
				\hline
				Resurser & 5.L1 admin\_layout.html  \\
				\hline
				Källor & Inga  \\
				\hline
				Process & Ej applicerbart  \\
				\hline
			\end{tabular} \end{table} \FloatBarrier
			\vspace{6mm}

			|Data|@mail

			@reciepts = session[:reciepts] - mailmottagare (semipermanent)

			@mail = session[:mail] - den aktuella mailmallen, kan vara temporär för att möjliggöra utskick utan mall||

			\paragraph{5.V36 MailingController\#show.html}\

			\begin {table} [ht] \begin{tabular} {  p{3.5cm} p{9.6cm} }
				\hline
				Typ & Vy  \\
				\hline
				Syfte & Administratören ska kunna hantera mailutskick (SR1.14, SR1.15)  \\
				\hline
				Funktion & Visa ett mailmall som sparats.   \\
				\hline
				Delkomponenter & Inga  \\
				\hline
				Beroenden & MailingController  \\
				\hline
				Gränssnitt & Ej applicerbart  \\
				\hline
				Resurser & 5.L1 admin\_layout.html  \\
				\hline
				Källor & Inga  \\
				\hline
				Process & Ej applicerbart  \\
				\hline
			\end{tabular} \end{table} \FloatBarrier
			\vspace{6mm}

			|Data|@mail

			@reciepts = session[:reciepts] - mailmottagare (semipermanent)

			@mail = session[:mail] - den aktuella mailmallen, kan vara temporär för att möjliggöra utskick utan mall|

			\paragraph{5.V37 MailingController\#edit.html}\

			\begin {table} [ht] \begin{tabular} {  p{3.5cm} p{9.6cm} }
				\hline
				Typ & Vy  \\
				\hline
				Syfte & Administratören ska kunna hantera mailutskick (SR1.14)  \\
				\hline
				Funktion & Ändra mottagare och innehåll i en befintlig mailmall.  \\
				\hline
				Delkomponenter & AD5.3.56  \\
				\hline
				Beroenden & MailingController  \\
				\hline
				Gränssnitt & Ej applicerbart  \\
				\hline
				Resurser & 5.L1 admin\_layout.html  \\
				\hline
				Källor & Inga  \\
				\hline
				Process & Ej applicerbart  \\
				\hline
			\end{tabular} \end{table} \FloatBarrier
			\vspace{6mm}

			|Data|@mail

			@reciepts = session[:reciepts] - mailmottagare (semipermanent)

			@mail = session[:mail] - den aktuella mailmallen, kan vara temporär för att möjliggöra utskick utan mall|

			\paragraph{5.V38 MailingController\#preview.html}\

			\begin {table} [ht] \begin{tabular} {  p{3.5cm} p{9.6cm} }
				\hline
				Typ & Vy  \\
				\hline
				Syfte & Administratören ska kunna hantera mailutskick (SR1.14)  \\
				\hline
				Funktion & Förhandsgranska de mail som ska skickas innan utskick.  \\
				\hline
				Delkomponenter & Inga  \\
				\hline
				Beroenden & MailingController  \\
				\hline
				Gränssnitt & Ej applicerbart  \\
				\hline
				Resurser & 5.L1 admin\_layout.html  \\
				\hline
				Källor & Inga  \\
				\hline
				Process & Ej applicerbart  \\
				\hline
			\end{tabular} \end{table} \FloatBarrier
			\vspace{6mm}

			|Data|@mail

			@reciepts = session[:reciepts] - mailmottagare (semipermanent)

			@mail = session[:mail] - den aktuella mailmallen, kan vara temporär för att möjliggöra utskick utan mall|

			\paragraph{5.P7 MailingController\#\_form}\

			\begin {table} [ht] \begin{tabular} {  p{3.5cm} p{9.6cm} }
				\hline
				Typ & Partiell vy  \\
				\hline
				Syfte & Administratören ska kunna hantera mailutskick (SR1.14)  \\
				\hline
				Funktion & Renderar ett formulär för att redigera Mailng-objekt.  \\
				\hline
				Delkomponenter & Inga  \\
				\hline
				Beroenden & MailingController  \\
				\hline
				Gränssnitt & Ej applicerbart  \\
				\hline
				Resurser & Mail  \\
				\hline
				Källor & Inga  \\
				\hline
				Process & Ej applicerbart  \\
				\hline
			\end{tabular} \end{table} \FloatBarrier
			\vspace{6mm}

			|Data|@mail

			@reciepts = session[:reciepts] - mailmottagare (semipermanent)

			@mail = session[:mail] - den aktuella mailmallen, kan vara temporär för att möjliggöra utskick utan mall|

		\subsubsection{Säljare}



			\paragraph{5.V39 SalesController\#index.html}\

			\begin {table} [ht] \begin{tabular} {  p{3.5cm} p{9.6cm} }
				\hline
				Typ & Vy  \\
				\hline
				Syfte & En säljare ska kunna genomföra ett kontantköp och lämna ut biljetter (SR1.5, SR1.6, SR1.25)  \\
				\hline
				Funktion & Visa en översikt på säljarens möjliga sysslor.  \\
				\hline
				Delkomponenter & Inga  \\
				\hline
				Beroenden & SalesController  \\
				\hline
				Gränssnitt & Ej applicerbart  \\
				\hline
				Resurser & 5.L2 sales\_layout.html  \\
				\hline
				Källor & Inga  \\
				\hline
				Process & Ej applicerbart  \\
				\hline
			\end{tabular} \end{table} \FloatBarrier
			\vspace{6mm}

			|Data|@step - steg i bokning

			@steps - array med vilka steg som finns

			@reservation = session[:reservation] - bokningsdata|

			\paragraph{5.V40 SalesController\#show\_choice.html}\

			\begin {table} [ht] \begin{tabular} {  p{3.5cm} p{9.6cm} }
				\hline
				Typ & Vy  \\
				\hline
				Syfte & En säljare ska kunna genomföra ett kontantköp och lämna ut biljetter (SR1.5, SR1.6, SR1.25, SR1.21)  \\
				\hline
				Funktion & Välja vilken föreställning som ärendet gäller för en vald omgång som är aktiv.  \\
				\hline
				Delkomponenter & Inga  \\
				\hline
				Beroenden & SalesController  \\
				\hline
				Gränssnitt & Ej applicerbart  \\
				\hline
				Resurser & 5.L2 sales\_layout.html  \\
				\hline
				Källor & Inga  \\
				\hline
				Process & Ej applicerbart  \\
				\hline
			\end{tabular} \end{table} \FloatBarrier
			\vspace{6mm}

			|Data|@step - steg i bokning

			@steps - array med vilka steg som finns

			@reservation = session[:reservation] - bokningsdata|

			\paragraph{5.V41 SalesController\#seats.html}\

			\begin {table} [ht] \begin{tabular} {  p{3.5cm} p{9.6cm} }
				\hline
				Typ & Vy  \\
				\hline
				Syfte & En säljare ska kunna genomföra ett kontantköp och lämna ut biljetter (SR1.5, SR1.24 i plus, SR1.25)  \\
				\hline
				Funktion & Välja vilken plats som ärendet gäller.  \\
				\hline
				Delkomponenter & Inga  \\
				\hline
				Beroenden & Om administratören inte har placerat ut den redan bokade platsen så gör säljaren det, men sektionsvalet är redan låst. Om det är ett nytt köp, placerar säljaren också ut platsen. SalesController  \\
				\hline
				Gränssnitt & Ej applicerbart  \\
				\hline
				Resurser & 5.L2 sales\_layout.html  \\
				\hline
				Källor & Inga  \\
				\hline
				Process & Ej applicerbart  \\
				\hline
			\end{tabular} \end{table} \FloatBarrier
			\vspace{6mm}

			|Data|@step - steg i bokning

			@steps - array med vilka steg som finns

			@reservation = session[:reservation] - bokningsdata|

			\paragraph{5.V42 SalesController\#confirm.html}\

			\begin {table} [ht] \begin{tabular} {  p{3.5cm} p{9.6cm} }
				\hline
				Typ & Vy  \\
				\hline
				Syfte & En säljare ska kunna genomföra ett kontantköp och lämna ut biljetter (SR1.5, SR1.6, SR1.26)  \\
				\hline
				Funktion & Fråga användaren om de angivna uppgifterna är korrekta och skall införas i systemet.  \\
				\hline
				Delkomponenter & Inga  \\
				\hline
				Beroenden & SalesController  \\
				\hline
				Gränssnitt & Ej applicerbart  \\
				\hline
				Resurser & 5.L2 sales\_layout.html  \\
				\hline
				Källor & Inga  \\
				\hline
				Process & Ej applicerbart  \\
				\hline
			\end{tabular} \end{table} \FloatBarrier
			\vspace{6mm}

			|Data|@step - steg i bokning

			@steps - array med vilka steg som finns

			@reservation = session[:reservation] - bokningsdata|

			\paragraph{5.V43 SalesController\#retrieve.html}\

			\begin {table} [ht] \begin{tabular} {  p{3.5cm} p{9.6cm} }
				\hline
				Typ & Vy  \\
				\hline
				Syfte & En säljare ska kunna genomföra ett kontantköp och lämna ut biljetter (SR1.5, SR1.6)  \\
				\hline
				Funktion & Uppmana säljaren att gå och hämta en viss biljett.  \\
				\hline
				Delkomponenter & Inga  \\
				\hline
				Beroenden & SalesController  \\
				\hline
				Gränssnitt & Ej applicerbart  \\
				\hline
				Resurser & 5.L2 sales\_layout.html  \\
				\hline
				Källor & Inga  \\
				\hline
				Process & Ej applicerbart  \\
				\hline
			\end{tabular} \end{table} \FloatBarrier
			\vspace{6mm}

			|Data|@step - steg i bokning

			@steps - array med vilka steg som finns

			@reservation = session[:reservation] - bokningsdata|

			\paragraph{5.V44 SalesController\#payment.html}\

			\begin {table} [ht] \begin{tabular} {  p{3.5cm} p{9.6cm} }
				\hline
				Typ & Vy  \\
				\hline
				Syfte & En säljare ska kunna genomföra ett kontantköp och lämna ut biljetter (SR1.5, SR1.6)  \\
				\hline
				Funktion & Uppmana säljaren att ta betalt för bokningen.  \\
				\hline
				Delkomponenter & Inga  \\
				\hline
				Beroenden & SalesController  \\
				\hline
				Gränssnitt & Ej applicerbart  \\
				\hline
				Resurser & 5.L2 sales\_layout.html  \\
				\hline
				Källor & Inga  \\
				\hline
				Process & Ej applicerbart  \\
				\hline
			\end{tabular} \end{table} \FloatBarrier
			\vspace{6mm}

			|Data|@step - steg i bokning

			@steps - array med vilka steg som finns

			@reservation = session[:reservation] - bokningsdata|

			\paragraph{5.V45 SalesController\#hand\_out.html}\

			\begin {table} [ht] \begin{tabular} {  p{3.5cm} p{9.6cm} }
				\hline
				Typ & Vy  \\
				\hline
				Syfte & En säljare ska kunna genomföra ett kontantköp och lämna ut biljetter (SR1.5, SR1.6)  \\
				\hline
				Funktion & Uppmana säljaren att lämna ut biljetten.  \\
				\hline
				Delkomponenter & Inga  \\
				\hline
				Beroenden & SalesController  \\
				\hline
				Gränssnitt & Ej applicerbart  \\
				\hline
				Resurser & 5.L2 sales\_layout.html  \\
				\hline
				Källor & Inga  \\
				\hline
				Process & Ej applicerbart  \\
				\hline
			\end{tabular} \end{table} \FloatBarrier
			\vspace{6mm}

			|Data|@step - steg i bokning

			@steps - array med vilka steg som finns

			@reservation = session[:reservation] - bokningsdata|

			\paragraph{5.V46 SalesController\#search.html}\

			\begin {table} [ht] \begin{tabular} {  p{3.5cm} p{9.6cm} }
				\hline
				Typ & Vy  \\
				\hline
				Syfte & En säljare ska kunna genomföra ett kontantköp och lämna ut biljetter (SR1.5, SR1.6, SR1.25, SR1.20)  \\
				\hline
				Funktion & Låter säljaren söka bland bokningar med exempelvis namn.  \\
				\hline
				Delkomponenter & Inga  \\
				\hline
				Beroenden & SalesController  \\
				\hline
				Gränssnitt & Ej applicerbart  \\
				\hline
				Resurser & 5.L2 sales\_layout.html  \\
				\hline
				Källor & Inga  \\
				\hline
				Process & Ej applicerbart  \\
				\hline
			\end{tabular} \end{table} \FloatBarrier
			\vspace{6mm}

			|Data|@step - steg i bokning

			@steps - array med vilka steg som finns

			@reservation = session[:reservation] - bokningsdata|

		\subsubsection{Kund}



			\paragraph{5.V47 BookingController\#show\_choice.html}\

			\begin {table} [ht] \begin{tabular} {  p{3.5cm} p{9.6cm} }
				\hline
				Typ & Vy  \\
				\hline
				Syfte & Kunden ska kunna boka sin biljett via Kårspexets webbsida (SR1.2, SR1.21)  \\
				\hline
				Funktion & Kunden får ett val om vilken föreställning i den nuvarande omgång som denne vill gå på.  \\
				\hline
				Delkomponenter & Inga  \\
				\hline
				Beroenden & BookingController  \\
				\hline
				Gränssnitt & Ej applicerbart  \\
				\hline
				Resurser & 5.L4 customer\_layout.html   \\
				\hline
				Källor & Inga  \\
				\hline
				Process & Ej applicerbart  \\
				\hline
			\end{tabular} \end{table} \FloatBarrier
			\vspace{6mm}

			|Data|@step - steg i bokning

			@steps - array med vilka steg som finns

			@reservation - bokningsdata|

			\paragraph{5.V48 BookingController\#section.html}\

			\begin {table} [ht] \begin{tabular} {  p{3.5cm} p{9.6cm} }
				\hline
				Typ & Vy  \\
				\hline
				Syfte & Kunden ska kunna boka sin biljett via Kårspexets webbsida (SR1.2, SR1.24 i  plus)  \\
				\hline
				Funktion & Kunden väljer vilken sektion som denne vill boka på den valda föreställningen.  \\
				\hline
				Delkomponenter & Inga  \\
				\hline
				Beroenden & BookingController  \\
				\hline
				Gränssnitt & Ej applicerbart  \\
				\hline
				Resurser & 5.L4 customer\_layout.html  \\
				\hline
				Källor & Inga  \\
				\hline
				Process & Ej applicerbart  \\
				\hline
			\end{tabular} \end{table} \FloatBarrier
			\vspace{6mm}

			|Data|@step - steg i bokning

			@steps - array med vilka steg som finns

			@reservation - bokningsdata|

			\paragraph{5.V49 BookingController\#payment.html}\

			\begin {table} [ht] \begin{tabular} {  p{3.5cm} p{9.6cm} }
				\hline
				Typ & Vy  \\
				\hline
				Syfte & Kunden ska kunna boka sin biljett via Kårspexets webbsida (SR1.2, SR1.4 i delux)  \\
				\hline
				Funktion & Kunden fyller i sin betalningsinformation och hur kunden vill hämta sin biljett.  \\
				\hline
				Delkomponenter & Inga  \\
				\hline
				Beroenden & BookingController  \\
				\hline
				Gränssnitt & Ej applicerbart  \\
				\hline
				Resurser & 5.L4 customer\_layout.html  \\
				\hline
				Källor & Inga  \\
				\hline
				Process & Ej applicerbart  \\
				\hline
			\end{tabular} \end{table} \FloatBarrier
			\vspace{6mm}

			|Data|@step - steg i bokning

			@steps - array med vilka steg som finns

			@reservation - bokningsdata|

			\paragraph{5.V50 BookingController\#confirm.html}\

			\begin {table} [ht] \begin{tabular} {  p{3.5cm} p{9.6cm} }
				\hline
				Typ & Vy  \\
				\hline
				Syfte & Kunden ska kunna boka sin biljett via Kårspexets webbsida (SR1.2, SR1.26)  \\
				\hline
				Funktion & Kunden bekräftar bokningen och alla betalningsuppgifter.  \\
				\hline
				Delkomponenter & Inga  \\
				\hline
				Beroenden & BookingController  \\
				\hline
				Gränssnitt & Ej applicerbart  \\
				\hline
				Resurser & 5.L4 customer\_layout.html   \\
				\hline
				Källor & Inga  \\
				\hline
				Process & Ej applicerbart  \\
				\hline
			\end{tabular} \end{table} \FloatBarrier
			\vspace{6mm}

			|Data|@step - steg i bokning

			@steps - array med vilka steg som finns

			@reservation - bokningsdata|

			\paragraph{5.V51 BookingController\#cancel\_booking.html}\

			\begin {table} [ht] \begin{tabular} {  p{3.5cm} p{9.6cm} }
				\hline
				Typ & Vy  \\
				\hline
				Syfte & Kunden ska kunna avboka sin biljett via Kårspexets webbsida (SR1.2, SR1.3)  \\
				\hline
				Funktion & Kunden avbokar sin bokning via en länk i sitt bekräftelsemail.  \\
				\hline
				Delkomponenter & Inga  \\
				\hline
				Beroenden & BookingController  \\
				\hline
				Gränssnitt & Ej applicerbart  \\
				\hline
				Resurser & Inga  \\
				\hline
				Källor & Inga  \\
				\hline
				Process & Ej applicerbart  \\
				\hline
			\end{tabular} \end{table} \FloatBarrier
			\vspace{6mm}

			|Data|@step - steg i bokning

			@steps - array med vilka steg som finns

			@reservation - bokningsdata|

			\paragraph{5.V52 BookingController\#destroy.html}\

			\begin {table} [ht] \begin{tabular} {  p{3.5cm} p{9.6cm} }
				\hline
				Typ & Vy  \\
				\hline
				Syfte & Kunden ska kunna avboka sin biljett via Kårspexets webbsida (SR1.2, SR1.3)  \\
				\hline
				Funktion & Kunden terminerar bokningen från databasen.  \\
				\hline
				Delkomponenter & Inga  \\
				\hline
				Beroenden & BookingController  \\
				\hline
				Gränssnitt & Ej applicerbart  \\
				\hline
				Resurser & 5.L4 customer\_layout.html  \\
				\hline
				Källor & Inga  \\
				\hline
				Process & Ej applicerbart  \\
				\hline
			\end{tabular} \end{table} \FloatBarrier
			\vspace{6mm}

			|Data|@step - steg i bokning

			@steps - array med vilka steg som finns

			@reservation - bokningsdata|

\clearpage
\section{Övriga komponenter}



			\paragraph{5.X1 InheritedResources}\

			\begin {table} [ht] \begin{tabular} {  p{3.5cm} p{9.6cm} }
				\hline
				Typ & Övrig komponent (Rails-gem)  \\
				\hline
				Syfte & Abstraherar bort hantering av resurser för kontrollers som hanterar dessa för att minimera kodupprepning.  \\
				\hline
				Funktion & Tillhandahåller alla de (eller valfria) standardactions för kontroller som hanterar resurser och ärver från denna klass.  \\
				\hline
				Delkomponenter & Inga  \\
				\hline
				Beroenden & Inga  \\
				\hline
			\end{tabular} \end{table} \FloatBarrier
			\vspace{6mm}

			|Gränssnitt|Alla kontrollers som ärver från denna klass får standardactions för resurser (@index@, @show@, @new@, @create@, @edit@, @update@ samt @destroy@) fördefinierade om inget annat anges.

			@actions@ - Hjälpmetod för att specificera vilka actions som ska fördefinieras.

			Hjälpmetoder som görs tillgängliga i vyer:

			@resource@ - Enskild instans av den representerade resursen (i @show@, @new@, @create@, @edit@, @update@, @destroy@).

			@collection@ - En array innehållandes alla objekt av resursen (endast i @index@).

			@resource\_class@ - En referens till den modell (klassdefinitionen) som hanteras som resurs.|

			\begin {table} [ht] \begin{tabular} {  p{3.5cm} p{9.6cm} }
				\hline
				Resurser & InheritedResources::Base (ingår i Rails-gemet inherited\_resources).  \\
				\hline
				Källor & inherited\_resources: https://github.com/josevalim/inherited\_resources  \\
				\hline
				Process & Inga  \\
				\hline
				Data & Instansvariabler av formatet @modellnamn (ex. @reservation för ReservationsController) respektive @modellnamn\_i\_plural för index-actions (ex. @reservations).  \\
				\hline
			\end{tabular} \end{table} \FloatBarrier


			\paragraph{5.X2 AuthLogic}\

			\begin {table} [ht] \begin{tabular} {  p{3.5cm} p{9.6cm} }
				\hline
				Typ & Övrig komponent (Rails-gem)  \\
				\hline
				Syfte & Hanterar autentisering och användarsessioner (SR1.1).  \\
				\hline
				Funktion & Tillhandahåller funktioner för att autentisera användare utifrån existerande modeller.  \\
				\hline
				Delkomponenter & UserSession.  \\
				\hline
				Beroenden & Inga  \\
				\hline
			\end{tabular} \end{table} \FloatBarrier
			\vspace{6mm}

			|Gränssnitt|Authlogic::Session::Base\#find - returnerar aktiv användarsession eller försöker initiera en ny om ingen redan finns

			ActiveRecord::Base\#acts\_as\_authentic - instruerar Authlogic att den aktuella modellen representerar en typ av autentisering|

			\begin {table} [ht] \begin{tabular} {  p{3.5cm} p{9.6cm} }
				\hline
				Resurser & Authlogic (ingår i Rails-gemet authlogic), UserSession och User.  \\
				\hline
				Källor & authlogic: https://github.com/binarylogic/authlogic  \\
				\hline
				Process & Inga  \\
				\hline
				Data & Inga  \\
				\hline
			\end{tabular} \end{table} \FloatBarrier


			\paragraph{5.X3 CanCan}\

			\begin {table} [ht] \begin{tabular} {  p{3.5cm} p{9.6cm} }
				\hline
				Typ & Övrig komponent (Rails-gem)  \\
				\hline
				Syfte & Definierar och hanterar användarroller (SR9.2).  \\
				\hline
				Funktion & Tillhandahåller metoder för att bestämma vilka användarroller som finns samt vad dessa har rättigheter till. Definierar även hjälpmetoder för att testa rättigheter mot den inloggade användaren.  \\
				\hline
				Delkomponenter & Ability  \\
				\hline
				Beroenden & En metod @ApplicationController\#current\_user@ måste definieras så att den returnerar ett objekt.  \\
				\hline
			\end{tabular} \end{table} \FloatBarrier
			\vspace{6mm}

			|Gränssnitt|@can?@ - returnerar huruvida användaren har rättigheter för att utföra en viss handling (tillgänglig i vyer och kontroller)

			@cannot?@ - inversen av @can?@

			@ApplicationController\#authorize!@ - kastar en exception om användaren inte har rättighet att utföra den handling som angivits

			@ApplicationController\#load\_and\_authorize\_resource@ - ladda in och autorisera alla standardactions för resurser automatiskt|

			\begin {table} [ht] \begin{tabular} {  p{3.5cm} p{9.6cm} }
				\hline
				Resurser & CanCan (ingår i Rails-gemet cancan) och Ability.  \\
				\hline
				Källor & cancan: https://github.com/ryanb/cancan  \\
				\hline
			\end{tabular} \end{table} \FloatBarrier
			\vspace{6mm}

			|Process|1. Rättigheter fördefinieras i Ability-modellen.

			2. @can?(:action, Object)@ anropas (:action = någon fördefinierad action, Object är någon modell alternativt en modellinstans).

			3. CanCan hämtar aktiv användare via @ApplicationController\#current\_user@.

			4. Användarens rättigheter undersöks av CanCan genom att skicka denna till Ability-modellen.

			5. @can?@-metoden returnerar huruvida rättigheter finns (eller saknas) för att utföra handlingen.|

			\begin {table} [ht] \begin{tabular} {  p{3.5cm} p{9.6cm} }
				\hline
				Data & Inga  \\
				\hline
			\end{tabular} \end{table} \FloatBarrier


			\paragraph{5.X4 ActiveRecord}\

			\begin {table} [ht] \begin{tabular} {  p{3.5cm} p{9.6cm} }
				\hline
				Typ & Övrig komponent (Rails)  \\
				\hline
				Syfte & ActiveRecord är en existerande baskomponent i Rails-ramverket som bland annat fungerar som ett gränssnitt till databasens tabeller.   \\
				\hline
				Funktion & Tillhandahåller gränssnitt för sökningar och manipulation av objekt databasen. Implicit ingår databas och databasschema i varje modell.  \\
				\hline
				Delkomponenter &  Inga  \\
				\hline
			\end{tabular} \end{table} \FloatBarrier
			\vspace{6mm}

			|Gränssnitt|Objektorienterat gränssnitt enligt active-record-patternen. En modell är en klass som ärver från ActiveRecord::Base och motsvarar en tabell i databasen. En instans av klassen motsvarar en rad i tabellen.

			Följande är några av de metoder som definieras av @ActiveRecord::Base@ och är tillgängliga för alla modeller.

			@all@ - hämta alla instanser av modellen (alla rader i databastabellen).

			@find@ - gör diverse sökningar i databasen och returnerar de instanser som matchar.

			@valid?@ - kontrollera att objektets data är korrekt. Vilka kontroller som görs definieras av varje modell.

			@save@ - lagrar/uppdaterar en instans av ett modell.

			@update@ - uppdaterar attributer för en instans i databastabellen.

			@destroy@ - raderar objektet i databasen.

			Dessutom finns accessors för varje attribut hos modellen. Dessa definieras dynamiskt.|

			\begin {table} [ht] \begin{tabular} {  p{3.5cm} p{9.6cm} }
				\hline
				Beroenden & Inga  \\
				\hline
				Resurser & Inga  \\
				\hline
			\end{tabular} \end{table} \FloatBarrier
			\vspace{6mm}

			|Källor|ActiveRecord: \url{http://ar.rubyonrails.org|}

			|Process|Följande typscenario visar skapande och uppdatering av en instans av modellen @M@.

			<pre>m = M.new

			m.att1 = ``value''

			if m.valid?

			then m.save</pre>|

			|Data|Alla modeller har datafälten @created\_at@ respektive @updated\_at@ av typen @timestamp@. Dessa uppdateras automatiskt av Rails vid skapande och uppdatering av objektet.

			Alla modeller har även attributet @id@ som primärnyckel.

			För övriga attribut se respektive modell. Enkla attribut listas, relationer till andra modeller beskrivs i delkomponenter.|

\clearpage
\section{Genomförbarhet- och resursuppskattning}


\emph{6. Feasibility and Resource Estimates. Summarise the conclusions of a feasibility study}

\emph{around the architectural model. Prioritise all components with a priority model (e.g.}

\emph{economy, standard, deluxe versions).}

\emph{Identify and describe all future project tasks. Identify task dependencies in terms of}

\emph{commencement and completion, preferably with a task flow chart. Estimate the effort}

\emph{required for each project task. Produce a task allocation plan and schedule for each}

\emph{project staff member for the remainder of the project. This information should preferably}

\emph{be presented in a table.}

\emph{Identify possible risks going forward, and for each risk, give a risk management proposal.}

\emph{Estimate the overall schedule for making a detailed design, coding this design, testing}

\emph{and delivering the final product and documentation according to the project deadlines.}

\emph{Identify the critical path in the project, and analyze possible project slippage along this}

\emph{path.}

	\subsection{Sammanfattning av slutsatser av en genomförbarhetsundersökning av arkitekturmodellen.}


	Arkitekturen Nyx kommer använda kallas Model-View-Controller, MVC. Alla komponenter under respektive punkt finns angivna nedan.

	I designen ser vi endast en del som kommer att vara mer komplicerad, nämligen stolsplaceringen. Det är dock fullt genomförbart, eftersom det inte är avancerade tekniker som används, utan att det är mycket data som ska sammankopplas.

	Modelleringen är redan gjord och simuleringen borde inte vara några problem, vi kommer nämligen använda rätt arkitektur från början.

	Slutsatsen är således att projektet är fullt genomförbart med den här arkitekturmodellen.

	\subsection{Komponentprioritering}


	\emph{Alla felaktiga numreringar måste fixas till i punkt 5 innan det kan fixas här. Prioritet är inte säkert, saknas många referenser till SRD under syfte i punkt 5. Fixas på måndag!!!}

		\subsubsection{Modeller}


		\begin {table} [ht] \begin{tabular} {  p{3.5cm} p{9.6cm} }
			\hline
			 Komponent  &  Prioritet   \\
			\hline
			 5.2.1 Reservation  &  Standard   \\
			\hline
			 5.2.2 ReservationCounter  &  Standard   \\
			\hline
			 5.2.3 Placement  &  Standard   \\
			\hline
			 5.2.4 PlacementLock  &  Standard   \\
			\hline
			 5.5.5 Theater   &  Standard   \\
			\hline
			 5.n Section  &  Standard   \\
			\hline
			 5.n Seat  &  Standard   \\
			\hline
			 5.n.8 Batch  &  Standard   \\
			\hline
			 5.n.9 Show  &  Standard   \\
			\hline
			 5.n.10 Pricing  &  Standard   \\
			\hline
			 5.2.11 User  &  Standard   \\
			\hline
			 5.2.12 MailTemplate  &  Standard   \\
			\hline
			 5.2.13 Mail  &  Standard   \\
			\hline
			 5.2.14 UserSession  &  Standard   \\
			\hline
			 5.2.15 Ability  &  Standard   \\
			\hline
		\end{tabular} \end{table} \FloatBarrier


		\subsubsection{Kontroller}


		\begin {table} [ht] \begin{tabular} {  p{3.5cm} p{9.6cm} }
			\hline
			 Komponent  &  Prioritet   \\
			\hline
			 5.C1 ApplicationController  &  Standard   \\
			\hline
			 5.n BookingController  &  Standard   \\
			\hline
			 5.n SessionController  &  Standard   \\
			\hline
			 5.n AdminIndexController  &  Standard   \\
			\hline
			 5.n ReservationsController  &  Standard   \\
			\hline
			 5.n PlacementsController  &  Standard   \\
			\hline
			 5.n TheatersController  &  Standard   \\
			\hline
			 5.n SectionsController  &  Standard   \\
			\hline
			 5.n SeatsController  &  Standard   \\
			\hline
			 5.n BatchesController  &  Standard   \\
			\hline
			 5.n ShowsController  &  Standard   \\
			\hline
			 5.n PricingsController  &  Standard   \\
			\hline
			 5.n UsersController  &  Standard   \\
			\hline
			 5.n MailingController  &  Standard   \\
			\hline
			 5.n SalesController  &  Standard   \\
			\hline
		\end{tabular} \end{table} \FloatBarrier


		\subsubsection{Vyer}


		\begin {table} [ht] \begin{tabular} {  p{3.5cm} p{9.6cm} }
			\hline
			 Komponent  &  Prioritet   \\
			\hline
			 5.L1 AdminLayout  &  Standard   \\
			\hline
			 5.L2 SalesLayout  &  Standard   \\
			\hline
			 5.L3 SimpleLayout  &  Standard   \\
			\hline
			 5.L4 CustomerLayout  &  Standard   \\
			\hline
			 5.V1 Session\#new.html  &  Standard   \\
			\hline
			 5.V2 AdminIndexController\#index.html  &  Standard   \\
			\hline
			 5.V3 AdminIndexController\#stat\_standard.html  &  Standard   \\
			\hline
			 5.V4 AdminIndexController\#stat\_plus.html  &  Plus   \\
			\hline
			 5.V5 AdminIndexController\#stat\_deluxe.html  &  Deluxe   \\
			\hline
			 5.V6 ReservationController\#index.html  &  Standard   \\
			\hline
			 5.V7 ReservationController\#new.html  &  Standard   \\
			\hline
			 5.V8 ReservationController\#show.html  &  Standard   \\
			\hline
			 5.V9 ReservationController\#edit.html  &  Standard   \\
			\hline
			 5.P1 ReservationController\#\_form  &  Standard   \\
			\hline
			 5.V10 PlacementController\#index.html  &  Standard   \\
			\hline
			 5.V11 PlacementController\#new.html  &  Standard   \\
			\hline
			 5.V12 PlacementController\#show.html  &  Standard   \\
			\hline
			 5.V13 PlacementController\#edit.html  &  Standard   \\
			\hline
			 5.P2 PlacementController\#\_form  &  Standard   \\
			\hline
			 5.V14 TheatersController\#index.html  &  Standard   \\
			\hline
			 5.V15 TheatersController\#new.html  &  Standard   \\
			\hline
			 5.V16 TheatersController\#show.html  &  Standard   \\
			\hline
			 5.V17 TheatersController\#edit.html  &  Standard   \\
			\hline
			 5.P3 TheatersController\#\_form  &  Standard   \\
			\hline
			 5.3.22 SectionController\#index.html  &  Standard   \\
			\hline
			 5.3.23 SectionController\#new.html  &  Standard   \\
			\hline
			 5.3.24 SectionController\#show.html  &  Standard   \\
			\hline
			 5.3.25 SectionController\#edit.html  &  Standard   \\
			\hline
			 5.3.26 SectionController\#\_form  &  Standard   \\
			\hline
			 5.3.27 SeatsController\#index.html  &  Standard   \\
			\hline
			 5.3.28 SeatsController\#new.html  &  Standard   \\
			\hline
			 5.3.29 SeatsController\#show.html  &  Standard   \\
			\hline
			 5.3.30 SeatsController\#edit.html  &  Standard   \\
			\hline
			 5.3.31 SeatsController\#\_form  &  Standard   \\
			\hline
			 5.3.32 BatchesController\#index.html  &  Standard   \\
			\hline
			 5.3.33 BatchesController\#new.html  &  Standard   \\
			\hline
			 5.3.34 BatchesController\#show.html  &  Standard   \\
			\hline
			 5.3.35 BatchesController\#edit.html  &  Standard   \\
			\hline
			 5.3.36 BatchesController\#\_form  &  Standard   \\
			\hline
			 5.3.37 ShowsController\#index.html  &  Standard   \\
			\hline
			 5.3.38 ShowsController\#new.html  &  Standard   \\
			\hline
			 5.3.39 ShowsController\#show.html  &  Standard   \\
			\hline
			 5.3.40 ShowsController\#edit.html  &  Standard   \\
			\hline
			 5.3.41 ShowsController\#\_form  &  Standard   \\
			\hline
			 5.3.42 PricingsController\#index.html  &  Standard   \\
			\hline
			 5.3.43 PricingsController\#new.html  &  Standard   \\
			\hline
			 5.3.44 PricingsController\#show.html  &  Standard   \\
			\hline
			 5.3.45 PricingsController\#edit.html  &  Standard   \\
			\hline
			 5.3.46 PricingsController\#\_form  &  Standard   \\
			\hline
			 5.3.47 UsersController\#index.html  &  Standard   \\
			\hline
			 5.3.48 UsersController\#new.html  &  Standard   \\
			\hline
			 5.3.49 UsersController\#show.html  &  Standard   \\
			\hline
			 5.3.50 UsersController\#edit.html  &  Standard   \\
			\hline
			 5.3.51 UsersController\#\_form  &  Standard   \\
			\hline
			 5.3.52 MailingController\#index.html  &  Standard   \\
			\hline
			 5.3.53 MailingController\#new.html  &  Standard   \\
			\hline
			 5.3.54 MailingController\#show.html  &  Standard   \\
			\hline
			 5.3.55 MailingController\#edit.html  &  Standard   \\
			\hline
			 5.3.56 MailingController\#\_form  &  Standard   \\
			\hline
			 5.3.58 SalesController\#index.html  &  Standard   \\
			\hline
			 5.3.59 SalesController\#choice.html  &  Standard   \\
			\hline
			 5.3.60 SalesController\#show.html  &  Standard   \\
			\hline
			 5.3.61 SalesController\#seats.html  &  Standard   \\
			\hline
			 5.3.61 SalesController\#confirm.html  &  Standard   \\
			\hline
			 5.3.62 SalesController\#tickets.html  &  Standard   \\
			\hline
			 5.3.63 SalesController\#payment.html  &  Standard   \\
			\hline
			 5.3.64 SalesController\#hand\_out.html  &  Standard   \\
			\hline
			 5.3.65 SalesController\#search.html  &  Standard   \\
			\hline
			 5.3.67 BookingController\#show.html  &  Standard   \\
			\hline
			 5.3.68 BookingController\#section.html  &  Standard   \\
			\hline
			 5.3.69 BookingController\#payment.html  &  Standard   \\
			\hline
			 5.3.70 BookingController\#confirm.html  &  Standard   \\
			\hline
			 5.3.71 BookingController\#cancel\_booking.html  &  Standard   \\
			\hline
			 5.3.72 BookingController\#destroy.html  &  Standard   \\
			\hline
		\end{tabular} \end{table} \FloatBarrier


		\subsubsection{Övriga komponenter}


		\begin {table} [ht] \begin{tabular} {  p{3.5cm} p{9.6cm} }
			\hline
			 Komponent  &  Prioritet   \\
			\hline
			 5.K1 InheritedResources &  Standard   \\
			\hline
			 5.K2 AuthLogic  &  Standard   \\
			\hline
			 5.K3 CanCan &  Standard   \\
			\hline
		\end{tabular} \end{table} \FloatBarrier


	\subsection{Framtida projektuppgifter}


	>> Utveckla hemsidan, det vill säga frontend .(?? Borde översättas??)

	>> Utveckla det bakomliggande systemet, backend.

	>> Verifiera funktioner genom testning. Dessa tester behöver utvecklas.

	>> Förbereda ett demo, produkten ska demonstreras och detta kräver förberedelser.

	>> Bestämma designen på hemsidan, den kommer användas av många och bör vara tilltalande samt passa in i Kårspexets egen design.

	\subsection{Beroenden mellan uppgifter (?? ska tasks översättas till uppgifter??)}


	Vad gäller komponenterna och då speciellt vyer kan man se beroenden mellan vy och kontroller på följande sätt; AdminIndexController\#new.html har AdminIndexController som beroende.

	\subsection{Tidsuppskattning för uppgifter}



	\subsection{Uppgiftsplanering}


	\begin {table} [ht] \begin{tabular} {  p{4cm} p{2cm} p{2cm} p{2cm} p{2cm} p{2cm} p{2cm} p{2cm} p{2cm} p{2cm} p{2cm} }
		\hline
		 Namn  &  Vecka 1 &  Vecka 2 &  Vecka 3 &  Vecka 4 &  Vecka 5 &  Vecka 6 &  Vecka 7 &  Vecka 8 &  Vecka 9 &  Vecka 10   \\
		\hline
		 Projektmedlem 1 \emph{mögel}  &  nått  &  nått  &  &  &  &  &  &  &  &  Testar   \\
		\hline
		 Projektmedlem 2 \emph{tarandi} &  nått  &  nått  &  &  &  &  &  &  &  &  Testar   \\
		\hline
		 Projektmedlem 3 \emph{perka} &  nått  &  nått  &  &  &  &  &  &  &  &  Testar   \\
		\hline
		 Projektmedlem 4 \emph{kalle} &  nått  &  nått  &  &  &  &  &  &  &  &  Testar   \\
		\hline
		 Projektmedlem 5 \emph{erik} &  nått  &  nått  &  &  &  &  &  &  &  &  Testar   \\
		\hline
		 Projektmedlem 6 \emph{daniel} &  nått  &  nått  &  &  &  &  &  &  &  &  Testar   \\
		\hline
		 Projektmedlem 7 \emph{andré} &  nått  &  nått  &  &  &  &  &  &  &  &  Testar   \\
		\hline
		 Projektmedlem 8 \emph{lemming} &  nått  &  nått  &  &  &  &  &  &  &  &  Testar   \\
		\hline
		 Projektmedlem 9 \emph{johan} &  nått  &  nått  &  &  &  &  &  &  &  &  Testar   \\
		\hline
		 Projektmedlem 10 \emph{rasmus} &  nått  &  nått  &  &  &  &  &  &  &  &  Testar   \\
		\hline
		 Projektmedlem 11 \emph{mia} &  nått  &  nått  &  &  &  &  &  &  &  &  Testar   \\
		\hline
	\end{tabular} \end{table} \FloatBarrier


	\subsection{Möjliga risker}


	Projektplaneringen visar att tiden kommer att räcka för att slutföra projektet och hittills har samarbetet fungerat väl. Vi har ingen nyckelkompetens som gör oss beroende av en enstaka projektmedlem. Den största risken ligger i att Kårspexet inte anser kraven uppfyllda eller kommer med ändringar i kraven. Då kommer det vara ont om tid för att korrigera dessa och det ger oss ingen marginal för fel. Vi kommer undvika detta genom ett nära samarbete med Kårspexet under implementationen. De kommer dessutom att ha en viktig roll under testningen som skall inledas i god tid. 

	Risk att vi implementerar fel funktionalitet (annat än det efterfrågade) och därmed förlorar tid.

	Risk att vi inte hinner implementera alla funktioner som efterfrågas

	Risk att vi implementerar saker i fel ordning så att vi förlorar tid på att vänta på varandra. Här kommer nyckelkompetens in i beräkningarna.

	Risk att vår utvecklingsserver krånglar.

	Risk att det krånglar när vi ska installera vårt system på kårspexets server.

\clearpage
\section{Spårningsmatris mellan mjukvarukrav och strukturella krav}


\emph{7. Software Requirements vs Components Traceability Matrix. Gives a table cross}

\emph{referencing architectural components (based on defined component identifers) to}

\emph{software requirements numbered in the SRD.}

\clearpage
	\appendix

\end{document}

