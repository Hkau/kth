\documentclass[a4paper,10pt,twoside]{article}
% includes (fold)
\usepackage[inner=3cm,top=3cm,outer=2cm,bottom=3cm]{geometry}
%\usepackage[swedish]{babel}
\usepackage[T1]{fontenc}
\usepackage{moreverb}
\usepackage{amssymb}
\usepackage{fancyhdr}
\usepackage{fancyvrb}
\usepackage{tipa}
\usepackage{listings}
\lstset{
  %language=plain, 
  fancyvrb=true, 
  tabsize=4,
  numbers=left,
  breaklines=true,
  basicstyle=\ttfamily
}
% \usepackage{algorithmic}
% \usepackage{algorithm}
\usepackage{alg}
\usepackage{amsmath}
\usepackage{graphicx}
\usepackage{color}
\definecolor{dark-blue}{rgb}{0, 0, 0.6}
\usepackage{hyperref}
\hypersetup{
  colorlinks=true, 
  linkcolor=dark-blue,
  urlcolor=dark-blue
}
\usepackage{tikz}
\usepgfmodule{shapes}
\usepackage{wrapfig}
% includes (end)

% defines (fold)
\def\names{Peter Boström} % ...och namn på alla andra som skrivit...
\def\theauthor{Peter Boström \\ pbos@kth.se}
\def\homeworknumber{Mästarprov 1} % fyll i vilket kursmoment det handlar om här
\def\coursename{ADK}
\def\course{DD1352}

\title{\homeworknumber\ - \course\ \coursename}
\date{2010-03-01} % TODO
\author{\theauthor}
% defines (end)

\begin{document}
% fancyheaders (fold)
\fancypagestyle{plain}{
  \headheight 13pt
  \fancyfoot{}
  \lhead{\course\ -- \names}
  \rhead{\homeworknumber}
  \fancyfoot[LE,RO]{\thepage}
}
\pagestyle{plain}
\fancypagestyle{empty}{
  \fancyhead{}
  \fancyfoot{}
}
% fancyheaders (end) 

\maketitle % skapa titelsida
\thispagestyle{empty}
\newpage

\thispagestyle{empty}
\tableofcontents % innehållsförteckning
\newpage
\setcounter{page}{1}

\lstset{
	numbers=left
}

% Skriv här
\section{Kommunikationsföde} % ny sektion

\subsection{Problem}

Datorer ska anslutas i ett nätverk. Varje dator kan vidareförmedla trafik, och alla datorer ska kunna prata med varandra. Detta ska ske så billigt som möjligt.

Varje dator kan sammankopplas med hur många andra som helst. Två datorer sammankopplas d hjälp av en adapter i varje ände och sladd däremellan. Kostnaden för en adapter beror på vilken dator den ska stoppas i. Meterpriset på sladd är däremot konstant.

\subsection{Lösning}

Detta problem kan formuleras som ett grafproblem där varje nod är en dator, och kanterna mellan noder har en vikt som är lika med den kostnad som finns för att sammankoppla dessa.

För att alla datorer ska kunna prata med varandra så behöver vi skapa en sammanhängande graf med så liten total kostnad som möjligt. Detta är precis vad ett minimalt spännande träd är. Vi skapar först en komplett graf med alla kantvikter som kostnader. Sedan använder vi Kruskals algoritm för att skapa det minimalt spännande trädet i grafen.

\subsection{Indata och kostnad för anslutning}

$n$ st. arenor $a_1,...,a_n$.

$d_{i,j}$ ger avståndet mellan arenor $a_i$ och $a_j$ i meter.

$P$ dollar per meter optisk kabel.

$A_i$ är kostnaden för att sätta fast en adapter i arena $a_i$.

Kostnad för anslutning mellan $a_i$ och $a_j$ är: $$A_i+A_j+P*d_{i,j}$$

\subsection{Pseudokod}

\begin{lstlisting}
function Kostnad(n, d, P, A) =
POST = kostnad för billigaste nätverket av dessa n st. arenor
	PRE = d(i, j) innehåller avståndet mellan arena i och j, P meterkostnad för kabel, A(i) ger adapterkostnad för arena i.
	Skapa en kantmängd E = {}
	for i = 1...n
		for j = 1...n
			if i != j
				E = ( i, j, A(i) + A(j) + P * d(i, j) )    # kant mellan i,j med kostnaden mellan dem

	ASSERT = E innehåller alla möjliga kanter mellan noderna a, med korrekta vikter.
	Sortera E efter kostnad

	Skapa en komponentmängd C = {1...n}
	Skapa totalkostnad Tk = 0

	while E != {}
		INV = C innehåller de största komponenter som går att bilda med tidigare förkastade eller använda noder, Tk är billigaste kostnaden för att bilda C
		Plocka ut minsta kanten ( i, j, k ) ur E

		if C(i) != C(j)
			Unifiera C(i) och C(j)
			Tk = Tk + k    # uppdatera totalkostnaden för att bygga C

	ASSERT = C är en sammanhängande graf, Tk är billigaste kostnaden för att sammankoppla denna

	return totalkontnad Tk
\end{lstlisting}

\subsection{Korrekthet}

Först skapas kanter för den kompletta grafen. För varje nod så går vi igenom alla andra noder och skapar en kant till dem. Dessa kanter får kostnaden $A_i$ + $A_j$ + $P$ * $d_{i,j}$ vilket är korrekta kostnaden för att koppla samman dem, förutsatt att PRE(3) stämmer. Eftersom vi ansluter varje nod till varje annan nod så kommer den att bilda en komplett graf. Alltså stämmer ASSERT(10).

INV(16) gäller från början då inga noder är sammankopplade och totalkostnaden för detta är 0.

Under en iteration av loopen (17-21) så kommer den billigaste ickegenomgångna kanten att läggas till i lösningen endast ifall den sammankopplar två olika komponenter. Eftersom noderna inte redan är sammankopplade måste denna kant vara billigaste sättet att sammankoppla dessa komponenter. Komponenterna unifieras, totalkostnaden ökar med den kostnad som kanten tar att bygga. Eftersom en kant endast tillförs om den sammanfogar komponenter så kommer inga (onödiga) cykler bildas. Tk och C uppdateras alltså korrekt och INV(16) fortsätter gälla för det allmäna fallet.

Denna loop avslutar när det inte längre finns kanter att testa. Eftersom E före loopen enligt ASSERT(10) innehåller kanter som sammankopplar alla noder, så kommer INV(16) leda till att alla komponenter slutligen blir sammanfogade till en när dessa tagit slut. INV(16) säger också att totalkostnaden uppdateras rätt under tiden och kommer även stämma i slutet. ASSERT(23) stämmer alltså även. 

POST(2) stämmer även då ASSERT(23) säger att totalkostnaden Tk är korrekt i slutet och denna sedan returneras.

\subsection{Tidskomplexitet}

Att skapa kanter för den kompletta grafen är $O(n^2)$ då varje algoritmen vid varje nod går igenom alla andra för att sammankoppla dessa. Att lägga till denna kant i kantmängden går på konstant tid om E t.ex. är en redan skapad vektor med $n^2$ reserverade element.

Att sortera denna vektor tar $O(|E|*log(|E|)) = O(n^2*log(n))$ med t.ex. mergesort.

Sedan skapas en komponentmängd med en union-find-struktur. Att lägga in alla noder i denna tar $O(n)$, då varje sådan tilläggning är $O(1)$.

Loopen kommer att ske en gång per kant, då en kant plockas ut varje gång. Dvs $O(|E|)$ gånger. Inuti loopen finns endast konstanta operationer, förutom de som relaterar till C(i) och C(j). Dessa operationer är $O(log(n))$ i en union-find-struktur. Totalt är denna loop $O(n^2*log(n))$ i komplexitet.

Totalt blir algoritmen $O(n^2*log(n))$ då det är högst av alla delsteg.

\pagebreak
\section{Pillabyrint}

\subsection{Problem}

En pillabytrint är en n*n-matris med skyltar i varje korsning. Varje ställe är anslutet åt max fyra håll (upp, ner, vänster, höger). Skyltarna i varje korsning anger vilka vägar som får tas därifrån (Framåt, Vänster eller Höger). Vilken skylt som gäller beror på vilket håll man kom ifrån. Man börjar i punkten (sx, sy) varifrån man går upp och söker sig mot målet (mx, my). Problemet gäller att hitta kortaste vägen ut.

\subsection{Lösning}

Modifierad BFS där man istället för att markera noden som besökt markerar ingångshållet som besökt. Detta eftersom multipla besök av samma nod kan vara del av lösningsstigen.

BFS fungerar med hjälp av en kö, eftersom det måste gå att backtracka så måste referens till tidigare nod sparas. I slutet när målet ev. hittats så skriver man ut lösninen, blir kön tom så går det inte att nå målet.

\subsection{Indata}

$M$, en $n*n$-matris med skyltar för varje väg.

$(sx, sy)$ startpunkt.

$(mx, my)$ slutpunkt.

\subsection{Pseudokod}

\begin{lstlisting}
function Pilla(M, sx, sy, mx, my) =
POST = Skriver ut kortaste lösningen eller skriver ut meddelande om målet inte går att nå.
	PRE = M(i, j) innehåller skyltar som kan peka fram, vänster eller höger, (sx, sy) startpunkt, (mx, my) slutpunkt.

	Skapa kö Q = {}
	Skapa nod S för noden (sx, sy, UP, nil)    # vid start går man alltid uppåt! Har ingen förälder alltså nil.

	Lägg till S sist i Q.

	while Q != {}
		INV=Noden som plockas ur kön refererar till kortaste vägen till den. Alla grannar till tidigare vägar är antingen genomgångna eller ligger i kön. Dessa innehåller information om vilken nod som är del av kortaste stigen till dem. Varje besökt nod är markerad som besökt.
		plocka ur noden N: (nx, ny, dir, parent) ur kön Q.

		if N befinner sig på (mx, my)
			# Framme vid målet, skriv ut hur vi tog oss hit.
			printpath(N)
			return

		if dir == UP
			next = (nx, ny+1)
			next_sign = DOWN    # ifall man senast gick upp ska man kolla på skylten nedanför nästa plats
		elif dir == DOWN
			next = (nx, ny-1)
			next_sign = UP
		elif dir == LEFT
			next = (nx-1, ny)
			next_sign = RIGHT
		else
			next = (nx+1, ny)
			next_sign = LEFT

		for Varje håll M(next.x, next.y):s skylt från next_sign-hållet pekar åt
			Skapa en ny nod D med koordinaterna för next, skylt åt inverterat next_sign-håll och N som förälder.
			# inverterat next_sign-håll betyder UP->DOWN, LEFT->RIGHT och tvärt om.
			Lägg till D i kön Q.

		Riv skylten i M(next.x, next.y) från next_sign-hållet.   # samma som att markera som besökt

	ASSERT = Det fanns ingen väg till (mx, my) från (sx, sy)

	print "går inte att hitta"


function printpath(N)
POST = Skriver (rekursivt) ut vägen till N
	if N.parent == nil
		return N.dir

	# Skriv ut vägen till tidigare nod före sista
	olddir = printpath(N.parent)

	if olddir == N.dir
		print "F"

	elif N.dir ligger till höger om olddir # t.ex. N.dir == UP, olddir == LEFT
		print "H"

	else    # måste vara vänster
		print "V"

	return N.dir

\end{lstlisting}


\subsection{Korrekthet}

Noden som plockas ur kön första gången är startnoden utan förälder. Eftersom stigen börjar på den noden så är den själv enskilt kortaste vägen. Inga tidigare har gåtts igenom, alltså stämmer INV(11) för första iterationen.

Under loopen (14-37) så tas ett steg från den nod som togs ut ur kön i avsedd riktning (dir). Sedan kollar man på den skylt som man stöter på och lägger till alla dessa riktningar i kön. Detta motsvarar att hitta grannar till en nod i BFS. Sedan på rad 37 så "rivs" skylten i matrisen M. Detta motsvarar att markera noden som besökt. Men det är viktigt att det är skylten som markeras som besökt och inte noden. Varje nod som skapas är sagd att ha noden N som förälder, som i sin tur har sitt ursprung som förälder. INV(11) stämmer efter iteration också.

Eftersom kön markerar alla vägar som besökta när de passeras så kommer det slutligen ta slut på vägar att gå på. Ifall man når målet (rad 15) så kommer programmet att skriva ut vägen dit och sedan terminera. Förutsatt att printpath fungerar så kommer även hela funktionen att vara korrekt. Målet hittades med kortaste väg, och vägen skrevs korrekt ut.

printpath(N) fungerar så att den skriver ut alla noder som kommer före den man vill skriva ut via rekursivt anrop. med basfallet att den har kommit till första noden, då den returnerar nodens gångväg. Denna sparas sedan av anroparen som jämför den med sitt egna gånghåll. Med hjälp av dessa två jämförs om den senare relativt svänger höger eller vänster. Sedan returnerar även den sitt gånghåll för att fortsätta fungera rekursivt.

Andra alternativet är att loopen slutar. I detta fall har alla möjliga vägar att upptäcka lagts till i kön, och sedan har dessa tagit slut. Ifall detta händer så finns det inga vägar kvar att gå i grafen. Har inte målet nåtts då så finns det ingen väg till målet från startpositionen. Då kommer ett meddelande att skrivas ut. Båda fall stämmer, alltså stämmer POST(2).

\subsection{Tidskomplexitet}

Värstafall är att varje nod kommer gå igenom alla sina grannar. I denna labyrint, som är matrisbegränsad är antal grannar begränsade till 3 Men varje nod går att besöka upp till 4 gånger. Så $4 * |v|$ operationer á konstant tid. Detta är $O(v)$ vilket är $O(n^2)$

Printpath är dock inte konstant tid, men kommer endast köra en gång, och är $O(n^2)$ då varje rekursiv körning sker endast en gång på maximalt $O(v)$ noder.

\section{Världens högsta hiss}

Uttömmande grafsökning har jag lyckats få ner till $O(n*log(n)+n*m+(n+m)^k)$, men den är kraftigt stor, och varken $O(n^k)$ eller $O(m^k)$ känns att leka med. Problemet ska förmodligen lösas med dynamisk programmering, men jag vet inte, för varje lösning beror på tidigare val, man kan inte "välja" en i taget. Jag får inte runt det. :)

\end{document}

