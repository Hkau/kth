\documentclass[a4paper]{article}

\usepackage[utf8]{inputenc} % -- använd denna "när det funkar", dvs på skolans nya datorer + linux, ibland på windows
% \usepackage[latin1]{inputenc} % -- använd denna om överstående inte fungerar (åäö etc.)

\usepackage{fancyvrb}
\usepackage{listings}
\lstset{language=Java,
	numbers=left,
	numberstyle=\footnotesize,
	title=\lstname,
	showstringspaces=false,
	fancyvrb=true,
	extendedchars=true,
	breaklines=true,
	breakatwhitespace=true,
	tabsize=4}

\lstset{ % För att å, ä och ö ska funka inuti lstlistings, kan behövas för andra symboler också..
	literate={ö}{{\"o}}1
		{ä}{{\"a}}1
		{å}{{\aa}}1
	}

\author{Mattis Kancans Envall \\ mattiske@kth.se}
\title{DD1341 inda10 \\ Uppgift 10}

\begin{document}

	\maketitle
	\section{Förord}
Jag har valt följande obligatoriska implementeringar:
\begin{itemize}
	\item{Nycklar - Finns i klassen Key}
	\item{Enkelriktad dörr - Finns i game.goRoom() längst ned}
	\item{Tidsgräns - Finns i klassen CutWrist}
	\item{Transporter room - Jag har ett "blue pill" som portar dig till första rummet om du sväljer det. Skapas i Room men finns även i Item.}
\end{itemize}
	\tableofcontents
\section{Game}
\lstinputlisting{Game.java}
\section{Room}
\listinputlisting{Room.java}
\section{Command}
\listinputlisting{Command.java}
\section{Parser}
\listinputlisting{Parser.java}
\section{Door}
\listinputlisting{Door.java}
\section{Item}
\listinputlisting{Item.java}
\section{ExpiringItem}
\listinputlisting{ExpiringItem.java}
\section{SolidItem}
\listinputlisting{SolidItem.java}
\section{Tool}
\listinputlisting{Tool.java}
\section{Key}
\listinputlisting{Key.java}
\section{Trap}
\listinputlisting{Trap.java}
\section{CutWrist}
\listinputlisting{CutWrist.java}
\end{document}
