\documentclass[a4paper] {article}
\usepackage[utf8]{inputenc}
\title {DD2380 Artificial Intelligence:\\
	Homework 2, Part B, Answers}
\author {Mattias Mikkola, mmikkola@kth.se\\
	Peter Boström, pbos@kth.se}

\begin {document}
\maketitle

\section*{N-Amazons}
	\paragraph{a) CSP constraints}
		\begin{itemize}
			\item $row(a_i) \neq row(a_j)$
			\item $col(a_i) \neq col(a_j)$
			\item $diag\_ascending(a_i) \neq diag\_ascending(a_j)$
			\item $diag\_descending(a_i) \neq diag\_descending(a_j)$
			\item $pos(a_i) \not \in knight\_moves(a_j)$
		\end{itemize}
		For all $i \neq j$.
	\paragraph{b) write a program}
		See \emph{amazon.c}
	\paragraph{c) method}
		We're using a min-conflicts heuristic starting with a randomly permuted board. All pieces are placed with no horizontal or vertical conflicts. The pieces with the most conflicts are moved to the lowest-conflict position inside its column iteratively until all pieces are ok. Among the pieces with least conflicts one is picked randomly. This leads to more variation and doesn't get stuck as easily. Getting stuck in a local optimum is also an issue. That's why we count iterations, and unless the board is solved within \emph{n} iterations, we shuffle the board and begin from the start.

		This random element means that the board is not \emph{guaranteed} to be solvable, but practically is. Larger boards seem to be easier as well. The random element also makes the solve time potentially unpredictable with bad luck. Most practical issues are accounted for. Run tests on this method seems to run in expected time $O(n^2)$, which is considered ok. It solves for $n=10,000$ in about a second and for $n=100,000$ about 2 minutes on this machine, so it's considered to scale pretty well. For larger values of $n$, this would of course become a problem.

		The implementation however, even for n-queens, doesn't come near the book's example of n-queens in which this method would've solved a million-queens problem with 50 iterations.

		A suggestion for improving the performance (not drastically) would be to cache more of the data, so that the number of conflicts doesn't have to be recalculated from the start each iteration.

	\paragraph{d) performance}
		This performance analysis should definitely be taken with a grain of salt. We chose not to do more calculations between 10 and 1000, since they're too fast either way. The script included (\emph{tests.sh}) can be modified to run tests for other values as well, should it be desired.

	\begin{figure}[h!]
		\begin{center}
		\begin{tabular}{|r|r|r|}
			\hline
			n & iterations & time (s) \\
			\hline
			10 & 779 & 0.001 \\
			100 & 390 & 0.002 \\
			500 & 308 & 0.005 \\
			1000 & 507 & 0.013 \\
			2000 & 966 & 0.052 \\
			3000 & 1298 & 0.103 \\
			4000 & 1706 & 0.163 \\
			5000 & 2149 & 0.262 \\
			6000 & 2463 & 0.362 \\
			7000 & 2883 & 0.488 \\
			8000 & 3305 & 0.685 \\
			9000 & 3671 & 0.798 \\
			10000 & 4070 & 1.002 \\
			\hline
		\end{tabular}
		\end{center}
	\end{figure}

\end{document}
