\documentclass{article}
\usepackage[utf8]{inputenc}
\newcommand{\Ordo}{\mathcal{O}}
\begin{document}

\section*{Loopar}

\[Loop 1: \Ordo(n)\]

\[Loop 2: \Ordo(n)\]

\[Loop 3: \Ordo(n^2)\]

\[Loop 4: \Ordo(n^2)\]

\[Loop 5: \Ordo(n^4)\]

\section*{Bevisa $(n+1)^3 \in \Ordo(n^3)$}

\[(n+1)^3 = n^3 + 3n^2 + 3n + 1\]

$f(x) \in \Ordo(n)$ om det finns positiva konstanter $c$ och $n_0$ så att $f(n) \leq cg(n)$ för alla $n \geq n_0$

Antag $c = 4$

\[4n^3 \geq n^3 + 3n^2 + 3n + 1\]

\[3n^3 \geq 3n^2 + 3n + 1\]

\[n^3 \geq n^2 + n + \frac{1}{3}\]

Överstående gäller t.ex. för $n=2$. Kvar att bevisa är att det även gäller för alla $n \geq 2$.

\[n^3 \geq n^2 + n + \frac{1}{3}\]

Detta antags gälla. Vi ska nu bevisa att ekvationen även gäller om vi byter ut $n$ mot $n+1$.

\[(n+1)^3 \geq (n+1)^2 + (n+1) + \frac{1}{3}\]

\[n^3 + 3n^2 + 3n +1 \geq n^2 + 2n + 1 + n + 1 + \frac{1}{3}\]

\[n^3 + 2n^2 \geq \frac{1}{3}\]

Detta gäller åtminstonde för $n+1 \geq 1$. Alltså har vi bevisat att $n^3$ ökar kraftigare än $n^2+n+\frac{1}{3}$ för alla $n \geq 2$. Vi har alltså även bevisat att $f(n) \leq cg(n)$ för alla $n \geq n_0$ där $c = 4$ och $n_0 = 2$. Detta definerar att $(n+1)^3 \in \Ordo(n^3)$.

\section*{Stable Marriage}

Operationerna före huvudloopen är linjära eller konstanta och algoritmen kommer inte att $\in \Ordo(n)$, därför är vi intresserade av while-loopen.

Alla operationer är konstanta inuti loopen, förutom \emph{prefers} som är linjär (for-loop, går i värsta fall igenom n-1 element, innan den hittar x eller y).

prefers är allstå $\Ordo(n)$. Kvar att bestämma är av vilken ordning sjävla while-loopen är.

I bästa fall är den linjär, då varje man frågar rätt kvinna direkt. Prefers kommer då vara konstant eftersom mannen som frågade kommer ligga på konstant förstaplats i kvinnans lista. Men detta var inte vad vi var intresserade av.

n olika män kommer att behöva fråga kvinnor. Som värst kan en man behöva fråga n gånger, dvs. fråga varje kvinna. För varje gång mannen frågar kvinnan görs en jämförelse om kvinnan gillar mannen mest, om kvinnan i fråga redan är gift, för att se vem hon föredrar. Denna jämförelse är som tidigare sagt storleksordningen $\Ordo(n)$.

Loopen är alltså $\Ordo(n^2)$ och i flertalet fall så genomförs en operation som är $\Ordo(n)$. Följer gör att algoritmen är $$\Ordo(n^3)$$

Räknade operationer skulle egentligen vara alla operationer inom for-loopen i \emph{prefers}. De är konstanta, och kommer att köras $\Ordo(n^3)$ gånger. Konstanta operationer inuti while-loopen kommer att köras $\Ordo(n^2)$ gånger och är därför av mindre vikt. Hur många operationer det är inuti prefers spelar egentigen ingen roll för hur programmets ökar, då det bara påverkar körtiden med en koefficient, men det är de som körs $\Ordo(n^3)$ gånger.

\end{document}

