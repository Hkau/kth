% LaTeX Curriculum Vitae Template
%
% Copyright (C) 2004-2009 Jason Blevins <jrblevin@sdf.lonestar.org>
% http://jblevins.org/projects/cv-template/
%
% You may use use this document as a template to create your own CV
% and you may redistribute the source code freely. No attribution is
% required in any resulting documents. I do ask that you please leave
% this notice and the above URL in the source code if you choose to
% redistribute this file.

\documentclass[a4paper]{article}
\usepackage[utf8]{inputenc}


\usepackage{hyperref}
\usepackage{geometry}

% Comment the following lines to use the default Computer Modern font
% instead of the Palatino font provided by the mathpazo package.
% Remove the 'osf' bit if you don't like the old style figures.
\usepackage[T1]{fontenc}
%\usepackage[sc,osf]{mathpazo}

% Set your name here
\def\name{Peter Boström}

% Replace this with a link to your CV if you like, or set it empty
% (as in \def\footerlink{}) to remove the link in the footer:
\def\footerlink{}%http://lemming.ceri.se/cv/}

% The following metadata will show up in the PDF properties
\hypersetup{
  colorlinks = true,
  urlcolor = black,
  pdfauthor = {\name},
  pdfkeywords = {},
  pdftitle = {\name: Curriculum Vitae},
  pdfsubject = {Curriculum Vitae},
  pdfpagemode = UseNone
}

\geometry{
  body={6.5in, 9in},
  left=1.0in,
  top=1.0in
}

% Customize page headers
\pagestyle{myheadings}
\markright{\name}
\thispagestyle{empty}

% Custom section fonts
\usepackage{sectsty}
\sectionfont{\rmfamily\mdseries\Large}
\subsectionfont{\rmfamily\mdseries\itshape\large}

% Other possible font commands include:
% \ttfamily for teletype,
% \sffamily for sans serif,
% \bfseries for bold,
% \scshape for small caps,
% \normalsize, \large, \Large, \LARGE sizes.

% Don't indent paragraphs.
\setlength\parindent{0em}

% Make lists without bullets
\renewenvironment{itemize}{
  \begin{list}{}{
    \setlength{\leftmargin}{1.5em}
  }
}{
  \end{list}
}

\begin{document}

% Place name at left
{\huge \name}

% Alternatively, print name centered and bold:
%\centerline{\huge \bf \name}

\vspace{8mm}

\begin{minipage}{0.45\linewidth}
  Gullmarsvägen 48 \\
  120 39 Årsta \\
\end{minipage}
\begin{minipage}{0.45\linewidth}
  \begin{tabular}{ll}
    Telefon: & 073-693 77 13 \\
    E-mail: & \href{mailto:pbos@kth.se}{\tt pbos@kth.se} \\
  \end{tabular}
\end{minipage}

\section*{Profil}

\begin{itemize}
	\item Social och engagerad 22-årig civilingenjörsstuderande inom datateknik. Läser för nuvarande 3:e året på Kungliga Tekniska Högskolan.
\end{itemize}

\section*{Utbildning}

\begin{itemize}
  \item Civilingenjör Datateknik, Kungliga Tekniska Högskolan -- \emph{2008-}

  \item Naturvetenskapliga programmet, Kärrtorps gymnasium -- \emph{2005-2008}
  \begin{itemize}
  	\item \emph{Inriktning:} Matematik och datavetenskap.
  \end{itemize}
\end{itemize}


\section*{Arbetslivserfarenhet}

\begin{itemize}
	\item Kungliga Tekniska Högskolan -- \emph{Oktober 2009-}
	\begin{itemize}
		\item \emph{Övningsassistent:} Kurshandledning av introduktionskursen i grundläggande programmering i Java och datalogi för datastudenter.
		\item \emph{Laborations- \& övningsassistent:} Kurshandledning av introduktionskurs för Fysikstudenter i grundläggande programmering i Python samt datalogi.
	\end{itemize}

	\item Tobii Technology AB -- \emph{Sommar 2010, deltid nov-dec 2010}
	\begin{itemize}
		\item \emph{Kvalitetssäkring:} Ingått i projektgrupp som utvecklat en handikappsanpassad dator som kommunikationsstöd. Arbetet bestod av att kvalitetssäkra både programvara i datorn och även den egenutvecklade datorn i sig.
	\end{itemize}

%	\item Städpoolen AB -- \emph{Sommar 2008}
%	\begin{itemize}
%		\item \emph{Lokalvårdare:} Trapphusstäd av fastigheter i Stockholm.
%	\end{itemize}

%	\item Martin Olsson Cashar Städpoolen AB -- \emph{Sommar 2008}
%	\begin{itemize}
%		\item \emph{Cashsäljare:} Kassabiträde på Martin Olsson, Storängsbotten.
%	\end{itemize}

\end{itemize}

\section*{Övrig erfarenhet}

\begin{itemize}
	\item Valnämnden, Stockholms Stad -- \emph{2009-06-07}
	\begin{itemize}
		\item \emph{Valförättare:} Röstmottagare under valet till EU-parlamentet 2009.
	\end{itemize}
\end{itemize}

\section*{Språkkunskaper}

\begin{itemize}
	\item Svenska -- \emph{Modersmål}
	\item Engelska -- \emph{Flytande}
	\item Japanska -- \emph{Grundläggande kunskaper}
\end{itemize}

\section*{Referenser}

\begin{itemize}
	\item Referenser finns från Tobii Technology och Kungliga Tekniska Högskolan och kan lämnas på behäran.
\end{itemize}

\bigskip

% Footer
\begin{center}
  \begin{footnotesize}
    Last updated: \today \\
    \href{\footerlink}{\texttt{\footerlink}}
  \end{footnotesize}
\end{center}

\end{document}
