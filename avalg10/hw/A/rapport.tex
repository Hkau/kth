% Att vi vill ha ett dokument som ser ut som en teknisk artikel, på a4-papper, tvåsidigt med 10 punkters font.
\documentclass[a4paper,10pt,titlepage]{article}
% sidmarginaler
\usepackage[inner=3cm,top=3cm,outer=2cm,bottom=3cm]{geometry}
% svenska avstavningsregler
\usepackage[swedish]{babel}
\usepackage[T1]{fontenc}
% teckenencoding
\usepackage[utf8x]{inputenc}

% för import av icke eps-bilder
\usepackage[pdftex]{graphicx}

% mattesymboler
\usepackage{amssymb}

% fina kodlistings
\usepackage{fancyvrb}
\usepackage{listings}

\usepackage{sectsty}
\sectionfont{\normalsize}

% lite inställningar till listings-paketet, bland annat så att den bryter för långa rader
\lstset{
	% vilket språk vi använder i våra kodlistings, så att listings-paketet vet hur den ska highligta saker
%	language=Python, 
	% huruvida vi ska ha syntax highlighting
	fancyvrb=true, 
	% hur stora tabstopp vi ska ha
	tabsize=4, 
	% huruvida vi ska tillåta andra tecken än a-z
	extendedchars=\true
	% hur breda listings vi vill ha (skriv exempelvis linewidth=0.5\textwidth för att få listings som bara tar upp halva bredden av sidan)
	linewidth=\textwidth, 
	% huruvida vi ska visa mellanslag
	showstringspaces=false, 
	% huruvida vi ska bryta rader som är för långa
	breaklines=true, 
	% huruvida den ska få bryta rader mitt i ord eller inte (true här betyder att den bara bryter mellan ord)
	breakatwhitespace=true, 
	% indentera radbrytningar automatiskt
	breakautoindent=true,
	% lägg in radnummer på vänster sida
	numbers=left, 
	% hur stora radnumren ska vara
	numberstyle=\tiny, 
	% hur långt det ska vara mellan radnumren och koden
	numbersep=8pt
}
% stoppa in fina hyperlänkar (som man kan klicka på) i tableofcontents
%\usepackage{hyperref}
%\hypersetup{colorlinks=true, linkcolor=blue}

% Ett litet paket för fin pseudokod
\usepackage{algorithmic}
\usepackage{algorithm}

% stoppa in ditt namn här nedanför
\def\myName{Peter Boström}
\def\myEmail{pbos@kth.se}

% kurskod här (skriver in indans kod som default)
\def\courseCode{DD2440}
% kursnamn här (återigen inda som default)
\def\courseName{Avancerade algoritmer}

% stoppa in numret på inlämningen här nedanför
\def\assignmentNumber{A}

% stoppa in datum för när du skriver den här inlämningen här
\def\writtenDate{2010-10-01}

% fina sidheaders/footers
\usepackage{fancyhdr}
% inställningar till fancyhdr
\pagestyle{fancy}\headheight 13pt
\fancyfoot{}
% sidhuvud, vänster sida, fyll i ditt namn här
\lhead{\courseCode\ - \myName}
% sidhuvud, höger sida, fyll i vilken uppgift detta gäller
\rhead{Uppgift \assignmentNumber}
% sidnumrering på vänster sida för jämna sidnummer, höger sida för ojämna sidnummer
%\fancyfoot[LE,RO]{\thepage}
\fancyfoot[RO]{\thepage}

\title{\courseName\\\vspace{4pt}Uppgift \assignmentNumber}
\date{\writtenDate}
\author{\myName\\\emph{\myEmail}}
\begin{document}
\maketitle

\section*{Assignment}
In the lectures you've seen how to sort n word-sized integers on a unit-cost RAM model in O(n log log n) time. In this homework you will study special cases where it's possible to find easier algorithms or better time bounds. 


\section{Give a short description of the unit-cost RAM model and explain how the word-size w gives an upper bound on the number of integers n in the sorting problem above. (5p)}

The unit-cost RAM model gives a simplified computing model where all operations, arithmetic as well as loading/storing are performed in constant time.

A RAM model using a word size $w$, uses word-sized address pointers as well. Each element to be sorted (or even stored) requires an unique address. A pointer using a word size of $w$ bits, cannot be used to address more than $2^w$ elements. This introduces a limit to the number of elements that can be addressed, and therefore sorted, to $2^w$.


\section {If the maximum element m is O(n) you may sort in linear time using a simple algorithm. Describe how. (5p) }

Counting sort. (Note that all integers are positive or zero.)

Reserve space for $m+1$ elements, initialize all to zero. This is assumed to be done in $O(m)$ and therefore $O(n)$ time. This array, $counts$, will be used to count the number of occurances of each element. Increase $counts[i]$ for each element $i$ in the list. All elements are now represented by the $counts$ array, and the sum of each element in the $counts$ array will be equal to the length of the original array, that is $n$. Now start writing back the counts of the array. Iterate through each possible value $i$ which goes from $0$ to $m$, and insert it into the array $counts[i]$ times. This is an $O(n) + O(m)$ operation, that is $O(n)$.

\section {Give an algorithm that sorts in linear time when the maximum element $m$ is $O(n^k)$, where $k$ is a positive constant. (5p)}

Radix sort. Each number will be looked upon $k$ times and sorted into $n$ different lists depending on their $i$th digit.

\section {Describe an algorithm that sorts in linear time if there are many repeated elements. How many distinct elements can your algorithm handle and how fast is it? (5p)}

\section {The algorithm discussed in class uses large amounts of memory. How much? By doing the radix sorting phase in more than two steps you may reduce the memory requirements while increasing the running time. Explain how to do this and give a formula describing the time-space tradeoff. (5p)}
\end{document}

